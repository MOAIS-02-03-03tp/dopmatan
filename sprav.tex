\documentclass[12pt]{article}
\usepackage[a4paper, total={7in,10in}]{geometry}
\usepackage{polyglossia}
\usepackage{ragged2e}
\usepackage{amsmath}
\usepackage{amssymb}
\usepackage{microtype}
\usepackage{graphicx}
\let\ORIincludegraphics\includegraphics
\renewcommand{\includegraphics}[2][]{\ORIincludegraphics[scale=0.65,#1]{#2}}
\usepackage{changepage}
\usepackage{hyperref}
\usepackage{cancel}
\graphicspath{{./images/}}
\setmainlanguage{russian}
\setotherlanguage{english}
\newfontfamily\russianfont[Script=Cyrillic]{Times New Roman}
\newfontfamily\englishfont{Times New Roman}
\setlength{\parindent}{0em}
\setlength{\parskip}{6pt}

\def\posl#1#2{\{#1_{#2}\}}
\DeclareMathOperator*{\sh-like}{\sinh-like}
\DeclareMathOperator*{\ch-like}{\cosh-like}
\DeclareMathOperator*{\th-like}{\tanh-like}
\DeclareMathOperator*{\cth-like}{\coth-like}
\DeclareMathOperator*{\tg-like}{\tan-like}
\DeclareMathOperator*{\ctg-like}{\cot-like}
\DeclareMathOperator*{\arctg-like}{\arctan-like}
\DeclareMathOperator*{\arcctg-like}{\arctan-like}

\setcounter{section}{7}

\begin{document}
    \justifying
    \begin{titlepage}
        \begin{center}
            \hfill \break
            \large{МИНОБРНАУКИ РОССИИ}\\
            \footnotesize{ФЕДЕРАЛЬНОЕ ГОСУДАРСТВЕННОЕ АВТОНОМНОЕ ОБРАЗОВАТЕЛЬНОЕ УЧРЕЖДЕНИЕ}\\ 
            \footnotesize{ВЫСШЕГО ПРОФЕССИОНАЛЬНОГО ОБРАЗОВАНИЯ}\\
            \small{\textbf{«Дальневосточный федеральный университет»}}\\
            \hfill \break
            \normalsize{ИНСТИТУТ МАТЕМАТИКИ И КОМПЬЮТЕРНЫХ ТЕХНОЛОГИЙ}\\
             \hfill \break
            \normalsize{Департамент программной инженерии и искусственного интеллекта}\\
            \hfill\break
            \hfill \break
            \hfill \break
            \hfill \break
            \large{Лекции 2 курса по дисциплине}\\
            \large{Математический Анализ}\\
            \hfill \break
            \hfill \break
            \hfill \break
            \begin{flushright}
              Подготовлено студентами гр.\\
              Б9123-02.03.03тп\\
              Макевкин C.C.\\
              Тарасенко Т.В.\\
              \hfill \break
            \end{flushright}
            \begin{center}
                Для\\
              Зиновьева Павла Владимировича
            \end{center}
            \hfill \break
            
            \hfill \break
            
            \hfill \break
            \hfill \break
            \end{center}
             
            \hfill \break
             
            \normalsize{ 
            
            }
            \begin{center} Владивосток \\ 2025 \end{center}
            \thispagestyle{empty}
    \end{titlepage}
    \pagebreak
    \tableofcontents
    \pagebreak
    \section{Криволинейные интегралы}
    \subsection{Криволинейные интегралы первого рода}
    \underline{Определение: } Кривая $\overline{r}(t)=x(t)\overline{i}+y(t)\overline{j}+z(t)\overline{c}$ \;$a\leq t \leq b$
    Называется непрерывной гладкой кривой, если x(t),y(t),z(t) непрерывна диф-мы на [a;b] и
    $x(t)^2+y(t)^2+z(t)^2\not = 0$\\
    \underline{Определение: } Кривая называется непрерывной кусочно-гладкой прямой, если она состоит из
    конечного числа гладких прямых  \\
    %%пикча надо нарисовать
    $\sqsupset  $кривая имеет массу $\rho=\frac{kg}{m}$
    \begin{enumerate}
      \item R
      \item В каждой эл-ой $\Delta l$ выберем произвольно $M_i$
      \item Возьмем $\rho(M_i)$
      \item Считаем что на всей $\Delta l_i \; \rho=const=\rho(M_i)$
      \item Составим $\sigma_r=\sum_{i=0}^{n-1} \rho(M_i)\Delta l_i$
      \item $\lim_{\lambda_R \to 0}\sigma_R=w=\int_{(l)}\rho \alpha e$
    \end{enumerate}
    Рассмотрим функцию $Z=f(x,y)$ заданную вдоль непрерывной кусочно-гладкой кривой l\\
    %опять пикча аааа(спиздить)%
    \begin{enumerate}
      \item R
      \item Выберем произвольную (.) $M_i \in \Delta l_i (.)M_i(\xi_i,\eta_i)$
      \item Выберем f($\xi_i,\eta_i$)
      \item Составим $\sigma_R=\sum_{i=0}^{n-1}$
      \item Вычислим $\lim_{\lambda_R \to 0}\sigma_R=\int_{(l)}f(x,y)dl$
    \end{enumerate}
    \underline{Определение: }Если существует конечный предел интегральной суммы $\sigma_R$ независящий от способа разбиения
    кривой и от выбора (.)$M_i(\xi_i,\eta_i)$ то он называется криволинейный интеграл I-рода
    от функции f(x,y) по кривой l \\
    \underline{Замечание:}Если кривая (AB) не замкнута то интеграл по кривой \[\int_{(AB)}f(x,y)dl=
    \int_{(BA)}f(x;y)dl\]\\
    !!! При переходе к определенному интегралу пределы интегрирования ставятся по мере возрастания переменной
    интегрирования\\
    %пикчи опять%
    \underline{Замечание:}Аналогично вводится интеграл по пространственной кривой.\\
    \[\int_{(l)}f(x,y)dl\]
    \subsection{Вычисление криволинейного интеграла I-рода}
    \begin{math}
      l=\begin{matrix}
        x=x(t)\\
        y=y(t)
      \end{matrix} a\leq t \leq b\\
      l(a)=0\\
      l(t_i)=l_i\\
      l(b)=l
    \end{math}\\
    Положение (.) $M_i$ однозначно определяется с помощью длины дуги, отсчитываемой от (.) A\\
    %выпускал егорика спиздить часть надо%
    \begin{enumerate}
      \item Если кривая задана уравнением y=f(x) $a \leq x \leq b$\\
      $\int_{(l)}g(x,y)dl=\int_{a}^{b}g(x,f(x))\underset{dl}{\sqrt{1+f'(x)^2}dx}$
      \item Если кривая задана параметрически\\
      \[\begin{matrix}
        x=x(t)\\
        y=y(t)
      \end{matrix} \; t_1 \leq t \leq t_2 \; \int_{(l)}g(x,y)dl=\int_{t_1}^{t_2}g(x(t),y(t))
      \sqrt{x'(t)^2+y'(t)^2}dt\]
      \item  Если кривая задана $r=r(\varphi) \; \alpha \leq \varphi \leq \beta$\\
      \[\int_{(l)}g(x,y)dl=\int_{\alpha}^{\beta}g(r(\varphi)cos(\varphi),r(\varphi)sin(\varphi))
      \sqrt{r^2(\varphi)+r'(\varphi)^2}d\varphi\] 
    \end{enumerate}
    \subsection{Свойства криволинейных интегралов I-рода}
    \begin{enumerate}
      \item $\int_{(l)}dl=L$
      \item m=$\int_{(l)}\rho dl$
      \item \[x_c=\frac{M_y}{m}=\frac{\int_{(l)}\rho x dl}{\int_{(l)} \rho dl} \\
      y_c=\frac{M_x}{m}=\frac{\int_{(l)}\rho y dl}{\int_{(l)}\rho dl}\]
    \end{enumerate}
    \subsection{Криволинейные интегралы II-рода}
    $\sqsupset$ задана Z=f(x,y), которая определена в каждой (.) l
    \begin{enumerate}
      \item R
      \item $M_i(\xi_i,\eta_i) \in \Delta l_i$
      \item f($M_i$)=f($\xi_i,\eta_i$)
      \item $\sum_{i=0}^{n-1}f(\xi_i,\eta_i)$
      \item $\lim_{\lambda_R \to 0}\sigma_R=\int_{(l)}f(x,y)dx$
    \end{enumerate}
    \underline{Определение: }Если существует конечный предел интегральной суммы $\sigma_R$
    независящий от способа разбиения кривой l и от выбора (.) $M_i$ то он называется криволинейным
    интегралом II-рода от f(x;y) по кривой l\\
    \underline{Замечание:} Аналогично вводится $\int_{(l)}f(x;y)dy$\\
    Если вдоль кривой определены функции P(x,y),Q(x,y) и $\exists \int_{(AB)}P(x,y)dx$ и $\int_{(BA)}Q(x,y)dy$
    то $\int_{(AB)}P(x,y)dx+Q(x,y)dy$ называется кривой интеграл II-рода общего вида\\
    \underline{Замечание:}\[\int_{(AB)}f(x;y)dx=-\int_{(BA)}f(x;y)dx\]
    %дописать не успел%
\end{document}