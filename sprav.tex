\documentclass[12pt]{article}
\usepackage[a4paper, total={7in,10in}]{geometry}

\usepackage{polyglossia}
\usepackage{ragged2e}
\usepackage{amsmath}
\usepackage{amssymb}
\usepackage{microtype}
\usepackage{graphicx}
\usepackage{changepage}
\usepackage{hyperref}
\usepackage{cancel}
\usepackage{wrapfig}
\usepackage{needspace}
\usepackage{mathtools}
\usepackage{tikz}
\usepackage{esint}
\newcommand*\circled[1]{\tikz[baseline=(char.base)]{
    \node[shape=circle, draw, inner sep=1pt, 
        minimum height=12pt] (char) {#1};}}



\let\ORIincludegraphics\includegraphics
\renewcommand{\includegraphics}[2][]{\ORIincludegraphics[scale=0.65,#1]{#2}}
\newcommand{\verteq}{\rotatebox{90}{$\,=$}}
\newcommand{\equalto}[2]{\underset{\scriptstyle\overset{\mkern4mu\verteq}{#2}}{#1}}
\let\oldint\int
\let\oldiint\iint
\let\oldiiint\iiint
\let\oldoint\oint
\let\oldsum\sum
\let\oldlim\lim
\let\oldoiint\oiint
\renewcommand{\int}{\oldint\limits}
\renewcommand{\oint}{\oldoint\limits}
\renewcommand{\iint}{\oldiint\limits}
\renewcommand{\oiint}{\oldoiint\limits}
\renewcommand{\iiint}{\oldiiint\limits}
\renewcommand{\sum}{\oldsum\limits}
\renewcommand{\lim}{\oldlim\limits}

\graphicspath{{./images/}}
\setmainlanguage{russian}
\setotherlanguage{english}
\newfontfamily\russianfont[Script=Cyrillic]{Times New Roman}
\newfontfamily\englishfont{Times New Roman}
\setlength{\parindent}{0em}
\setlength{\parskip}{6pt}

\def\posl#1#2{\{#1_{#2}\}}
\DeclareMathOperator*{\sh-like}{\sinh-like}
\DeclareMathOperator*{\ch-like}{\cosh-like}
\DeclareMathOperator*{\th-like}{\tanh-like}
\DeclareMathOperator*{\cth-like}{\coth-like}
\DeclareMathOperator*{\tg-like}{\tan-like}
\DeclareMathOperator*{\ctg-like}{\cot-like}
\DeclareMathOperator*{\arctg-like}{\arctan-like}
\DeclareMathOperator*{\arcctg-like}{\arctan-like}

\setcounter{section}{7}

\begin{document}
    \justifying
    \begin{titlepage}
        \begin{center}
            \begin{center}
              \includegraphics[scale=0.1]{logo.jpg}
            \end{center}
            \normalsize{МИНИСТЕРСТВО НАУКИ И ВЫСШЕГО ОБРАЗОВАНИЯ РОССИЙСКОЙ ФЕДЕРАЦИИ}\\
            \footnotesize{ФЕДЕРАЛЬНОЕ ГОСУДАРСТВЕННОЕ АВТОНОМНОЕ ОБРАЗОВАТЕЛЬНОЕ УЧРЕЖДЕНИЕ}\\ 
            \footnotesize{ВЫСШЕГО ОБРАЗОВАНИЯ}\\
            \small{\textbf{«Дальневосточный федеральный университет»}}\\
            \noindent\rule{17cm}{0.4pt}\\
            \large{\textbf{ИНСТИТУТ МАТЕМАТИКИ И КОМПЬЮТЕРНЫХ ТЕХНОЛОГИЙ}}\\
             \hfill \break
            \large{\textbf{Департамент программной инженерии и искусственного интеллекта}}\\
            \hfill\break
            \hfill \break
            \hfill \break
            \hfill \break
            \large{\textbf{ЛЕКЦИОННЫЙ МАТЕРИАЛ}}\\
            \begin{center}
              \normalsize{по дисциплине "Математический Анализ"}\\
              \normalsize{по образовательной программе подготовки бакалавров по направлению}\\
              \normalsize{02.03.03тп - Математическое Обеспечение и Администрирование Информационных систем}\\
              \normalsize{семестры обучения III-IV}
            \end{center}
            \hfill \break
            \hfill \break
            \hfill \break
            \begin{flushright}
                \includegraphics[scale=1.5]{rospis.png}
            \end{flushright}
            \hfill \break
            
            \end{center}
            \hfill \break
            \normalsize{ 
            
            }
            \begin{center} Владивосток \\ 2025 \end{center}
            \thispagestyle{empty}
    \end{titlepage}
    \pagebreak
    \tableofcontents
    \pagebreak
  \section{Криволинейные, кратные и поверхностные интегралы}
  \subsection{Криволинейные интегралы I рода}
  \underline{Определение: } Кривая $\overline{r}(t)=x(t)\overline{i}+y(t)\overline{j}+z(t)\overline{k} \hspace{10pt}$
  $a \leq t \leq b$ называется непрерывной гладкой кривой, если x(t),y(t),z(t) непрерывно дифференцируемы
  на [a;b] и $x'(t)^2+y'(t)^2+z'(t)^2 \not = 0$\\
  \underline{Определение: } Кривая называется непрерывной кусочно-гладкой кривой, если она состроит из конечного числа
  гладких кривых.\\
  Рассмотрим непрерывную кусочно-гладкую кривую:\\
  \begin{minipage}{0.45\textwidth} % Левая колонка — для изображения
      \includegraphics[width=\linewidth]{8.1.1.png}
  \end{minipage}%
  \hspace{1em} % Горизонтальный отступ между колонками
  \begin{minipage}{0.65\textwidth} % Правая колонка — для текста
      Пусть кривая имеет массу $\rho=\frac{\mathrm{kg}}{\mathrm{m}}$.
      \begin{enumerate}
        \item R
        \item В каждой элементарной $\Delta l_i$ выберем\\ произвольную $M_i$ 
        \item Вычислим $\rho(M_i)$
        \item Считаем, что на всём $\Delta l_i \rho=const=\rho(M_i)$
        \item Составим $\sigma_R=\sum_{i=0}^{n-1}\rho(M_i)\Delta l_i$
        \item $\lim_{\lambda_R \to 0} \sigma_R = m = \int_{(l)} \rho dl$
      \end{enumerate}
  \end{minipage}
  \vspace{1em}
  \par Рассмотрим функцию Z=f(x,y), заданную вдоль непрерывной кусочно-гладкой кривой l\\

  \begin{minipage}{0.45\textwidth}
    \includegraphics[scale=0.6]{8.1.2.png}
  \end{minipage}
  \hspace{1em} % Горизонтальный отступ между колонками
  \begin{minipage}{0.65\textwidth} % Правая колонка — для текста
      Пусть кривая имеет массу $\rho=\frac{\mathrm{kg}}{\mathrm{m}}$.
      \begin{enumerate}
        \item R
        \item Выберем произвольную (.) $M_i \in \Delta l_i$\\ (.)$M_i (\xi_i;\eta_i)$
        \item Вычислим $f(\xi_i;\eta_i)$
        \item Составим $\sigma_R = \sum_{i=0}^{n-1} f(\xi_i,\eta_i) \Delta l_i$
        \item Вычислим $\lim_{\lambda_R \to 0} \sigma_R = \int_{(l)} f(x;y) dl$
      \end{enumerate}
  \end{minipage}
  \vspace{1em}
  \par
  \underline{Определение: } Если существует конечный предел интегральной суммы $\sigma_R$, не 
  зависящий предел от способа разбиения кривой и выбора $(.) M_i(\xi_i,\eta_i)$, то он называется
  криволинейным интегралом I рода от функции f(x;y) по кривой l.\\
  \[m=\int_{(l)} f(x;y) dl\]\\
  \underline{Замечание:} Если кривая (AB) не замкнута: \[\int_{(AB)} f(x;y)dl = \int_{(BA)} f(x;y)dl\]
  !!! При переходе к определенному интегралу пределы интегрирования ставятся по мере возрастания
  переменной интегрирования.

  \begin{minipage}{0.45\textwidth}
    \includegraphics[scale=0.6]{8.1.3.png}
  \end{minipage}
  \hspace{1em}
  \begin{minipage}{0.65\textwidth}
      \[\int = \int_{(l_1)}+\int_{(l_2)}+\int_{(l_3)}\]\\
      \[\int_{0}^{1}\dots dy+\int_{0}^{1}\dots dx+\int_{0}^{1}\dots dx\]
  \end{minipage}
  \vspace{1em}
  \par
  \underline{Замечание:} Аналогично вводится интеграл по пространственной кривой $\int_{(l)} f(x;y;z)dl$
  \subsection{Вычисление криволинейного интеграла I рода.}
  \[\int_{(l)} f(x;y) dl \hspace{10pt} l: \begin{cases}
    x=x(t) & a\leq t\leq b\\
    y=y(t)
  \end{cases}\] \\
  \begin{minipage}{0.45\textwidth}
    \includegraphics[scale=0.4]{8.2.1.png}
  \end{minipage}
  \hspace{1em}
  \begin{minipage}{0.65\textwidth}
    Положение (.) $M_i$ однозначно определяется с помощью \\
    длины дуги, отсчитываемой от (.)A. $\begin{cases}
      x=x(l)\\
      y=y(l)
    \end{cases}$
  \end{minipage}
  \vspace{1em}
  \par
  \[\int_{(l)}f(x;y)dl=\int_{0}^{l} f(x(l);y(l))dl\]
  \begin{enumerate}
    \item Если кривая задана уравнением y=f(x) $a \leq x \leq b$\\
    \[  \int_{(l)}g(x;y)dl=\int_{a}^{b}g(x,f(x))\sqrt{1+f'(x)^2}dx\]
    \item Если кривая задана параметрически\\
    \[\begin{cases}
      x=x(t)\\
      y=y(t)
    \end{cases} t_1 \leq t \leq t_2 \hspace{20pt} \int_{(l)}g(x;y)dl=\int_{t_1}^{t_2}g(x(t);y(t))
    \sqrt{x'(t)^2+y'(t)^2}dt\]
    \item Если кривая задана $r=r(\varphi) \hspace{20pt} \alpha \leq \varphi \leq \beta$\\
    \[\int_{(l)}g(x;y)dl = \int_{\alpha}^{\beta} g(r(\varphi)\cos(\varphi);r(\varphi)\sin(\varphi))
    \sqrt{r^2(\varphi)+r'(\varphi)^2}d\varphi\]
  \end{enumerate}
  \subsection*{Свойства криволинейных интегралов I-рода:}
  \begin{enumerate}
    \item $\int_{(l)}dl=L$
    \item m=$\int_{(l)}\rho dl$
    \item $x_c=\frac{M_y}{m}=\frac{\int_{(l)}\rho xdl}{\int_{(l)}\rho dl}$ \hspace{20pt}
    $M_y$ - статический момент кривой относительность оси y.\\
    $y_c = \frac{M_x}{m}=\frac{\int_{(l)}\rho ydl}{\int_{(l)}\rho dl}$ \hspace{20pt}
    $M_x$ - статический момент кривой относительно оси X
  \end{enumerate}
  \subsection{Криволинейные интегралы II рода}
  Пусть задана Z=f(x;y), которая определена в каждой (.) l\\
  \begin{minipage}{0.45\textwidth}
    \includegraphics[scale=0.8]{8.3.1.png}
  \end{minipage}
  \hspace{1em}
  \begin{minipage}{0.65\textwidth}
    \begin{enumerate}
      \item R
      \item $M_i(\xi_i,\eta_i) \in \Delta l_i$
      \item $f(M_i)=f(\xi_i;\eta_i)$
      \item $\sum_{i=0}^{n-1}f(\xi_i;\eta_i)\Delta x_i=\sigma_R$
      \item $\lim_{\lambda_R \to 0}\sigma_R=\int_{(l)}f(x;y)dx$
    \end{enumerate}
  \end{minipage}
  \vspace{1em}
  \par
  \underline{Определение: } Если существует конечный предел суммы $\sigma_R$, не зависящий предел от способа
  разбиения кривой l и выбора (.) $M_i(\xi_i;\eta_i)$, то он называется криволинейным интегралом
  II рода от функции f(x;y) по кривой l\\
  \underline{Замечание:} Аналогично вводится\\
  \[\int_{(l)}f(x;y)dy\]\\
  Если вдоль кривой определенны функции $P(x;y),Q(x;y)$ и существует $\int_{(AB)}P(x;y)dx$ и
  $\int_{(AB)}Q(x;y)dy$, то $\int_{(AB)}P(x;y)dx+Q(x;y)dy$ называется криволинейным интегралом
  II рода общего вида.\\
  \underline{Замечание:} \[\int_{(AB)}f(x;y)dx=-\int_{(BA)}f(x;y)dx\]\\
  Аналогично вводится: \[\int_{(AB)}P(x;y;z)dx+Q(x;y;z)dy+R(x;y;z)dz\]
  \subsection{Существование и вычисление криволинейного интеграла II рода}
  \subsubsection*{Теорема 8.4.1}\label{th:8.4.1}
  \par\noindent
  Пусть кривая AB задана параметрически:\\
  $\begin{cases}
    x=\varphi(t) \hspace{20pt} \varphi(t),\psi(t) \text{ непрерывны } \forall t \in [a;b]\\
    y=\psi(t)
  \end{cases}$\\
  Пусть f=f(x;y) непрерывна вдоль кривой AB. $\varphi'(t)$ существует и непрерывна $\forall t \in [a;b]$.
  Тогда существует криволинейный интеграл $\int_{(AB)}f(x;y)dx=\int_{a}^{b}f(\varphi(t);\psi(t))\varphi'(t)dt$\\
  \underline{Доказательство:}
  \begin{adjustwidth}{1.5em}{1.5em}
    \begin{minipage}{0.45\textwidth}
      \includegraphics[scale=0.6]{8.4.1.png}
    \end{minipage}
    \hspace{1em}
    \begin{minipage}{0.55\textwidth}
      \begin{enumerate}
        \item Произведем разбиение R кривой (AB) точками $A_i(\varphi(t_i);\psi(t_i))$
        \item Выберем (.) $M_i(\varphi(\tau_i);\psi(\tau_i))$
        \item $\Delta x_i = \varphi(t_{i+1})-\varphi(t_i)=\int_{t_i}^{t_{i+1}}\varphi'(t)dt$
        \item $\sigma_R=\sum_{i=0}^{n-1}f(\varphi(\tau_i);\psi(\tau_i))\Delta x_i=\\=\sum_{i=0}^{n-1}
        f(\varphi(\tau_i);\psi(\tau_i))\int_{t_i}^{t_{i+1}}\varphi'(t)dt$
      \end{enumerate}
    \end{minipage}
    \vspace{1em}
    \par
  \end{adjustwidth}
  Рассмотрим правую часть: $\int f(\varphi(\tau_i);\psi(\tau_i))\varphi'(t)dt = \sum_{i=1}^{n}
    \int_{t_i}^{t_{i+1}}f(\varphi(\tau_i);\psi(\tau_i))\varphi'(t)dt=I$\\
  Рассмотрим $|\sigma_R-I|=\sum_{i=1}^{n} \int_{t_i}^{t_{i+1}} [f(\varphi(\tau_i);\psi(\tau_i))-
  f(\varphi(t);\psi(t))]\varphi'(t)dt \boxed{<}$\\
  Т.к. f(x;y) непрерывна вдоль кривой (AB)\\
  \[\forall \varepsilon >0 \exists \delta = \delta(\varepsilon):|\Delta t_i| < \delta \Rightarrow
  |f(\varphi(t_{i+1});\psi(t_{i+1}))-f(\varphi(t_i);\psi(t_i))|<\varepsilon\]\\
  или $\tau_i \in [\tau_i;\tau_{i+1}] \hspace{20pt} \varphi'(t)$ непрерывна на $[t_i;t_{i+1}] \Rightarrow$ она ограничена на нём.\\
  Пусть $\forall i |\varphi'(t_i)|<L$\\
  \boxed{<} $\varepsilon L \sum_{i=1}^{n} \int_{t_i}^{t_{i+1}}dt=\varepsilon L(b-a)\Rightarrow \lim_{\lambda_R \to 0}\sigma_R=I$
  \begin{center}
    \textbf{Ч.т.д.}
  \end{center}
  Аналогично: \[\int_{(AB)}f(x;y)dy=\int_{a}^{b}f(\varphi(t);\psi(t))\psi'(t)dt\]\\
  Если кривая задана как y=y(x) \hspace{10pt} $a\leq x\leq b$. Тогда\\
  $\begin{cases}
    y=y(t)\\
    x=t
  \end{cases} \hspace{20pt} a\leq t\leq b \Rightarrow \int_{(AB)}f(x;y)dx =\int_{a}^{b} f(t;y(t))dt=
  |x=t|=\int_{a}^{b}f(x;y(x))dx$\\
  \underline{Замечание:} $\int_{(AB)}f(x;y)dx=0,$ если AB-прямоугольный отрезок || оси OY. Аналогично \\
  $\int_{(CD)}f(x;y)dy=0,$ если CD-отрезок || OX.\\
  Если L-замкнутый контур.\\
  \begin{minipage}{0.45\textwidth}
    \includegraphics[scale=0.6]{8.4.2.png} 
  \end{minipage}
  \hspace{1em}
  \begin{minipage}{0.25\textwidth}
    \[\int_{(l)}=\int_{(AB)}+\int_{(BA)}\]
  \end{minipage}
  \vspace{1em}
  \par
  Среди двух возможных обходов, в качестве положительного обхода, выбирается обход против часовой стрелки, то есть при обходе
  рассматриваемая область находится слева.

  \subsection{Вычисление площади криволинейной трапеции с помощью криволинейного интеграла.}
  \begin{minipage}{0.45\textwidth}
    \includegraphics[scale=0.6]{8.5.1.png} %TODO:стрелку справа 
  \end{minipage}
  \hspace{1em}
  \begin{minipage}{0.55\textwidth}
    \begin{center}
      Найти S(PSRQ)
    \end{center}
    \[S=\int_{a}^{b}y_2(x)dx-\int_{a}^{b}y_1(x)dx \boxed{=}\]
     \[\text{Рассмотрим} \int_{a}^{b}y_2(x)dx=\int_{(SR)}ydx\]
      \[\text{Рассмотрим}\int_{a}^{b}y_1(x)dx=\int_{(PQ)}ydx\]
  \end{minipage}
  \vspace{1em}
  \[\boxed{=} \int_{(SR)}ydx-\int_{(PQ)}ydx=\int_{(SR)}ydx+\int_{(QP)}ydx+\int_{(PS)}ydx+\int_{(RQ)}ydx=\int_{(PSRQP)}ydx \boxed{=}\]
  \par
  Пусть L=(PQRSP) - контур взятый в положительном направлении. $\boxed{=}-\int_{(L)}ydx=S$\\
  Аналогично, если рассмотреть:
  \begin{flushleft}
    \includegraphics[scale=0.4]{8.5.2.png}
  \end{flushleft}
  То получим S=$\int_{(l)}xdy$\\
  Для произвольного замкнутого контура L\\
  \[\boxed{S=\frac{1}{2}\int_{(l)}xdy-ydx}\]
  \underline{Замечание:}
  \begin{flushleft}
      \includegraphics{8.5.3.png}
  \end{flushleft}
  \subsection{Связь между криволинейными интегралами I и II рода}
  Пусть кривая задана в виде $\overline{r}=\overline{r}(t) \hspace{20pt} a\leq t \leq b$\\
  \begin{flushright}
    $\overline{r}=\overline{r}(x(t^*);y(t^*))$
  \end{flushright}
  \begin{figure}[h!]
    \centering
    % Первая колонка (левая картинка)
    \begin{minipage}{0.45\textwidth}
        \centering
        \includegraphics[width=\textwidth]{8.6.1.png} % путь к первой картинке %TODO стрелка сверху
        \vspace{0.5em} % небольшой отступ между картинкой и текстом
        $\Delta \overline{r}=\overline{r}(t_o+\Delta t)-\overline{r}(t_0)$\\
        $\frac{\Delta \overline{r}}{\Delta t}=\frac{1}{\Delta t}*\Delta \overline{r}$\\
        $\lim_{\Delta t \to 0} \frac{\Delta \overline{r}}{\Delta t}=r(t_0)$\\
        $|\overline{r}'(t)|=l'(t)$
    \end{minipage}
    \hfill
    % Вторая колонка (правая картинка)
    \begin{minipage}{0.45\textwidth}
        \centering
        \includegraphics[width=\textwidth]{8.6.2.png} % путь ко второй картинке TODO dr/dt сверху где дельты
        \vspace{0.5em} % небольшой отступ между картинкой и текстом
        $\overline{r}'(t)=\overline{x'(t);y'(t)}$\\
        $|\overline{r}'(t)=\sqrt{x'(t)^2+y'(t)^2}|$\\
        $dl=\sqrt{x'(t)^2+y'(t)^2}dt|:dt$\\
        $l'(t)=\sqrt{x'(t)^2+y'(t)^2}$
    \end{minipage}
  \end{figure}
  \par
  Пусть l-переменная длина дуги\\
  \[\frac{d \overline{r}}{dl} = \frac{d \overline{r}}{dt}* \frac{dt}{dl}=\frac{\frac{d \overline{r}}{dt}}{\frac{dl}{dt}}
  \hspace{20pt} \Big| \frac{d \overline{r}}{dl} \Big| = \Big| \frac{r'(t)}{l'(t)} \Big| = 1\]\\
  \underline{Замечание:} Если $|\overline{a}|=1 \Rightarrow \overline{a}(\cos(\alpha),\cos(\beta),\cos(\gamma))
  \hspace{20pt} \cos^2 \alpha+\cos^2 \beta+\cos^2 \gamma=1$\\
  \[\frac{d \overline{r}}{dl}=(\overline{\cos \alpha; \cos \beta})=(\overline{\frac{dx}{dl};\frac{dy}{dl}})\]
  \[\frac{dx}{dl}=\cos \alpha \hspace{20pt} dx=\cos \alpha dl\]
  \[\frac{dy}{dl}=\cos \beta \hspace{20pt} dy=\cos \beta dl\]
  Рассмотрим $\int_{(l)}f(x;y)dx=\int_{(l)}f(x;y)\cos \alpha dl$\\
  В общем случае $\int_{(l)}P(x;y;z)dx+Q(x;y;z)dy+R(x;y;z)dz=$\\
  \[\int_{(l)} [P(x;y;z)\cos \alpha + Q(x;y;z)\cos \beta + R(x;y;z)\cos \gamma]dl\]
  \subsection*{Физический смысл криволинейным интеграла II рода}
  Рассмотрим $\int_{(l)} Pdx+Qdy+Rdz$\\
  Пусть F(P;Q;R) - сила, под действием которой (.) перемещается по кривой l.\\
  d $\overline{S}(dx;dy;dz)$\\
  Рассмотрим $\overline{F}*\overline{ds}=A$\\
  Криволинейный интеграл II рода равен работе силы F по перемещению (.) по кривой l.
  \subsection{Условие независимости криволинейного интеграла от путей интегрирования.}
  Рассмотрим $\begin{matrix}
    P=P(x;y)\\
    Q=Q(x;y)
  \end{matrix}$\\
  Рассмотрим $\int_{(AB)}Pdx+Qdy \boxed{1}$\\
  Найти условия, при которых значение интеграла не зависит от вида кривой (AB) и однозначно определяется положением
  (.) A и (.) B\\
  Рассмотрим Pdx+Qdy \hspace{20pt} dF = $\frac{\delta F}{\delta x}dx + \frac{\delta F}{\delta y}dy$\\
  Если $P=\frac{\delta F}{\delta x} \hspace{20pt} Q=\frac{\delta F}{\delta y}$, то Pdx+Qdy является полными
  дифференциалом от некоторой F(x;y).\\
  \subsubsection*{Теорема 8.7.1}\label{th:8.7.1}
  \par\noindent
  Для того, чтобы (1) не зависел от пути интегрирования
  $\Leftrightarrow$ чтобы Pdx+Qdy было в рассматриваемой области дифференциалом от некоторой функции 
  F(x;y)\\
  \underline{Доказательство:}
  \begin{adjustwidth}{1.5em}{1.5em}
    $\Rightarrow$ пусть \boxed{1} не зависит от путей интегрирования \\
    \hspace{1em}
    \begin{minipage}{0.55\textwidth}
      \[\int_{(AB)} Pdx+Qdy = \int_{(x_0;y_0)}^{(x_1;y_1)}Pdx+Qdy\]\\
      \[\text{Рассмотрим F(x;y)=}\int_{(x_0;y_0)}^{(x;y)}Pdx+Qdy\]
    \end{minipage}
    \vspace{1em}
    \begin{minipage}{0.45\textwidth}
      \includegraphics[scale=0.6]{8.7.1.png}
    \end{minipage}
    \par
    \[F(x_1;y_1)=\int_{(x_0;y_0)}^{(x_1;y_1)}Pdx+Qdy\]\\
    \[F(x_1+\Delta x;y_1)=\int_{(x_0;y_0)}^{(x_1+\Delta x;y_1)}Pdx+Qdy=\int_{(AB)}Pdx+qdy+
    \int_{(BC)}Pdx+Qdy\]\\
    Рассмотрим $F(x_1+\Delta x,y_1)-F(x_1;y_1)=\int_{(BC)}Pdx=\int_{x_1}^{x_1+\Delta x}P(x;y_1)dx=
    \Big| \text{\underline{Замечание:}} \int_{a}^{b}f(x)dx=f(\xi)(b-a)\Big| = \underset{\xi \in (x_1;x_1+\Delta x)}{P(\xi;y_1)}
    \Delta x = \underset{0<\theta<1}{P(x_1+\theta \Delta x;y_1)}\Delta x$\\
    \[\frac{F(x_1+\Delta x;y_1)-F(x_1;y_1)}{\Delta x} = P(x_1+\theta\Delta x;y_1)\]\\
    \[\lim_{\Delta x \to 0} \frac{F(x_1+\Delta x;y_1)-F(x_1;y_1)}{\Delta x}=\frac{\delta F}{\delta x} \Big|_{(x_1;y_1)}
    =P(x;y)\Big|_{(x_1;y_1)}\]\\
    \[\forall x \in D \hspace{20pt} \frac{\delta F}{\delta x}=P(x;y)\]\\
    Аналогично $\frac{\delta F}{\delta y}=Q(x;y)$\\
    $\Leftarrow$ Пусть Pdx+Qdy=$\frac{\delta F}{\delta x}+\frac{\delta F}{\delta y}dy$\\
    \[\int_{(l)}Pdx+Qdy= \begin{vmatrix}
      x=\varphi(t) & \varphi(a)=x_a & \varphi(b)=x_b\\
      y=\psi(t) & \psi(a)=y_a & \psi(b)=y_b  \\
      dx=\varphi'(t)dt\\
      dy=\psi'(t)dt
    \end{vmatrix} = \int_{a}^{b}[P(\varphi(t);\psi(t))\varphi'(t)+Q(\varphi(t);\psi(t))\psi'(t)]dt=\]\\
    \[=\int_{a}^{b}[\frac{\delta F}{\delta x}\varphi'(t)+\frac{\delta F}{\delta y}\psi'(t)]dt=\int_{a}^{b}
    \frac{dF(\varphi(t);\psi(t))}{dt}dt=F(\varphi(t);\psi(t))\Big|^b_a= F(x_b;y_b)-F(x_a;y_a)\]
    \begin{center}
      \textbf{Ч.т.д.}
    \end{center}
  \end{adjustwidth}
  \begin{center}
    Как установить, является ли выражение Pdx+Qdy полным дифференциалом?\\
    $Pdx+Qdy \overset{?}{=} dF$\\
    Если $\begin{matrix}
      P=\frac{\delta F}{\delta x} & Q=\frac{\delta F}{\delta y}\\
      \frac{\delta P}{\delta y}=\frac{\delta^2 F}{\delta y \delta x} & \frac{\delta Q}{\delta x}=\frac{\delta^2 F}{\delta x \delta y}
    \end{matrix} \Rightarrow \frac{\delta P}{\delta y}= \frac{\delta Q}{\delta x}$
  \end{center}
  
  
  \begin{minipage}{0.55\textwidth}
    \underline{Вопрос}: Как найти функцию F(x;y)?\\
    \[\int_{(l)}dF=F(x;y)+C\]\\
    Зафиксируем (.)$(x_0;y_0)$
  \end{minipage}
  \hspace{1em}
  \begin{minipage}{0.45\textwidth}
    \includegraphics[scale=0.6]{8.7.2.png}
  \end{minipage}
  \vspace{1em}
  \par
  $\begin{matrix}
    \text{(x,y) - произвольная (.) области } & \hspace{20pt} & l_1: y=y_0 & dy=0\\
    \text{Обозначим её} (x^*;y^*) & \hspace{20pt} & l_2:x=x^x & dx=0
  \end{matrix}$\\
  \[\int_{(l)}Pdx+Qdy=\int_{(l_1)}Pdx+Qdy+\int_{(l_2)}Pdx+Qdy+C = \int_{x_0}^{x^*}P(x;y_0)dx+
  \int_{y_0}^{y^x}Q(x^*;y)dy+C=F(x^*;y^*)\]
  \subsection{Интеграл по замкнутому контуру}
  Рассмотрим $\int_{(l)}Pdx+Qdy$ L-замкнутый контур в положительном направлении.
  \begin{center}
    \underline{Вопрос:} При каких условиях интеграл по замкнутому контуру обращается в ноль?
  \end{center}
  \subsubsection*{Теорема 8.8.1}\label{th:8.8.1}
  \par\noindent
  Если $\int_{(AB)}Pdx+Qdy$ не зависит от пути интегрирования, то $\int_{(l)}Pdx+Qdy=0.$Верно и обратное утверждение.\\
  \begin{minipage}{0.25\textwidth}
    \includegraphics[scale=0.5]{8.8.1.png}
  \end{minipage}
  \hspace{1em}
  \begin{minipage}{0.75\textwidth}
    \underline{Доказательство:}\\
      $\Rightarrow : \int_{(AB)}Pdx+Qdy$ не зависит от пути интегрирования.\\
      $\int_{(AMB)}=\int_{(AMB)} \Rightarrow \int_{(AMB)}-\int_{(AMB)}=0 \Rightarrow
      \Rightarrow \int_{(AMB)}+\int_{(BNA)}=0 \Rightarrow \int_{(L)}=0$
  \end{minipage}
  \vspace{1em}
  \par
  $\Leftarrow:$ Пусть $\int_{(L)}=0$
  \begin{center}
    $\int_{(AMB)}+\int_{(BNA)}=0$\\
    $\int_{(AMB)}=-\int_{(BNA)}$\\
    $\int_{(AMB)}=\int_{(ANB)}$\\
  \end{center}
  \begin{center}
    \textbf{Ч.т.д.}
  \end{center}
  \underline{Замечание:} Существует область, для которых условие $\frac{\delta P}{\delta y}=\frac{\delta Q}{\delta x}$
  не является достаточным для того, чтобы криволинейный интеграл не зависел от пути интегрирования.\\
  \par
  \underline{Определение: } Область, для которой $\forall$ расположенный в ней замкнутый контур можно путём
  непрерывной деформации стянуть в (.), не выходя за пределы области, называется односвязной. В противном
  случаем - неодносвязной.\\
  \par
  \begin{minipage}{0.5\textwidth}
    односвязный\\
    \includegraphics[scale=0.6]{8.8.2.png}
  \end{minipage}
  \hspace{1em}
  \begin{minipage}{0.5\textwidth}
      не односвязный\\
    \includegraphics[scale=0.6]{8.8.3.png}
  \end{minipage}
  \vspace{1em}
  \par
  \textbf{Утверждение:} Для того, чтобы\\
  \[\oint_{(L)} Pdx+Qdy=0 \text{ необходимо, а в случае односвязной области достаточно, чтобы } \frac{\delta P}{\delta y}=\frac{\delta Q}{\delta x}\]
  Рассмотрим контур, имеющий внутри себя особую точку\\
  \textbf{Утверждение:} Пусть $\frac{\delta P}{\delta y}=\frac{\delta Q}{\delta x}$. Тогда
  \begin{enumerate}
    \item $\oint_{L}Pdx+Qdy$ может быть отличным от нуля.
    \item Все интегралы взятые в "+" направлении, по $\forall$ замкнутому контуру, будут равны между собой.
  \end{enumerate}
  \underline{Доказательство:}\\
  2)
  \begin{minipage}{0.45\textwidth}
    \includegraphics[scale=0.6]{8.8.4.png}
  \end{minipage}
  \hspace{1em}
  \begin{minipage}{0.55\textwidth}
    Рассмотрим \\
    \[\begin{matrix}
      \int_{(A_1M_1B_1)}+\int_{B_1B_2}+\int_{B_2M_2A_2}+\int_{A_1A_1} = 0\\
      \int_{B_1N_1A_1}+\int_{A_1A_2}+\int_{A_2N_2B_2}+\int_{B_2B_1} =0 
    \end{matrix} \Bigg\} \Rightarrow\] 
    \[\Rightarrow \int_{A_1M_1B_1N_1A_1} + \int_{B_2M_2N_2N_2B_2} =0 \]
    \[\int_{A_1M_1B_1N_1A_1}=-\int_{B_2M_2A_2N_2B_2}\]
    \[\int_{(l_1)}=\int_{(l_2)}\]
  \end{minipage}
  \vspace{1em}
  \begin{center}
    \textbf{Ч.т.д.}
  \end{center}
  Рассмотрим $\oint_{(L)} Pdx+Qdy=\sigma$\\
  \begin{minipage}{0.45\textwidth}
    \includegraphics[scale=0.6]{8.8.5.png}
  \end{minipage}
  \hspace{1em}
  \begin{minipage}{0.4\textwidth}
    \[\int_{(AB)}Pdx+Qdy+\int_{(BA)}Pdx+Qdy=n\sigma\]
    \[\int_{(AB)}Pdx+Qdy=\int_{(AB)}Pdx+Qdy+n\sigma\]
  \end{minipage}
  \vspace{1em}
  \par
  То есть интеграл зависит от пути интегрирования. В смысле прибавления кратного числа к циклической
  постоянной.\\
  Присоединив к кривой AB некоторое число петель, окружающих особую (.), можно добиться, чтобы криволинейный
  интеграл принял значение, близкое к заданному.\\
  \underline{Пример:} $\oint_{(L)} \frac{xdy-ydx}{x^2+y^2} \boxed{=} \hspace{20pt} L:x^2+y^2=a^2, r=a$\\
  \begin{minipage}{0.2\textwidth}
    \includegraphics[scale=0.4]{8.8.6.png}
  \end{minipage}
  \hspace{1em}
  \begin{minipage}{0.8\textwidth}
    $P=\frac{-y}{x^2+y^2} \hspace{20pt} Q=\frac{x}{x^2+y^2} \hspace{10pt}$
    $\begin{matrix}
      x=r(\varphi)\cos(\varphi)=a\cos(\varphi) & dx=-a\sin(\varphi)d\varphi\\
      y=a\sin(\varphi) & dy=a\cos(\varphi)d\varphi
    \end{matrix}$
  \end{minipage}
  \vspace{1em}
  \par
  \[\boxed{=} \int_{0}^{2\pi} \frac{a^2\cos^2(\varphi)+a^2\sin^2(\varphi)}{a^2}d\varphi = 
  \int_{0}^{2\pi}d\varphi=\varphi \Big|^{2\pi}_0 = 2\pi = \sigma\]
  \begin{center}
    В общем случае $\oint_{(L)} Pdx+Qdy=n_1\sigma_1+n_2\sigma_2+\dots+n_k\sigma_k$
  \end{center}
  \subsection{Трёхмерный случай}
  Рассмотрим $\int_{(AB)}Pdx+Qdy+Rdz$\\
  Он не зависит от пути интегрирования, если существует F(x;y;z):
  \[\frac{\delta F}{\delta x}=P;\frac{\delta F}{\delta y}=Q;\frac{\delta F}{\delta Z}=R\]

  \[\text{Пусть существует непрерывные производные } \frac{\delta P}{\delta y};
  \frac{\delta P}{\delta z};\frac{\delta Q}{\delta x};\frac{\delta Q}{\delta z}
  ;\frac{\delta R}{\delta x};\frac{\delta R}{\delta y}\]

  \[\frac{\delta P}{\delta y}=\frac{\delta^2 F}{\delta y \delta x};
  \frac{\delta P}{\delta z}=\frac{\delta^2 F}{\delta z \delta x};
  \frac{\delta Q}{\delta x}=\frac{\delta^2 F}{\delta x \delta y};
  \frac{\delta Q}{\delta z}=\frac{\delta^2 F}{\delta z \delta y};
  \frac{\delta R}{\delta x}=\frac{\delta^2 F}{\delta x \delta z};
  \frac{\delta R}{\delta y}=\frac{\delta^2 F}{\delta y \delta z}\]

  \[\begin{matrix}
    \frac{\delta P}{\delta y} = \frac{\delta Q}{\delta x}\\
    \par \\
    \frac{\delta P}{\delta z} = \frac{\delta R}{\delta x}\\
    \par \\
    \frac{\delta Q}{\delta z} = \frac{\delta R}{\delta y}
  \end{matrix} \Bigg\} \text{- условие независимости криволинейного интеграла от пути интегрирования}\] 
  $P=\frac{\delta F}{\delta x}$\\
  Рассмотрим $\int_{x_0}^{x} P(x;y;z)dx=\int_{x_0}^{x}\frac{\delta F}{\delta x}dx = F(x;y;z)-F(x_0;y;z)$\\
  Зафиксируем $x_0. \int_{y_0}^{y}Q(x_0;y;z)dy=\int_{y_0}^{y}\frac{\delta F}{\delta y|_{x=x_0}}
  dy=F(x_0;y;z)-F(x_0;y_0;z)$\\
  Зафиксируем $x_0,y_0. \int_{z_0}^{z}R(x;y;z)dz=\int_{z_0}^{z}\frac{\delta F}{\delta z |_{x_0,y_0}}dz=
  F(x_0;y_0;z)-F(x_0;y_0;z_0)$
  \[F(x;y;z)=\int_{x_0}^{x}P(x;y;z)dx+\int_{y_0}^{y}Q(x_0;y;z)dy+\int_{z_0}^{z}R(x_0;y_0;z)dz+\equalto{F(x_0;y_0;z_0)}{C}\]
  \subsection{Кратные интегралы. Двойной интеграл в прямоугольной области.}
  \begin{minipage}{0.25\textwidth}
    Рассмотрим P: $\begin{matrix}
      a\leq x \leq b\\
      c \leq y \leq d
    \end{matrix}$
  \end{minipage}
  \hspace{1em}
  \begin{minipage}{0.45\textwidth}
    \includegraphics[scale=0.6]{8.10.1.png}
  \end{minipage}
  \vspace{1em}
  \par
  \begin{enumerate}
    \item R: $\begin{matrix}
      a=x_0 < x_1 < \dots < x_n =b\\
      c=y_0 <y_1 < \dots < y_n =d
    \end{matrix}$
    \item Выберем в каждом элементарном $P_{ij}$ произвольную (.) $(\xi_i;\eta_i)$
    \item Рассмотрим Z=f(x;y). Вычислим $f(\xi_i;\eta_i)$
    \item Вычислим $f(\xi_i;\eta_i)S_{P_{ij}}$
    \item Вычислим $\sum_{i=0}^{n-1} \sum_{j=0}^{m-1} f (\xi_i;\eta_i)S_{P_{ij}}$
    \item Обозначим $\lambda_R = max(d_{ij})$
    \item $\lim_{\lambda_R \to 0}\sigma_R=\iint_P f(x;y)dxdy$
  \end{enumerate}
  \underline{Определение: } Если существует конечный $\lim_{\lambda_R \to 0} \sigma_R$, не зависящий
  от способа разбиения R и выбора (.) $(\xi_i;\eta_i)$, то он называется двойным интегралом от f(x;y)
  по прямоугольнику P.\\
  \par\noindent
  \underline{Определение: } I называется двойным интегралом от f(x;y) по P, если 
  \[\forall \varepsilon > 0 \exists \delta =\delta(\varepsilon): \lambda_r < \delta \Rightarrow |I-\sigma_R| < \varepsilon\]

  \subsection{Суммы Дарбу}
  \[\overline{S}=\sum_{i=0}^{n-1} \sum_{j=0}^{m-1} M_{ij}\Delta x_i \Delta y_i = \sum_{i=1}^{n-1} \sum_{j=0}^{n-1}M_{ij}\Delta P_{ij}\]
  \[\underline{S}=\sum_{i=0}^{n-1} \sum_{j=0}^{m-1} m_{ij}\Delta x_i \Delta y_i = \sum_{i=1}^{n-1} \sum_{j=0}^{n-1}m_{ij}\Delta P_{ij}\]
  \[m_{ij} \leq f(x;y) \leq M_{ij} \hspace{20pt} \forall(x;y)\in P_{ij}\]
  \[M_{ij}=\underset{(x;y) \in \Delta P_{ij}}{sup f(x;y)} \hspace{20pt} m_{ij}=\underset{(x;y) \in \Delta P_{ij}}{inf f(x;y)} \]
  \subsection*{Свойства}
  \begin{enumerate}
    \item $\underline{S}\leq \sigma_R \leq \overline{S}, \forall R$
    \item Пусть $R \sqsubset R'$\\
    $\overline{S_{R'}}\leq \overline{S_{R}} \hspace{20pt} \underline{S_{R'}}\geq \underline{S_{R}}$
    \item $\forall R_1,R_2 \hspace{20pt} \underline{S_{R_1}}\leq \overline{S_{R_2}}$
    \item Множество верхних сумм Дарбу ограничено снизу.\par
    Множество нижних сумм Дарбу ограниченно сверху.\\
    $I_*=\underset{R}{sup \{\underline{S}\}} \hspace{20pt} I^*=\underset{R}{inf \{\underline{S}\}}$
    \item Th. о существовании интеграла\\
    f(x;y) интегрируема на P $\Leftrightarrow \lim_{\lambda_R \to 0} (\overline{S_R}-\underline{S_R})=0$\\
    $I_*=I^*=I$
  \end{enumerate}

  \subsection{Двойной интеграл в произвольной области}
  Пусть $\underset{\text{(квадратируема по Жордану)}}{\text{область D имеет конечную площадь}}$\\
  \begin{minipage}{0.45\textwidth}
    \includegraphics[scale=0.7]{8.12.1.png}
  \end{minipage}
  \hspace{1em}
  \begin{minipage}{0.55\textwidth}
    Пусть F(x;y)=$\begin{cases}
      f(x;y) & (.)(x;y) \in D\\
      0 & (.)(x;y) \not \in D
    \end{cases}$\\
    Если F(x;y) интегрируема в Р, то существует: \[\iint_{P} F(x;y)dxdy \]
    \[\iint_{P} F(x;y dxdy=\iint_{D} f(x;y)dxdy)\]
  \end{minipage}
  \vspace{1em}
  \par
  \begin{minipage}{0.45\textwidth}
    \includegraphics[scale=0.6]{8.12.2.png}
  \end{minipage}
  \hspace{1em}
  \begin{minipage}{0.55\textwidth}
    \begin{enumerate}
      \item Разобьем D спрямляемыми кривыми(кривая имеющая конечную длину)
      \item Выберем (.) $((\xi_i;\eta_i)) \in D_i$
      \item Вычислим $f((\xi_i;\eta_i))\Delta D_i$
      \item Составим $\sigma_R=\sum_{i=0}^{n-1}f((\xi_i;\eta_i))\Delta D_i$
      \item Пусть $\lambda_R = \underset{M_1,M_2 \in D_i}{\max \rho(M_1,M_2)}$
      \item Вычислим $\lim_{\lambda_R \to 0}\sigma_R=\iint_{D}f(x;y)dxdy$
    \end{enumerate}
  \end{minipage}
  \vspace{1em}
  \break
  \underline{Определение: } Если существует конечный предел $\sigma_R$, при $\lambda_R \to 0$, не
  зависящий от способа разбиения области D и выбора (.) $(\xi_i;\eta_i)$, то он называется двойным
  интегралом от функции f(x;y) по области D\\
  \subsection*{Свойства:}
  \begin{enumerate}
    \item $\iint_D dxdy=S_p$
    \item Пусть $D=D_1 \lor D_2: D_1\land D_2 =0$\\
    Если f(x;y) интегрируема в D, то она интегрируема в $D_1$ и $D_2$
    \[\iint_D f(x;y) dxdy=\iint_{D_1}f(x;y)dxdy+\iint_{D_2}f(x;y)dxdy\]
    \item Если f и g интегрируемы в D, то существует 
    \[\iint_D [\alpha f(x;y)+ \beta g(x;y)]dxdy=\alpha \iint_D f(x;y)dxdy+ \beta \iint_D g(x;y)dxdy\]
    \item Пусть f,g $\geq 0$ в D и $\forall (.) (x;y) \in D f\leq g$\\
    \[\iint_D f(x;y)dxdy\leq \iint_D g(x;y)dxdy\]
    \item Пусть f>0 $D_1 \subset D_2$
    \[\iint_{D_1} f(x;y)dxdy<\iint_{D_2} f(x;y)dxdy\]
    \item Если f интегрируема в D, то |f| тоже интегрируема в D
    \[\Big| \iint_D f(x;y)dxdy \Big| \leq \iint|f(x;y)|dxdy\]
    \item  
    \subsubsection*{Теорема о среднем}\label{th:8.12.1}
    \par\noindent
    Пусть f и g ограничены и интегрируемы в D, g не меняет знак в D
    \[m\leq f(x;y)\leq M. \text{ Тогда существует }\lambda: m\leq \lambda \leq M: 
    \iint_D f(x;y) g(x;y)dxdy=\lambda \iint_D g(x;y)dxdy\]\\
    Если дополнить f(x;y) непрерывна в D, то существует
    \[(.) (\xi_i;\eta_i) \in D: \iint_D f(x;y)dxdy=f (\xi_i;\eta_i) \Delta D\]
  \end{enumerate}
  \subsection{Вычисление двойного интеграла(сведение двойного интеграла к повторному)}
  Пусть f(x;y) задана в P = [a;b]x[c;d]. $\hspace{20pt}$ Рассмотрим F(x)=$\int_{a}^{c}f(x;y)dy$
  \subsubsection*{Теорема 8.13.1}\label{th:8.13.1}
  \par\noindent
  Если f(x;y) непрерывна в P, то F(x) непрерывна на [a;b]\\
  \underline{Доказательство:}
  \begin{adjustwidth}{1.5em}{1.5em}
    Рассмотрим $F(x+\Delta x)-F(x)=\int_{c}^{d}[f(x+\Delta x)-f(x;y)]dy \hspace{20pt} x,x+\Delta x \in [a;b]$\\
    По условию f(x;y) непрерывна в P $\Rightarrow f(x;y)$ будет равномерно непрерывна в P. \[\forall f(x;y)
    ,(x+\Delta x;y)\in P: |\Delta x|<\delta \Rightarrow |f(x+\Delta x;y)-f(x;y)|< \frac{\varepsilon}{d-c}\]\\
    То есть F(x) непрерывна в (.) x. Так как x-произвольная (.) $\in [a;b],$то F(x) непрерывна на [a;b]
    \begin{center}
      \textbf{Ч.т.д.}
    \end{center}
  \end{adjustwidth}
  \subsubsection*{Теорема 8.13.2}\label{th:8.13.2}
  \par\noindent
  Пусть $\iint_P f(x;y)dxdy$ существует. Тогда 
  \[\iint_D f(x;y)dxdy=\int_{a}^{b}dx \int_{c}^{d} f(x;y)dy=\int_{c}^{d}dy \int_{a}^{b}f(x;y)dx\]
  \underline{Доказательство:}
  \begin{adjustwidth}{1.5em}{1.5em}
    R:$\begin{cases}
      a:x_0<x_1<\dots<x_n=b & m_{ij}\leq f(x;y) \leq M_{ij} \forall (x;y) \in P_{ij} (*)\\
      c=y_0<y_1<\dots<y_k=d
    \end{cases}$\\
    Зафиксируем x=$\xi_i \in [x_i;x_{i+1}]$ и проинтегрируем (*) по переменной y
    \[\int_{y_j}^{y_{j+1}}m_{ij}dy \leq \int_{y_j}^{y_{j+1}}f(\xi_i;y)dy \leq \int_{y_j}^{y_{j+1}} M_{ij}dy\]
    \[m_{ij}\Delta y_j \leq \int_{y_j}^{y_{j+1}} f (\xi_i;y)dy \leq M_{ij}\Delta y_j\]
    \[\sum_{j=0}^{k-1} m_{ij}\Delta y_j \leq \sum_{j=0}^{k-1}\int_{y_j}^{y_{j+1}}f (\xi_i;y)dy
    \leq \sum_{j=0}^{k-1} M_{ij}\Delta y_j \Big| *\Delta x_i \text{ и проинтегрируем по } 
    i=\overline{0,n-1}\]
    \[\sum_{i=0}^{n-1}\sum_{j=0}^{k-1} m_{ij}\Delta x_j \Delta y_j \leq \sum_{i=0}^{n-1}
    [\int_{c}^{d} \equalto{f (\xi_i;y)}{I(\xi_i)}]\Delta x_i \leq 
    \sum_{i=0}^{n-1} \sum_{j=0}^{k-1} M_{ij} \Delta x_i \Delta y_j\]
    \[\text{При }\lambda_R \to 0: m_{ij},M_{ij} \rightarrow I= \iint_P f(x;y) dxdy\]
    %TODO:Найти способ как написать стрелку под текстом, пикча в Images лежит 
    Тогда $\iint_P f(x;y) dxdy = \lim_{\lambda_R \to 0} \sum_{i=0}^{n-1}I(\xi_i)\Delta x_i =
    \int_{a}^{b}I(x)dx=\int_{a}^{b}dx \int_{c}^{d}f(x;y)dy$\\
    Аналогично $\iint_P f(x;y)dxdy=\int_{c}^{d}dy \int_{a}^{b}f(x;y)dx$
    \begin{center}
      \textbf{Ч.т.д.}
    \end{center}
  \end{adjustwidth}
  \subsection{Вычисление двойного интеграла по произвольной области}
  \subsubsection*{Теорема 8.14.1}\label{th:8.14.1}
  \par\noindent
  Пусть $y_1(x)$ и $y_2(x)$ непрерывна на [a;b].\\
  Рассмотрим D:
  $\begin{cases}
    a\leq x\leq b\\
    y_1(x)\leq y \leq y_2(x)
  \end{cases}$\\
  Если f(x;y) задана в D и $\forall x \in [a;b]$ она интегрируема по y на $[y_1(x);y_2(x)]$ и
  f(x;y) непрерывна в D, то F(x) = $\int_{y_1(x)}^{y_2(x)}f(x;y)dy$ непрерывна на [a;b].\\
  \underline{Доказательство:}
  \begin{adjustwidth}{1.5em}{1.5em}
    Замена $y=y_1(x)+(y_2(x)-y_1(x))t \hspace{20pt} 0\leq t\leq 1 \hspace{20pt} dy=(y_2(x)-y_1(x))dt$
    \[F(x)=\int_{0}^{1} \underbrace{f(x;y_1(x)+(y_2(x)-y_1(x))t)(y_2(x)-y_1(x))}_{g(x;t)}dt=
    \int_{0}^{1}g(x;t)dt\]
    \begin{center}
      g(x;t) определена в прямоугольнике\\
      $a\leq x\leq b$\\
      $0\leq t\leq1$
    \end{center} 
    Тогда F(x) непрерывна по \hyperref[th:8.13.1]{Теореме 8.13.1}
    \begin{center}
      \textbf{Ч.т.д.}
    \end{center}
  \end{adjustwidth}
  \subsubsection*{Теорема 8.14.2}\label{th:8.14.2}
  \par\noindent
  Пусть f(x;y) интегрируема и непрерывна в D и $\forall x \in [a;b]$ прямая || OY пересекает границу
  D не более чем в $2^x$(.). То есть область D задана в виде D:
  $\begin{cases}
    a\leq x\leq b\\
    y_1(x)\leq y\leq y_2(x)
  \end{cases}$.\\
  Тогда:
  \[\iint_D f(x;y)dxdy = \int_{a}^{b}dx \int_{y_1(x)}^{y_2(x)}dy\]
  \underline{Замечание:} Аналогично можно рассмотреть: $\iint_D f(x;y)dxdy = \int_{c}^{d}dy
  \int_{x_1(y)}^{x_2(y)}f(x;y)dx$, если прямая || OX пересекает D не более чем в $2^x$(.).\\
  \underline{Замечание:}Если контур пересекается не более чем в $2^x$(.) как прямыми || OY,
  так и || OX, то справедливы обе формулы.\\
  \underline{Замечание:} Если контур не подходит ни под один случай, то его разбивают на $D_1,D_2,\dots$\\
  \underline{Доказательство:}
  \begin{adjustwidth}{1.5em}{1.5em}
    F(x) $\overset{\text{(на [a;b])}}{\text{непрерывна}}$ по \hyperref[th:8.14.1]{Теореме 8.14.1}. Тогда F(x)
    интегрируема на [a;b]. То есть, существует: \[\int_{a}^{b}F(x)dx=\int_{a}^{b}dx \int_{y_1(x)}^{y_2(x)}
    f(x;y)dy\]\\
    Введем разбиение:
    $\begin{cases}
      \vspace{2pt}
      x_i = a + \frac{b-a}{k}i \\
      \vspace{2pt}
      x_i-x_{i+1}=\frac{b-1}{k} \\
      i=\overline{1,k}
    \end{cases}$\\
    Введем вспомогательные функции:
    $\begin{cases}
      \vspace{2pt}
      \varphi_0(x)=y_1(x)\\
      \vspace{2pt}
      \varphi_1(x)=y_1(x)+\frac{y_2(x)-y_1(x)}{k}i\\
      \vdots\\
      \varphi_j(x)=y_1(x)+\frac{y_2(x)-y_1(x)}{k}j\\
      \vspace{2pt}
      \varphi_k(x)=y_1(x)+\frac{y_2(x)-y_1(x)}{k}k
    \end{cases}$\\
    Получим разбиение: 
    \[E^{k}_{ij}=
    \begin{cases}
      x_{i-1}\leq x \leq x_i\\
      \varphi_{j-1}\leq y\leq \varphi_j
    \end{cases} \hspace{20pt} i,j = \overline{1,k}\]
    \[m_{ij}=\underset{(x,y) \in E_{ij}^k}{\inf f(x;y)} \hspace{20pt} M_{ij}=\underset{(x,y) \in E_{ij}^k}{\sup f(x;y)}\]\\
    \[ \text{Рассмотрим} 
    \int_{a}^{b}dx+\int_{y_1(x)}^{y_2(x)}f(x;y)dy =
    \sum_{i=1}^{k} \int_{x_{i-1}}^{x_i}dx \sum_{j=1}^{k} \int_{\varphi_{j-1}}^{\varphi_j}f(x;y)dy =
    \sum_{i=1}^{k}\sum_{j=1}^{k} \int_{x_{i-1}}^{x_i} dx\int_{\varphi_{j-1}}^{\varphi_j}f(x;y)dy \leq\]
    \[\leq \sum_{i=1}^{k}\sum_{j=1}^{k} M_{ij} \int_{x_{i-1}}^{x_i}dx \int_{\varphi_{j-1}}^{\varphi_j}dy=
    \sum_{i=1}^{k}\sum_{j=1}^{k} M_{ij} M_{ij} \int_{x_{i-1}}^{x_i}dx[\varphi_j(x)-\varphi_{j_1}(x)]=
    \sum_{i=1}^{k}\sum_{j=1}^{k} \overbrace{\Delta E_{ij}^k}^{\text{S площадь}}=
    \overline{S}_R\]\\
    Аналогично $\int_{a}^{b}dx \int_{y_1(x)}^{y_2(x)}f(x;y)dy \geq \underline{S}_R$\\
    Получим $\underline{S}_R \leq \int_{a}^{b}dx \int_{y_1(x)}^{y_2(x)} f(x;y) dy \leq \overline{S}_R$
    %TODO переделать стрелки, лежит images peredelat2
    \begin{center}
      \textbf{Ч.т.д.}
    \end{center}
  \end{adjustwidth}
  \subsection{Формула Грина}
  \begin{minipage}{0.45\textwidth}
    \includegraphics[scale=1]{8.15.1.png}
  \end{minipage}
  \hspace{1em}
  \begin{minipage}{0.55\textwidth}
    \[\oint_{L}Pdx+Qdy\]
  \end{minipage}
  \vspace{1em}
  \par

  \begin{minipage}{0.45\textwidth}
    \includegraphics[scale=0.6]{8.15.2.png}
  \end{minipage}
  \hspace{1em}
  \begin{minipage}{0.55\textwidth}
    Рассмотрим 
    $\begin{cases}
      PQ:y=y_1(x)\\
      SR: y=y_2(x)\\
      a\leq x \leq b\\
      PS|OY, QR||OY
    \end{cases}$\\
  \end{minipage}
  \vspace{1em}
  \par
  Пусть в области D задана P(x;y). P(x;y) непрерывна $\frac{\delta P}{\delta y}$ непрерывна.
  \[\text{Рассмотрим }\] \[ \iint_D \frac{\delta P}{\delta y}dxdy= 
  \int_{a}^{b}dx \int_{y_1(x)}^{y_2(x)} \frac{\delta P}{\delta y}dy=
  \int_{a}^{b}dx \Big[P(x;y)\Big|_{y_1(x)}^{y_2(x)}\Big]=
  \int_{a}^{b}dx \Big[P(x;y_2(x))-P(x;y_1(x))\Big] \boxed{=}\]
  \[\text{Рассмотрим} \int_{a}^{b}P(x;y_2(x))dx=\int_{(SR)}P(x;y)dx\]
  \[\text{Рассмотрим} \int_{a}^{b}P(x;y_1(x))dx=\int_{(PQ)}P(x;y)dx=-\int_{(QP)}P(x;y)dx\]
  \[\boxed{=} \int_{(SR)}P(x;y)dx +\int_{(QP)}P(x;y)dx+\int_{(RQ)}P(x;y)dx+\int_{(PS)}P(x;y)dx \boxed{=}\]
  Пусть L-контур в положительном направлении.
  \[\boxed{=} - \int_{(L)}P(x;y)dx\]\\
  \break
  Аналогично:\\
  \par
  \begin{minipage}{0.45\textwidth}
    \includegraphics[scale=0.8]{8.15.3.png}
  \end{minipage}
  \hspace{1em}
  \begin{minipage}{0.55\textwidth}
    \[0,\frac{\delta Q}{\delta x} \text{непрерывна в D}\]
    \[\iint_P \frac{\delta Q}{\delta x}dxdy=\dots=\int_{(L)} Q(x;y)dy\] 
  \end{minipage}
  \vspace{1em}
  \par
  \begin{center}
    Если область D удовлетворяет обоим случаям, тогда справедлива формула:
    \[\boxed{\int_{(L)}Pdx+Qdy=\iint_D \Big[ \frac{\delta Q}{\delta x}-\frac{\delta P}{\delta y}\Big]dxdy}\]
  \end{center}
  \break
  \subsection{Замена переменных в двойном интеграле}
  \subsection*{I Преобразование плоских областей.}
  Рассмотрим 2 прямоугольных СК: xy и $\xi \eta$\\
  \begin{minipage}{0.5\textwidth}
    \includegraphics[scale=0.8]{8.16.1.png}
  \end{minipage}
  \hspace{1em}
  \begin{minipage}{0.5\textwidth}
    \includegraphics[scale=0.7]{8.16.2.png}
  \end{minipage}
  \vspace{1em}
  \par
  Пусть в области $\Delta$ задана системой непрерывных функций:
  $\begin{cases}
    x=x(\xi,\eta)\\
    y=y(\xi,\eta)
  \end{cases} (1)$\\
  Если различными (.) $(\xi;\eta) \in \Delta$ отвечают различные (.) $(x,y) \in D,$ то система (1)
  однозначно разрешима относительно $\xi$ и $\eta$.
  $\begin{cases}
    \xi=\xi(x;y)\\
    \eta=\eta(x;y)
  \end{cases} (2)$\\
  \underline{Определение: } $\frac{D(x;y)}{D(\xi,\eta)}=
  \begin{vmatrix}
    \frac{\delta x}{\delta \xi} & \frac{\delta x}{\delta \eta}\\
    \frac{\delta y}{\delta \xi} & \frac{\delta y}{\delta \eta}
  \end{vmatrix}$ Якобиан перехода из КСК в ДСК.\\
  \begin{itemize}
    \item Если $\frac{D(x;y)}{D(\xi;\eta)} \not = 0,$ то внутренней (.) $\xi,\eta$ соответствует
    внутренней (.) (x;y).\\Граничной (.) $(\xi,\eta)$ соответствует граничная (.) $(x;y)$
    \item Если $\frac{D(x;y)}{D(\xi;\eta)} > 0$, то при переходе в ДСК, направление обхода не меняется.
    \item Если $\frac{D(x;y)}{D(\xi;\eta)} < 0$, то при переходе в ДСК, направление обхода поменяется.
    \item Если в $\Delta$ взять гладкую кривую, то с помощью (1):
    $\begin{matrix}
        \begin{cases}
          \xi=\xi(t)\\
          \eta=\eta(t)
        \end{cases} \alpha\leq t\leq\beta\\
        \begin{cases}
          x=x(\xi(t);\eta(t))=x(t)\\
          y=y(\xi(t);\eta(t))=y(t)
        \end{cases} (3)
      \end{matrix}$
  \end{itemize}
  \underline{Примеры:}\\
  a)ПСК
  \[\begin{matrix}
    x=r\cos(\varphi) & r=\sqrt{x^2+y^2}\\
    y=r\sin(\varphi) & \varphi=
    \begin{cases}
      \arcctg \frac{y}{x},(.) (x;y) \in \text{ I, II ч.}\\
      \arcctg \frac{y}{x}+\pi,(.) (x;y) \in  \text{II,III ч.}\\
    \end{cases}
  \end{matrix}\]\\
  \break
  б) Семейство пересекающих парабол.
  \[y^2 =2px \hspace{20pt} x^2=2qy\] 
  \begin{minipage}{0.45\textwidth}
    \includegraphics[scale=0.6]{8.16.3.png}
  \end{minipage}
  \hspace{1em}
  \begin{minipage}{0.55\textwidth}
    $\text{ Пусть } 2p=\xi \hspace{20pt} 2q=\eta$\\
    $y^2=\xi x \hspace{20pt} x^2=\eta y$\\
    $x^4=\eta^2y^2=\eta^2 \xi x$\\
    $x^3=\eta^2 \xi$\\
    $\begin{vmatrix}
      x=\sqrt[3]{\eta^2 \xi}\\
      y=\sqrt[3]{\xi^2 \eta}
    \end{vmatrix} 
    \begin{matrix}
      \xi=\frac{x}{y^2}\\
      \eta=\frac{y}{x^2}
    \end{matrix}$
  \end{minipage}
  \vspace{1em}
  \par
  \subsection*{II Вычисление площади в криволинейной системе координат:}
  Рассмотри область $D \in x;y,$ ограниченную кусочно-гладкую контуром L. Пусть существует (1) и (2).
  Пусть существует $\frac{\delta^2 y}{\delta \xi \delta \eta}$ и она непрерывна. Найти $S_D$.\\
  \[S_D =\int_{(L)}xdy=
  \begin{vmatrix}
    x=x(t)\\
    y=y(t)\\
    \alpha \leq t \leq \beta\\
    \text{\underline{Замечание:} при изменении t от } \alpha \text{ до } \beta\\
     \text{L - положительный контур.}
  \end{vmatrix} 
  = \int_{a}^{b} x(t)y'(t)dt \overset{(3)}{=} \] 
  \[\overset{(3)}{=} \int_{\alpha}^{\beta} x(\xi(t),\eta(t))
  \Big[ \frac{\delta y}{\delta \xi}\xi'(t)+\frac{\delta y}{\delta \eta}\eta'(t)\Big]dt \boxed{=}\]\\
  Рассмотрим $\int_{\Sigma} x(\xi;\eta) \Big[\frac{\delta y}{\delta\xi}d\xi + \frac{\delta y}{\delta \eta}d \eta\Big]
  \rightarrow \int_{\alpha}^{\beta}x(\xi(t);\eta(t))\Big[ \frac{\delta y}{\delta \xi} \xi'(t)+
  \frac{\delta y}{\delta \eta}\eta'(t)\Big]dt$\\
  \[\boxed{=} \pm \int_{\Sigma}x(\xi(t);\eta(t)) \Big[\frac{\delta y}{\delta\xi}d\xi + \frac{\delta y}{\delta \eta}d \eta\Big]
  \boxed{=}
  \begin{matrix}
    \text{"+", если положительный обход у L соответствует} \\
    \text{положительному обходу } \Sigma\\
    \text{"-", если положительный обход у L соответствует} \\
    \text{отрицательному обходу } \Sigma
  \end{matrix}\]
  \begin{center}
    \includegraphics[scale=0.7]{8.16.4.png}
  \end{center}
  \subsection*{Формула Грина:}
  \[\int_{(L)} Pdx+Qdy=\iint_D \Big[ \frac{\delta Q}{\delta x} - \frac{\delta P}{\delta y}\Big] dxdy\]
  \[\int_{(\Sigma)}P(\xi;\eta)d\xi+Q(\xi;\eta)d\eta= \iint_{\Delta} \Big[\frac{\delta Q}{\delta \xi} -\frac{\delta P}{\delta \eta} d\xi d\eta \Big]\]
  \[P(\xi;\eta)=x \frac{\delta y}{\delta \xi} \hspace{20pt} Q(\xi;\eta)=x \frac{\delta y}{\delta \eta}\]
  \[\frac{\delta Q}{\delta \xi}=\frac{\delta}{\delta \xi}(x \frac{\delta y}{\delta \eta})=
  \frac{\delta x}{\delta \xi}\frac{\delta y}{\delta \eta}+x\frac{\delta^2y}{\delta\xi \delta\eta}\]
  \[\frac{\delta P}{\delta \eta}=\frac{\delta}{\delta \eta}(x \frac{\delta y}{\delta \xi})=
  \frac{\delta x}{\delta \eta} \frac{\delta y}{\delta \xi}+x \frac{\delta^2 y}{\delta \eta \delta \xi}\]
  \[\frac{\delta Q}{\delta \xi}-\frac{\delta P}{\delta \eta}=\frac{\delta x}{\delta \xi} \frac{\delta y}{\delta \eta}-
  \frac{\delta x}{\delta \eta}\frac{\delta y}{\delta \xi}=
  \begin{vmatrix}
    \frac{\delta x}{\delta \xi} & \frac{\delta x}{\delta \eta}\\
    \frac{\delta y}{\delta \xi} & \frac{\delta y}{\delta \eta}\\
  \end{vmatrix} = \frac{D(x;y)}{D(\xi;\eta)}\]\\
  \[\boxed{=} \pm \iint_\Delta \frac{D(x;y)}{D(\xi;\eta)}d\xi d\eta \overset{\text{опр Якобиана}}{=} 
  \underbracket{\iint_\Delta \Big|\frac{D(x;y)}{D(\xi;\eta)}\Big|d\xi d\eta \boxed{=}}\]
  Теорема Лагранжа:
  \[f(b)-f(a)=f'(\xi)(b-a) \hspace{20pt} 
  \begin{matrix}
    \varepsilon = f'(\xi)\delta\\
    \lim_{\lambda_R \to 0} \frac{\varepsilon}{\delta}=f'(\xi)
  \end{matrix} \hspace{20pt} f'(\xi)=\frac{\varepsilon}{\delta}\]
  $f'(\xi)$ является коэффициентом растяжение(сжатия) прямой x в прямоугольнике в данной её (.)($\xi$)\\
  \[\overset{\hyperref[th:8.12.1]{\text{Теорема о среднем}}}{\boxed{=}}
  \Big|\frac{D(x;y)}{D(\xi;\eta)} \Big|_{M \in \Delta} \hspace{20pt} S_\Delta=S_D\]
  \[\iint_D f(x;y)dxdy=f(\xi;\eta)\iint_D dxdy = f(\xi;\eta)S_D\]
  \[\Big| \frac{D(x;y)}{D(\xi;\eta)}\Big|_{M \in \Delta} = \lim_{\Delta \to M} \frac{S_D}{S_\Delta}\]
  Модуль величины Якобиана в (.) есть коэффициент растяжения (сжатия) плоскости $\xi \eta$ при преобразовании
  её в плоскость $xy$
  \subsection*{III Замена переменных в двойном интеграле:}
  Рассмотрим $\iint_{D} f(x;y)dxdy$\\
  \begin{enumerate}
    \item Разбиваем область D кусочно-гладкими кривыми
    \item Разбиение R и так далее
  \end{enumerate}
  \[\sigma_R = \sum_{i=0}^{n-1} f(x_i;y_i)S_{D_i} =\sum_{i=0}^{n-1} f(x_i;y_i) \Big| I \Big| \Big|_{\overline{M_i}(\xi_i;\eta_i)}
  * S_{\Delta_i} \boxed{=} \]
  \begin{center}
    Так как $(x_i;y_i)$ - произвольная (.), то выберем её так\\ 
    чтобы в области $D_i$ ей соответствовала (.) $(\overline{\xi_i};\overline{\eta_i})$ в $\Delta_i$
  \end{center}
  \[\boxed{=} \sum_{i=0}^{n-1} f(x(\overline{\xi_i},\overline{\eta_i});y(\overline{\xi_i},\overline{\eta_i}))
  |I|_{\overline{M_i}(\xi_i;\eta_i)}S_{\Delta_i}\]
  \[\lim_{\lambda_R \to 0} \sigma_R \iint_D f(x(\xi;\eta);y(\xi;\eta))|I|d\xi d\eta\]
  Для ПСК: 
  $
  \begin{matrix}
    x=r\cos\varphi\\
    y=r\sin\varphi
  \end{matrix} \hspace{20pt}
  I = \begin{vmatrix}
    \frac{\delta x}{\delta r} & \frac{\delta x}{\delta \varphi}\\
    \frac{\delta y}{\delta r} & \frac{\delta y}{\delta \varphi}
  \end{vmatrix}=
  \begin{vmatrix}
    \cos\varphi & -r\sin\varphi\\
    \sin \varphi & r\cos\varphi 
  \end{vmatrix} = r\cos^2\varphi+r\sin^2\varphi=r
  $
  \[|I|=r\]
  \subsection{Приложения двойных интегралов.}
  \begin{enumerate}
    \item \[S_D=\iint_D dxdy\]
    \item \[V=\iint_D f(x;y)dxdy=\iint_{\overline{D}} f(x;z)dxdz = \iint_{\overline{\overline{D}}}f(y;z) dydz
    =\iint_{\overline{\overline{\overline{D}}}}F(x(u;v);y(u;v);z(u;v))dudv\]
    \item \[m=\iint_D \underbrace{\rho}_{\frac{kg}{m^2}}dxdy\]
    \item \[
    \begin{matrix}
      M_x=\iint \rho y dxdy & M_y=\iint_{\rho}\rho xdxdy\\
      x_c=\frac{M_y}{m} & y_c=\frac{M_x}{m}
    \end{matrix}\]
  \end{enumerate}
  \subsection{Площадь поверхности.}
  Пусть $Z=f(x;y)$. Можно говорить о верхней и нижней стороне поверхности. Если поверхность замкнута,
  то можно говорить о внешней стороне поверхности и о внутренней. Возьмём на поверхности произвольную
  (.) $m_0$, проведем в ней нормаль. И проведем из (.) $m_0$ замкнутый контур, начиная и заканчивая 
  в (.) $m_0$(при этом нормаль изменяется непрерывно). При возврате в (.) $m_0$ можно получить 2 
  варианта: 
  \begin{enumerate}
    \item вектор нормали сохранил своё направление;
    \item вектор нормаль поменяет непрерывные на противоположные.
  \end{enumerate}
  \underline{Определение: } Если при обходе по контуру нормаль меняет направление на противоположное,
  то поверхность называется односторонней. Если не поменяет по замкнутому контуру - двусторонней.\\
  \begin{minipage}{0.54\textwidth}
    \underline{Пример:} лист Мёбиуса - односторонняя поверхность. 
  \end{minipage}
  \hspace{1em}
  \begin{minipage}{0.3\textwidth}
    \includegraphics[scale=1]{8.18.1.png}
  \end{minipage}
  \vspace{1em}
  \par
  \underline{Определение: } Совокупность всех (.) поверхности с приписанными в них направлениями 
  нормалей называется стороной поверхности.\\
  \underline{Ориентация поверхностей}: 
  \begin{itemize}
    \item Верхняя сторона поверхности (острый угол с осью $OZ$) и замкнутую поверхность (внешняя сторона) 
    будем считать положительно ориентированными(правая ориентация)
    \item Нижняя сторона поверхности или внутренняя сторона(замкнутая поверхность) будем считать
    отрицательной ориентацией(левая ориентация)
  \end{itemize}
  \underline{Площадь поверхности}: Рассмотрим не замкнутую гладкую поверхность $S$, ограниченную
  кусочно-гладким контуром L.\\
  Разложим эту поверхность на элементарные $S_1,\dots,S_n$\\
  В каждой элементарной $S_i$ выберем произвольную (.) $M_i$ и в этой (.) проведем касательную
  плоскость к $S_i$\\
  \par
  \begin{minipage}{0.45\textwidth}
    \includegraphics[scale=0.6]{8.18.2.png}
  \end{minipage}
  \hspace{1em}
  \begin{minipage}{0.55\textwidth}
    Построим на границе области $S_i$ цилиндрическую поверхность с образующими || оси OZ.Пересечение
    этого цилиндра с касательной плоскостью образует плоскую фигуру $T_i$.
    \[\text{Тогда }S=\lim_{\lambda_R \to 0} \Sigma \Delta T_i\]
  \end{minipage}
  \vspace{1em}
  \par
  Пусть поверхность задана:
  \[
  \begin{matrix}
    x=x(u;v)\\
    y=y(u;v)\\
    z=z(u;v)
  \end{matrix} \hspace{20pt}
  \begin{cases}
    x=r\cos(u)\\
    y=r\sin(u)\\
    z=v
  \end{cases} \text{ - цилиндрическая поверхность}
  \]
  Пусть $\mu' \in S \hspace{20pt} \mu'(x';y';z')$. Пересечем СК в (.) $\mu'$ и перейдем к новой
  СК $\xi \eta \chi$.\\
  \break
  \underline{Формулы преобразования координат:}
  \[\xi=(x-x')\cos(\alpha_1)+(y-y')\cos(\beta_1)+(z-z')\cos(\gamma_1)\]
  \[\eta=(x-x')\cos(\alpha_2)+(y-y')\cos(\beta_2)+(z-z')\cos(\gamma_2)\]
  \[\chi=(x-x')\cos(\alpha_3)+(y-y')\cos(\beta_3)+(z-z')\cos(\gamma_3)\]
  \begin{tabular}{ p{10pt}|p{10pt}|p{10pt}|p{10pt} }
    & X & Y & Z\\
    \hline
    $\xi$ & $\alpha_1$ & $\beta_1$ & $\gamma_1$\\
    \hline
    $\xi$ & $\alpha_2$ & $\beta_2$ & $\gamma_2$\\
    \hline
    $\xi$ & $\alpha$ & $\beta$ & $\gamma$\\
  \end{tabular}
  \hspace{20pt} $T_i = \iint_{D_{UV}}dudv=\iint_{D_{\xi \eta}}\Big| \frac{D(\xi;\eta)}{D(u;v)}\Big|
  d \xi d \eta$
  \[\frac{D(\xi;\eta)}{D(u;v)}=
  \begin{vmatrix}
    \frac{\delta \xi}{\delta u} &\frac{\delta \xi}{\delta V}\\
    \frac{\delta\eta}{\delta u} &\frac{\delta \eta}{\delta V}
  \end{vmatrix}=
  \]
  \[=\Big|
  \begin{matrix}
  \frac{\delta \xi}{\delta x}\frac{\delta x}{\delta u}+\frac{\delta \xi}{\delta y}\frac{\delta y}{\delta u}+
  \frac{\delta \xi}{\delta z}\frac{\delta Z}{\delta u} = x'_u\cos(\alpha_1)+y'_u\cos(\beta_1)+z'_u\cos(\gamma_1) & x'_u\cos(\alpha_1)+y'_u\cos(\beta_1)+z'_u\cos(\gamma_1) \\
  x'_v\cos(\alpha_2)+y'_v\cos(\beta_2)+z'_v\cos(\gamma_2) & x'_v\cos(\alpha_2)+y'_v\cos(\beta_2)+z'_v\cos(\gamma_2) 
  \end{matrix}\Big| =
  \]
  \[\begin{vmatrix}
    \begin{pmatrix}
      x'_u & y'_u & z'_u\\
      x'_v & y'_v & z'_v
    \end{pmatrix}
    \begin{pmatrix}
      \cos(\alpha_1)& \cos(\alpha_2)\\
      \cos(\beta_1)&\cos(\beta_2)\\
      \cos(\gamma_1)&\cos(\gamma_2)
    \end{pmatrix}
  \end{vmatrix}=\]
  \[=\underbrace{\begin{vmatrix}
    y'_u & z'_u\\
    y'_v & z'_v
  \end{vmatrix}}_{A}
  \underbrace{\begin{vmatrix}
    \cos(\beta_1) & \cos(\gamma_1)\\
    \cos(\beta_2) & \cos(\gamma_2)
  \end{vmatrix}}_{\cos(\alpha)}
  +
  \underbrace{\begin{vmatrix}
    z'_u&x'_u\\
    z'_v&x'_v
  \end{vmatrix}}_{B}
  \underbrace{\begin{vmatrix}
    \cos(\gamma_1) & \cos(\alpha_1)\\
    \cos(\gamma_2)&\cos(\alpha_2)
  \end{vmatrix}}_{\cos(\beta)}
  +
  \underbrace{\begin{vmatrix}
    x'_u & y'_u\\
    x'_v & y'_v
  \end{vmatrix}}_{C}
  \underbrace{\begin{vmatrix}
    \cos(\alpha_1)&\cos(\beta_1)\\
    \cos(\alpha_2)&\cos(\beta_2)
  \end{vmatrix}}_{\cos(\gamma)}
  \boxed{=}\]
  \par
  \underline{Замечание:} Каждый из координатных ортов ($\overline{\cos(\alpha_1);\cos(\beta_1);\cos(\gamma_1)}$),($\overline{\cos(\alpha_2);\cos(\beta_2);\cos(\gamma_2)}$)
  ,($\overline{\cos(\alpha);\cos(\beta);\cos(\gamma)}$) взаимно перпендикулярны.
  Поэтому:
  \[(\overline{\cos(\alpha);\cos(\beta);\cos(\gamma)}) = 
  \begin{vmatrix}
    i&j&k\\
    \cos(\alpha_1)&\cos(\beta_1)&\cos(\gamma_1)\\
    \cos(\alpha_2)&\cos(\beta_2)&\cos(\gamma_2)
  \end{vmatrix}=\]
  \[= \overline{i}
  \equalto{
    \begin{vmatrix}
    \cos(\beta_1)&\cos(\gamma_1)\\
    \cos(\beta_2)&\cos(\gamma_2)
    \end{vmatrix}
  }{\cos(\alpha)}+ \overline{j}
  \begin{vmatrix}
    \cos(\alpha_1)&\cos(\gamma_1)\\
    \cos(\alpha_2)&\cos(\gamma_2)
  \end{vmatrix}
  + \overline{k}
  \begin{vmatrix}
    \cos(\alpha_1)&\cos(\beta_1)\\
    \cos(\alpha_2)&\cos(\beta_2)
  \end{vmatrix}
  \]
  \[\boxed{=}A\cos(\alpha)+B\cos(\beta)+C\cos(\gamma)\boxed{=}\]
  \underline{Определение: } Если поверхность задана $\begin{matrix}
    x=x(u;v)\\
    y=y(u;v)\\
    z=z(u;v)
  \end{matrix}$, то нормаль к поверхности задается$\begin{vmatrix}
    \overline{i}&\overline{j}&\overline{k}\\
    x'_u & y'_u & z'_u\\
    x'_v & y'_v & z'_v
  \end{vmatrix}=(n_x;n_y;n_z)=\overline{(A;B;C)}$\\
  \underline{Пример:}\\
  \[\begin{matrix}
    \text{Рассмотрим } z=f(x;y)\\
    u=x\\
    v=y
  \end{matrix}
  \hspace{20pt}
  \begin{vmatrix}
    \overline{i}&\overline{j}&\overline{k}\\
    1&0&z'_x\\
    0&1&z'_y 
  \end{vmatrix}=
  \begin{matrix}
    \overline{i}(-z'_x)-jz'_y+\overline{k}\\
    (-z'_x;-z'_y;1) \text{ или }\\
    (z'_x;z'_y;-1)
  \end{matrix}\]
  \underline{С другой стороны:}
  \[
  \cos(\alpha)=\frac{A}{\pm \sqrt{A^2+B^2+C^2}}\\
  \cos(\beta)=\frac{B}{\pm \sqrt{A^2+B^2+C^2}}\\
  \cos(\gamma)=\frac{C}{\pm \sqrt{A^2+B^2+C^2}}
  \]
  Пусть $A',B',C'$ - значения определителей $A,B,C$ в (.) $\mu'$
  \[\boxed{=}
  \frac{AA'+BB'+CC'}{\pm \sqrt{A^2+B^2+C^2}}
  \boxed{=}\]
  Если расстояние между (.) ($u;v$) и ($u';v'$) стремится к нулю, то $\frac{D(\xi;\eta)}{D(u;v)} = \sqrt{A'^2+B'^2+C'^2}+\varepsilon'$
  \[
  S_{T_i}=\iint_{D_{uv}}\sqrt{A^2+B^2+C^2}dudv+\varepsilon;S_\text{поверхности}=\lim_{\lambda_R \to 0}\Sigma S_{T_i}=\iint_{D_{uv}}\sqrt{A^2+B^2+C^2}dudv=S
  \]
  \[
  \text{Рассмотрим }
  \begin{pmatrix}
    x_u & y'_u &z'_u\\
    x'_v&y'_v&z'_v
  \end{pmatrix}
  \begin{pmatrix}
    x'_u &x'_v\\
    y'_u&y'_v\\
    z'_u&z'_v
  \end{pmatrix}
  =
  \begin{pmatrix}
    x'^2_u+y'^2_u+z'^2_u & x'_u x'_v+y'_u y'_v+ z'_u z'_v\\
    \underbrace{x'_u x'_v+y'_u y'_v+ z'_u z'_v}_F & \underbrace{x'^2_v+y'^2_v+z'^2_v}_{G}
  \end{pmatrix}
  \]
  \[
  \text{Рассмотрим } \begin{vmatrix}
    E & F\\
    F & G
  \end{vmatrix} =EG-F^2=A^2+B^2+C^2 \hspace{20pt} S=\iint_{D_{uv}}\sqrt{EG-F^2}dudv
  \]
  $\text{Рассмотрим } Z=f(x;y) \text{ уравнение поверхности}.
  \begin{matrix}
    u \sim x\\
    v \sim y
  \end{matrix}$ 
  \[ A=
  \begin{vmatrix}
    y'_u & z'_u \\
    y'_v & z'_v
  \end{vmatrix} =\begin{vmatrix}
    0 & z'_x\\
    1 & z'_y
  \end{vmatrix}=-z'_x\]
  \[ B=
  \begin{vmatrix}
    z'_u & x'_u \\
    z'_v & x'_v
  \end{vmatrix} =\begin{vmatrix}
    z'_x & 1\\
    z'_y & 0
  \end{vmatrix}=-z'_y\]
  \[ C=
  \begin{vmatrix}
    x'_u & y'_x\\
    x'_v & y'_y
  \end{vmatrix} =\begin{vmatrix}
    1 & 0\\
    0 & 1
  \end{vmatrix}=1\]
  \[S=\iint_{D_{xy}}\sqrt{1 + z'^2_x + z'^2_y}dxdy\]
  Рассмотрим $\cos(\gamma)=\frac{C}{\pm \sqrt{A^2+B^2+C^2}}=\frac{1}{\pm \sqrt{1+z'^2_x+z'^2_y}}$
  \[S=\iint_{D_{xy}}\sqrt{1+z'^2_x+z'^2_y}dxdy=\iint_{D_{xy}}\frac{dxdy}{|\cos(\gamma)|} \hspace{20pt} \gamma - \forall \text{ угол}\]
  \break
  \subsection{Поверхностные интегралы I и II рода.}
  Пусть в каждой (.) двусторонней гладкой(или кусочно-гладкой) поверхности, ограниченной кусочно-гладким контуром
  определена функция f(x;y;z)
  \begin{enumerate}
    \item Производим разбиение R поверхности
    \item Выберем (.) $\mu_i$ произвольно: $\mu_i \in S_i$
    \item Вычислим $f(\mu_i) = f(x_i;y_i;z_i)$
    \item Вычислим $f(x_i;y_i;z_i)\Delta S_i$
    \item Составим $\sigma_R=\sum_{i=0}^{n-1}f(x_i;y_i;z_i)\Delta S_i$
    \item Вычислим $\lim_{\lambda_R \to 0} \sigma_R = \iint_{S} f(x;y;z)dS$
  \end{enumerate}
  \underline{Определение: }Если существует конечный $\lim_{\lambda_R \to 0}$, не зависящий от способа разбиения поверхности
  S и выбора (.) $\mu_i$ , то он называется поверхностным интегралом I рода от функции $f(x;y;z)$ по поверхности S.\\
  \underline{Вычисление: } 
  \[ \iint_S f(x;y;z)dS=\iint_{D_{xy}}f(x;y;z(x;y)) \sqrt{1+z'^2_x+z'^2_y}dxdy = 
  \iint_{D_{uv}} f(x(u;v);y(u;v);z(u;v))\sqrt{EG-F^2}dudv\]
  \subsection*{Поверхностные интегралы II рода:}
  Пусть в каждой (.) двусторонней гладкой(или кусочно-гладкой) поверхности, ограниченной кусочно-гладким контуром
  определена функция $f(x;y;z)$.\\ Выберем одну из сторон поверхности:
  \begin{enumerate}
    \item Произведем разбиение R к поверхности
    \item Выберем (.) $\mu_i$ произвольно: $\mu_i \in S_i$
    \item Вычислим $f(\mu_i)=f(x;y;z)$
    \item Вычислим $f(x_i;y_i;z_i)\Delta D_i; \hspace{20pt} \Delta D_i$ - проекция $S_i$ на плоскость $Dy$\\
    Проекция берется со знаком "+", если вектор нормали к поверхности образует острый угол с осью $OZ$. "-" если угол
    между нормалью и $OZ$ тупой.
    \item Составим $\sigma_R=\sum_{i=0}^{n-1}f(x_i;y_i;z_i)\Delta D_i$
    \item $\lim_{\lambda_R \to 0}\sigma_R=\iint_S f(x;y;z)dxdy$
  \end{enumerate}
  \underline{Определение: }  Если существует конечный $\lim_{\lambda_R \to 0}$, не зависящий от способа разбиения
  поверхности S и выбора (.) $\mu_i$, то он называется поверхностным интегралом II рода от функции $f(x;y;z)$ по 
  поверхности S.\\
   \[\text{Аналогично: }\iint_S f(x;y;z)dxdz, \iint_S f(x;y;z)dydz\]
  \[\iint_S P(x;y;z)dydz+Q(x;y;z)dxdz+R(x;y;z)dxdy\text{ - поверхностный интеграл II рода общего вида}\]
  \underline{Вычисление: }
  \[\iint_S f(x;y;z)dxdy = \pm \iint_{D_{xy}} f(x;y;z(x;y))dxdy\]
  \pagebreak
  \subsection{Связь между поверхностными интегралами I и II рода.}
  Рассмотрим $\sigma_R=\sum_{i=1}^{n} f(x_i;y_i;z_i)\Delta D_i$\\
  Рассмотрим $\Delta S_i = \iint_D \frac{dxdy}{\cos(\gamma_i)} \hspace{20pt} \gamma_i$- острый угол\\
  \underline{Применим \hyperref[th:8.12.1]{теорему о среднем}}:
  \[\Delta S_i = \frac{1}{\cos(\overline{\gamma_i})}\iint_{D_i} dxdy= \frac{1}{\cos(\overline{\gamma_i})}\Delta D_i
  \Rightarrow \Delta D_i = \cos(\overline{\gamma_i})\Delta S_i\]
  Тогда $\sigma_R=\sum_{i=1}^{n}f(x_i;y_i;z_i)\cos(\overline{\gamma_i})\Delta S_i.\sigma_R$ не является интегральной суммой
  (зависит не от одной(.)).\\
  Рассмотрим $\sigma_R^*=\sum_{i=1}^{n}f(x_i;y_i;z_i)\cos(\overline{\gamma_i})\Delta S_i$. \\
  Оценим $|\sigma_R-\sigma_R^*|= \sum_{i=1}^{n}f(x_i;y_i;z_i)|\cos(\gamma_i)-\cos(\overline{\gamma_i})|\Delta S_i \boxed{<}$
  \begin{enumerate}
    \item $f(x;y;z)$ интегрируема $\Rightarrow$ она ограниченна в каждой (.) поверхности S.
    \[|f(x_i;y_i;z_i)|<M_i\]
    \item Функция $\cos(\gamma)$ непрерывна, т.е.:
    \[\Delta S_i < \delta \Rightarrow |\cos(\gamma_i)-\cos(\overline{\gamma_i})|<\varepsilon\]
  \end{enumerate}
  \boxed{<} $M\varepsilon \Delta S =\varepsilon^*$
  \[\lim_{\lambda_R \to 0} \sigma_R = \lim_{\lambda_R \to 0} \sigma_R^* = \iint_S f(x;y;z)\cos(\gamma)dS\]
  \underline{Замечание:} $\iint_{S} f(x;y;z)dxdy \boxed{=} $ S - цилиндрическая поверхность с образующими || $OZ \boxed{=} 0$\\
  \underline{Замечание:} \[\iint_{S} f(x;y;z)dxdy \boxed{=} \pm \iint_{D_{uv}} f(x(u;v);y(u;v);z(u;v))C dudv\] 
  \[C=\begin{vmatrix}
    x'_u & y'_u\\
    x'_v & y'_v
  \end{vmatrix}\]
  \underline{В общем случае:}
  \[\iint_S P(x;y;z)dydz+Q(x;y;z)dxdz+R(x;y;z)dxdy=\]\[\iint_S (P\cos(\alpha)+Q\cos(\beta)+R\cos(\gamma))dS=\]
  \[\pm \iint_{D_{uv}}(PA+QB+RC)dudv\]
  \subsection{Формула Стокса.}
  Пусть S-гладкая, двусторонняя поверхность, ограниченная кусочно-гладким контуром (L).\\
  $Z=f(x;y) \hspace{20pt}
  \begin{matrix}
    x=x(u;v)\\
    y=y(u;v)\\
    z=z(u;v)
  \end{matrix} \Bigg\}$ уравнение поверхности S\\
  \begin{itemize}
    \item Пусть $D_{uv}$ - проекция поверхности S.
    \item Пусть $D_{uv}$ - взаимно-однозначно связана с (.) поверхности S.
    \item Пусть ($l$) - контур, охватывающий область $D_{uv}$
    \item Пусть положительному обходу контура (L) соответствует положительных обход (l).
    \item Пусть функция $P(x;y;z)$ определена в каждой (.) поверхности S. %это не мой прикол так писать ПУСТЬ ПУСТЬ ПУСТЬ, пикча pust.png
    \item Она непрерывна вместе со своими $z_n$.
  \end{itemize}
  \[\text{Тогда} \int_{(L)}P(x;y;z)dx=\iint_S [\frac{\delta p}{\delta z}dxdz-\frac{\delta P}{\delta y}dxdy]\]
  \underline{Доказательство:}
  \begin{adjustwidth}{1.5em}{1.5em}
    \begin{minipage}{0.5\textwidth}
      \[(l): \begin{matrix}
        u=u(t)\\
        v=v(t)
      \end{matrix} \hspace{10pt} a\leq t \leq b\]
    \end{minipage}
    \hspace{1em}
    \begin{minipage}{0.5\textwidth}
      \[(L):\hspace{5pt} \begin{matrix}
        x=x(u(t);v(t))\\
        y=y(u(t);v(t))\\
        z=z(u(t);v(t))
      \end{matrix}\hspace{10pt} a\leq t\leq b\]
    \end{minipage}
    \vspace{1em}
    \par
    Рассмотрим 
    \[\int_{(L)} P(x;y;z)dx=\int_{a}^{b}P(x(u(t);v(t));y(u(t);v(t));z(u(t);v(t)))x'_t dt=\]
   
    \[\int_{l} P(x(u;v);y(u;v);z(u;v))dx=\int_{l}P(x(u;v);y(u;v);z(u;v))\Big[\frac{\delta x}{\delta u}du + \frac{\delta x}{\delta v}dv\Big]\hspace{5pt} \boxed{=}\]
   
    \[\text{Формула Грина: } \int_{(l)}Pdx+Qdy=\iint_{D_{xy}}\Big[ \frac{\delta Q}{\delta x} -\frac{\delta P}{\delta y}\Big]dxdy\]
   
    \[dx \sim du \hspace{20pt} P^*=P(x(u;v);y (u;v);z (u;v))\frac{\delta x}{\delta u}\]
   
    \[dy \sim dv \hspace{20pt} Q^*=P \frac{\delta x}{\delta V}\]
    
    \[\frac{\delta Q^*}{\delta u}=
    \Big(\frac{\delta P}{\delta x}\frac{\delta x}{\delta u}+
    \frac{\delta P}{\delta y}\frac{\delta y}{\delta u}+
    \frac{\delta P}{\delta Z}\frac{\delta Z}{\delta u}\Big)
    \frac{\delta x}{\delta v}+P \frac{\delta^2 x}{\delta u \delta v}\]

    \[\frac{\delta P^*}{\delta V}=
    \Big(\frac{\delta P}{\delta x}\frac{\delta x}{\delta v}+
    \frac{\delta P}{\delta y}\frac{\delta y}{\delta v}+
    \frac{\delta P}{\delta Z}\frac{\delta Z}{\delta v}\Big)+
    P\frac{\delta^2 x}{\delta V \delta u}\]

    \[\frac{\delta Q^*}{\delta u}-\frac{\delta P^*}{\delta V}=
    \frac{\delta P}{\delta Z}\Big(
      \frac{\delta Z}{\delta u}\frac{\delta x}{\delta v}-
      \frac{\delta Z}{\delta v}\frac{\delta x}{\delta u}
    \Big)-
    \frac{\delta P}{\delta y}\Big(
      \frac{\delta y}{\delta v}\frac{\delta x}{\delta u}-
      \frac{\delta y}{\delta v}\frac{\delta x}{\delta v}
    \Big)=
    \frac{\delta P}{\delta Z} \frac{D(z;x)}{D(u;v)}-\frac{\delta P}{\delta y}\frac{D(x;y)}{D(u;v)}=
    \frac{\delta P}{\delta Z}B-\frac{\delta P}{\delta y}C\]

    \[\boxed{=} \iint_{D_{uv}}\Big(
      \frac{\delta P}{\delta Z}B - \frac{\delta P}{\delta y}C
    \Big) dudv= \iint_S \frac{\delta P}{\delta z}dxdz - \frac{\delta P}{\delta y}dxdy\]

    Аналогично:
    \[\int_{(L)} Qdy = \iint_S \Big(
      \frac{\delta Q}{\delta x}dxdy - \frac{\delta Q}{\delta z}dydz
    \Big)\]
    \[\iint_{(L)}Rdz = \iint_S \Big(
      \frac{\delta R}{\delta y} dydz - \frac{\delta R}{\delta x}dzdx
    \Big)\]

    Тогда \[\int_{(L)} Pdx+Qdy+Rdz = 
    \iint_S
    \Big[\frac{\delta Q}{\delta x}-\frac{\delta P}{\delta y}\Big]dxdy+
    \Big[\frac{\delta R}{\delta y}-\frac{\delta Q}{\delta z}\Big]dydz+
    \Big[\frac{\delta P}{\delta Z}-\frac{\delta R}{\delta x}\Big]dxdz=\]
    \[\iint_S \Big(
      \Big[\frac{\delta Q}{\delta x}-\frac{\delta P}{\delta y}\Big]\cos(\gamma)+
      \Big[\frac{\delta R}{\delta y}-\frac{\delta Q}{\delta z}\Big]\cos(\alpha)+
      \Big[\frac{\delta P}{\delta z}-\frac{\delta R}{\delta x}\Big]\cos(\beta)
    \Big)dS\]
    \underline{Замечание:} \[\iint_{(L)} = 0 \Leftrightarrow \begin{matrix}
      \frac{\delta Q}{\delta x} = \frac{\delta P}{\delta y}\\
      \frac{\delta R}{\delta y} = \frac{\delta Q}{\delta z}\\
      \frac{\delta P}{\delta z} = \frac{\delta R}{\delta x}
    \end{matrix}\]
  \end{adjustwidth}
  \begin{center}
    \textbf{Ч.т.д.}
  \end{center}
  \subsection{Вычисление объёма с помощью поверхностных интегралов.}
  \[V = \iint_D f(x;y)dxdy\]
  Рассмотрим ограниченное тело, определяемое поверхностями: $\begin{matrix}
    S_1: z=z_1(x;y)\\
    S_2: z=z_2(x;y)\\
    S_3: \begin{matrix}
      \text{ цилиндрическая поверхность}\\
      \text{ с образующими || OZ}
    \end{matrix}
  \end{matrix}$
  Найти V\\
  $S_3$\hspace{5pt}\begin{minipage}{0.3\textwidth}
    \includegraphics[scale=0.6]{8.22.1.png}
  \end{minipage}
  \hspace{1em}
  \begin{minipage}{0.55\textwidth}
    \[V=\iint_{D_{xy}}z_2(x;y)dxdy -\iint_{D_{xy}}z_1(x;y)dxdy=\]
    \[\iint_{S_2}z(x;y)dxdy+\iint_{S_1}z(x;y)dxdy+\iint_{S_3}\equalto{Z}{0}dxdy=\iint{\equalto{S}{S_1+S_2+S_3}}z dxdy\]
    S - внешняя сторона объемного тела
  \end{minipage}
  \vspace{1em}
  \par
  Аналогично можно получить: \[ V=\iint_S x dydz \hspace{20pt} V=\iint_S y dxdz\]
  Для поверхностей общего вида: \[V = \frac{1}{3}\iint_S xdydz+ydzdx+zdxdy=\frac{1}{3}\iint_S(x\cos(\alpha)+y\cos(\beta)+z\cos(\gamma))dS\]
  \subsection{Тройные интегралы}
  Пусть в некоторой пространственной области V задана функция $f(x;y;z)$
  \begin{enumerate}
    \item Произведем разбиение R
    \item В каждой элементарной области $V_i$ произвольно выберем (.) $\mu_i (\xi_i;\eta_i;\chi_i)$
    \item Вычислим $f(M_i)$
    \item Вычислим $f(M_i)\Delta V_i$
    \item Составим $\sigma_R=\sum_{i=1}^{n}f(M_i)\Delta V_i$
    \item Вычислим $\lim_{\lambda_R \to 0}\sigma_R=\iiint_V f(x;y;z)dv = \iiint_V f(x;y;z) dxdydz$
  \end{enumerate}
  \underline{Определение: } Если существует конечный предел интегральной суммы $\sigma_R$, не зависящий от выбора (.) $M_i$
  и способа разбиения пространственной области V, то она называется тройным интегралом от $f(x;y;z)$ по объемному телу V.\\
  Для существования интеграла $\Leftrightarrow \lim_{\lambda_R \to 0} (\overline{S}-\underline{S})=0$
  \[\begin{matrix}
    \underline{S}=\sum_{i=1}^{n}m_i V_i & m_i = \underset{(x_i;y_i;z_i) \in V_i}{inf f}\\
    \overline{S}=\sum_{i=1}^{n}m_i V_i & M_i = \underset{(x_i;y_i;z_i) \in V_i}{sup f}
  \end{matrix}\]
  Справедливы все свойства определенного интеграла.\\
  \break
  \underline{Вычисление тройного интеграла}\\
  \begin{minipage}{0.45\textwidth}
    \includegraphics[scale=0.6]{8.23.1.png}
  \end{minipage}
  \hspace{1em}
  \begin{minipage}{0.55\textwidth}
    \[\iiint_V f(x;y;z) dxdydz = \iint_{D_{xy}}dxdy \Bigg[\int_{z_1(x;y)}^{z_2(x;y)}f(x;y;z)dz \Bigg]\]
    \[=\int_{a}^{b}dx \int_{y_1(x)}^{y_2(x)}\int_{z_1(x;y)}^{z_2(x;y)}dz=f(x;y;z)\]
  \end{minipage}
  \vspace{1em}
  \par
  \subsection{Формула Остроградского-Гаусса}
  Рассмотрим тело V, ограниченные кусочно-гладкими поверхностями. $\begin{matrix}
    S_1: z=z_1(x;y)\\
    S_2: z=z_2(x;y)\\
    S_3: \begin{matrix}
      \text{цилиндрическая поверхность}\\
      \text{с образующими || } OZ
    \end{matrix}
  \end{matrix}$\\
  Пусть в области V задана функция $R(x;y;z)$ непрерывная и имеет непрерывную частную производную $\frac{\delta R}{\delta z}$. 
  Тогда \[\iiint_V \frac{\delta R}{\delta z}dxdydz=\iint_S R dxdy\]
  \underline{Доказательство:}
    \[\text{Рассмотрим: }\underbrace{\iiint_V \frac{\delta R}{\delta z}dxdydz}=
    \iint_{D_{xy}}dxdy \Big[\int_{z_1(x;y)}^{z_2 (x;y)} \frac{\delta R}{\delta Z}dz\Big]=
    \iint_{D_{xy}}dxdy R(x;y;z_1(x;y))=\]
    \[\iint_{S_2}dxdy R(x;y;z)+
    \iint_{S_1}dxdy R(x;y;z)+
    \iint_{S_3}dxdy \equalto{R}{0}(x;y;z)=
    \underbracket{\iint_S dxdy R(x;y;z)}
    \]
  S - внешняя сторона поверхности, ограничивающей объемное тело V.
  \begin{center}
    \textbf{Ч.т.д.}
  \end{center}
  Аналогично: 
  \[\iiint_V \frac{\delta P}{\delta x}dxdydz = \iint_S P dydz \hspace{30pt} 
  \iiint_V \frac{\delta Q}{\delta y} dxdydz = \iint_S Q dxdz\]
  Общий вид:
  \[\iiint_V (\frac{\delta P}{\delta x}+\frac{\delta Q}{\delta y}+\frac{\delta R}{\delta z})dxdydz=
  \iint_S P dydz +Q dxdz+ R dxdy = 
  \iint_S(P\cos(\alpha)+Q\cos(\beta)+\cos(\gamma))dS\]
  \subsection{Замена переменных в тройном интеграле.}
  Рассмотрим две системы координат: $(x;y;z) \text{ и } (\xi,\eta,\chi)$\\
  Рассмотрим два замкнутых объемных тела: $V_D \in (x;y;z)$ \hspace{20pt} $V_D \in (\xi;\eta;\chi)$\\
  Оба тела ограничены поверхностями: $S \in (x;y;z) \hspace{20pt} \Sigma \in (\xi;\eta;\chi)$\\
  Пусть существует взаимно-однозначные соответствие между $(x;y;z)$ и $(\xi;\eta;\chi)$
  \[\begin{matrix}
    x=x(\xi;\eta;\chi)\\
    y=y(\xi;\eta;\chi)\\
    z=z(\xi;\eta;\chi)
  \end{matrix}\hspace{10pt}(1) \hspace{20pt}
  \begin{matrix}
    \xi=\xi(x;y;z)\\
    \eta=\eta(x;y;z)\\
    \chi=\chi(x;y;z)
  \end{matrix}\hspace{10pt}(2)\]
  \underline{Замечание:} при отображении (1) внутренней (.) переходит во внутреннюю, граничная (.) переходит в граничную.\\
  Пусть функции в (1)  имеют непрерывную частную производную.
  \[\text{Тогда} \hspace{10pt} \frac{D(x;y;z)}{D(\xi;\eta;\chi)} = I(\xi;\eta;\chi)=\begin{vmatrix}
    \frac{\delta x}{\delta \xi} & \frac{\delta x}{\delta \eta} & \frac{\delta x}{\delta \chi}\\
    \frac{\delta y}{\delta \xi} & \frac{\delta y}{\delta \eta} & \frac{\delta y}{\delta \chi}\\
    \frac{\delta z}{\delta \xi} & \frac{\delta z}{\delta \eta} & \frac{\delta z}{\delta \chi}\\
  \end{vmatrix}\]
  \[I(\xi;\eta;\chi)\not = 0 \text{ и сохраняет знак}\]
  Аналогично, как и в двойном интеграле, можно показать:
  \[\iiint_{V_D}dxdydz=\iiint_{V_\Delta}|I|d\xi d\eta d\chi\].
  \[|I|=\lim_{V_\Delta \to 0}\frac{V_D}{V_\Delta}\] Абсолютная величина Якобиана есть коэффициент растяжения(сжатия)
  пространства $(\xi;\eta;\chi)$ в пространство $(x;y;z)$\\
  \underline{Формула замены переменных в тройном интеграле:}
  \[\iiint_{V_D}f(x;y;z)dxdydz = \iiint_{V_\Delta}f(x(\xi;\eta;\chi);y(\xi;\eta;\chi);z(\xi;\eta;\chi))I d\xi d\eta d\chi\]
  \underline{Цилиндрическая система координат:}\\
  \begin{minipage}{0.45\textwidth}
    \includegraphics[scale=0.8]{8.25.1.png}
  \end{minipage}
  \hspace{1em}
  \begin{minipage}{0.55\textwidth}
    \[\begin{matrix}
      x=r\cos(\varphi)\\
      y=r\sin(\varphi)\\
      z=z
    \end{matrix}\hspace{20pt}
    I = \begin{vmatrix}
      -r\sin(\varphi) & \cos(\varphi) & 0\\
      r\cos(\varphi) & \sin(\varphi) & 0 \\
      0 & 0 & 1
    \end{vmatrix} =\]
    \[1 * A_{33}=M_{33}=-r\sin^2(\varphi)-r\cos^2(\varphi)=-r\]
    \[|I|=r\]
  \end{minipage}
  \vspace{1em}
  
  \par
  \underline{Сферическая система координат:}\\
  \par
  \begin{minipage}{0.45\textwidth}
    \includegraphics[scale=0.6]{8.25.2.png}
  \end{minipage}
  \hspace{1em}
  \begin{minipage}{0.55\textwidth}
    \[\varphi \in [0;2\pi]\]
    \[r \in [0;+\infty]\]
    \[\theta \in [0;\pi]\]
  \end{minipage}
  \vspace{1em}
  \par
  \begin{minipage}{0.45\textwidth}
    \includegraphics[scale=0.5]{8.25.3.png}
  \end{minipage}
  \hspace{1em}
  \begin{minipage}{0.55\textwidth}
    \[\begin{matrix}
      x=r\sin(\theta)\cos(\varphi)\\
      y=r\sin(\theta)\sin(\varphi)\\
      z=r\cos(\theta)
    \end{matrix}\]\[
    I = \begin{vmatrix}
      \sin(\theta)\cos(\varphi) & r\cos(\theta)\cos(\varphi) & -r\sin(\theta)\sin(\varphi)\\
      \sin(\theta)\sin(\varphi) & r\cos(\theta)\sin(\varphi) & -r\sin(\theta)\cos(\varphi)\\
      \cos(\theta)&-r\sin(\theta)&0
    \end{vmatrix}\]
    \[|I|=r^2\sin(\theta)\]
  \end{minipage}
  \vspace{1em}
  \par
  Рассмотрим сферу радиуса R с центром вращения в (.) (0;0;0)\\
  \begin{minipage}{0.45\textwidth}
    \includegraphics[scale=0.8]{8.25.4.png}
  \end{minipage}
  \hspace{1em}
  \begin{minipage}{0.55\textwidth}
    \[V-?\]
    \[V=\iiint_{V_D}dxdydz=\iiint_{V_\Delta}r^2\sin(\theta)drd\varphi d\theta=\]
    \[=\int_{0}^{2\pi}d\varphi \int_{0}^{\pi}d\theta \int_{0}^{R}dr r^2\sin(\theta)=
    \int_{0}^{2\pi}d\varphi \int_{0}^{\pi}d\theta \sin(\theta) \frac{R^3}{3}=\]
    \[=\frac{R^3}{3}\int_{0}^{2\pi}d\varphi 2=\frac{2}{3}R^3 2\pi=\frac{4 \pi R^3}{3}\]
  \end{minipage}
  \vspace{1em}
  \par
  \underline{Некоторые приложения тройного интеграла:}
  \begin{enumerate}
    \item \[V=\iiint_V dxdydz\]
    \item \[m=\iiint_V \rho(x;y;z)dxdydz\]
    \item \[
    \begin{matrix}
      M_{yz} = \iiint_{V} x\rho dxdydz & M_{xz}=\iint_V y \rho dxdydz & M_{xy}=\iint_V z\rho dxdydz\\
      x_c = \frac{M_{yz}}{m} & y_c=\frac{M_{xz}}{m} & z_c=\frac{M_{xy}}{m}
    \end{matrix}
    \]
  \end{enumerate}
  \subsection{n-кратные интегралы.}
  Рассмотрим $
  \begin{matrix}
    a_1 &\leq &x_1 &\leq &b_1\\
    a_2 &\leq &x_2 &\leq &b_2\\
    \vdots & &\vdots & & \vdots\\
    a_n &\leq &x_n &\leq &b_n
  \end{matrix}
  $
  \[P_n=[a_1 b_1][a_2 b_2]x\dots x[a_n b_n] \text{- n-мерный параллелепипед}\]
  Пусть в области V задана функция $y=f(x_1,\dots,x_n)$
  \begin{enumerate}
    \item Разбитие R объемного тела.
    \item Произвольным образов выберем (.) $\varepsilon_1,\dots,\varepsilon_n$.
    \item Вычислим $f(\varepsilon_1,\dots,\varepsilon_n)$
    \item Составим $\sigma_R=\sum_{i=1}^{n} f(\varepsilon_1,\dots,\varepsilon_n)\Delta V$
    \item $\lim_{\lambda_R \to 0} \sigma_R = \int \underset{V_n}{\dots} \int f(x_1,\dots,x_n)dx_1 \dots dx_n$
  \end{enumerate}
  \underline{Определение: } Если существует $\lim_{\lambda_R \to 0} \sigma_R$ независящей от R и выбора (.) $\varepsilon$,
  то он называется n-кратным интегралом от $f(x_1,\dots,x_n)$ по объемному телу V.\\
  \underline{Вычисление}\\
  Пусть V представима в виде: 
  \[\begin{matrix}
    x_1^* &\leq& x_1& \leq& x_1^1\\
    x_2^*(x_1) &\leq&x_2 &\leq&x_2^1(x_1)\\
    x_3^*(x_1;x_2)&\leq& x_3 &\leq&x_3^1(x_1;x_2)\\
                  &&\vdots&&\\
    x_n^*(x_1,\dots,x_{n-1})&\leq&x_n&\leq&x_n^1(x_1;\dots;x_{n-1})
  \end{matrix}\]
  \[\text{Тогда } \int \underset{V_n}{\dots} \int f(x_1;\dots;x_n) dx_1 \dots dx_n=
  \int_{x_1^*}^{x_1^1}dx_1 \int_{x_2^*(x_1)}^{x_2^1(x_1)}dx_2 \dots \int_{x_3^*(x_1;\dots;x_{n-1})}^{x_3^1(x_1;\dots;x_{n-1})}
  [f(x_1;\dots;x_n)]\]
  \underline{Замена переменных в n-кратном интеграле.}\\
  Пусть заданы 2 n-мерные области
  \[V_D \text{ и } V_\Delta\]
  \begin{minipage}{0.45\textwidth}
    \[\begin{cases}
      x_1&=x_1(\xi_1;\dots;\xi_n)\\
      \vdots&\\
      x_n&=x_n(\xi_1;\dots;\xi_n)
    \end{cases} \circled{1}\]
  \end{minipage}
  \hspace{1em}
  \begin{minipage}{0.55\textwidth}
    \[\begin{cases}
      \varepsilon_1&=\varepsilon_1(x_1;\dots;x_n)\\
      \vdots&\\
      \varepsilon_n&=\varepsilon_n(x_1;\dots;x_n)
    \end{cases} \circled{2}\]
  \end{minipage}
  \vspace{1em}
  \par
  \[\frac{D(x_1;\dots;x_n)}{D(\xi_1;\dots;\xi_n)}=I(\xi_1;\dots;\xi_n)
  \begin{vmatrix}
    \frac{\delta x_1}{\delta \xi_1}&\dots&\frac{\delta x_1}{\delta \xi_n}\\
    \vdots &&\vdots \\
    \frac{\delta x_n}{\delta \xi_1}&\dots&\frac{\delta x_n}{\delta \xi_n}
  \end{vmatrix}\]
  Пусть $I(\xi_1;\dots;\xi_n)\not =0$ и сохраняет знак в каждой (.).\\
  Тогда 
  \[\int \underset{V_D}{\dots} \int f(x_1;\dots;x_n)dx_1\dots dx_n = \int \underset{V_D}{\dots} \int 
  f(x_1(\xi_1;\dots;\xi_n);\dots;x_n(\xi_1;\dots;\xi_n)|I|d\xi_1\dots d\xi_n)\]
  \section{Теория Поля.}
  \subsection{Основные понятия.}
  \underline{Определения: }\\
  Скаляр - величина характеризующиеся своим числовым значением.\\
  Вектор - величина характеризующиеся числовым значением и направлением.\\
  Если с каждой (.) m некоторой пространственной области связана некоторая скалярная или векторная величина то говорят что
  задано поле этой величины. Заданное поле скалярной величины равносильно заданной скалярной функции и $(x;y;z)$.\\
  Чтобы задать векторное поле необходимо задать:
  \[A_x(x;y;z) \hspace{20pt} A_y(x;y;z) \hspace{20pt} A_z(x;y;z)\]
  \[\overline{A}=\underbracket{A_x i}_{P}+\underbracket{A_y j}_{Q}+\underbracket{A_z k}_{R}\]
  
  \begin{minipage}{0.55\textwidth}
    \underline{Определение: } Векторной линией называется кривая касательной которой в каждой ее (.) m совпадает
  с направление $\overline{A}$ отвечающим этой (.).\\
  \end{minipage}
  \hspace{1em}
  \begin{minipage}{0.45\textwidth}
    \includegraphics[scale=0.6]{9.1.png}
  \end{minipage}
  \vspace{1em}
  \par
  \underline{Определение: } Поверхность составленной из векторных линий называется векторной поверхностью.
  Характеризуется: в каждой ее (.) M. Соответствующий вектор $\overline{A}(M)$ лежит в касательной плоскости к
  этой поверхности в этой (.).
  Если взять в рассматриваемой области какую-нибудь линию отличную от векторной и через каждую ее
  (.) провести векторную линию то геометрическое место этих линий задает векторную поверхность.\\
  Если исходная линия была замкнутой, то полученная таким образом векторная поверхность называется 
  векторной трубкой.
  \break
  \subsection{Градиент}
  Пусть задана скалярное поле $u(x;y;z)$
  \[\overline{grad u} = \Big(\frac{\delta u}{\delta x};\frac{\delta u}{\delta y};\frac{\delta u}{\delta z}\Big)\]
  \[\frac{\delta u}{\delta l}=\overline{grad u} * \overline{l_0}=
  \frac{\delta u}{\delta x}\cos(\alpha)+
  \frac{\delta u}{\delta y}\cos(\beta)+
  \frac{\delta u}{\delta z}\cos(\gamma)\]
  \begin{center}
  Максимальное значение $\frac{\delta u}{\delta l}$ имеет в направлении вектора $\overline{gradu}$ 
  \end{center}
  \[\overline{\triangledown}\Big(\frac{\delta u}{\delta x};\frac{\delta u}{\delta y};\frac{\delta u}{\delta z}\Big)
  \overline{\triangledown} \text{- оператор Набла(оператор Гамильтона)}\]
  \[\overline{gradu}=\overline{\triangledown u}\]
  \subsection*{Дифференциальные свойства градиента}
  \begin{enumerate}
    \item \[\overline{grad}(u_1+u_2)=\overline{grad u_1}+\overline{grad u_2}\]
    \item \[\overline{grad(C_u)}=C\overline{grad u}, C=const\]
    \item \[\overline{grad}(u_1;u_2)=u_1\overline{grad u_2}+u_2\overline{grad u_1}\]
  \end{enumerate}
  \underline{Доказательство:}
  \begin{adjustwidth}{1.5em}{1.5em}
    \[\overline{grad}(u_1;u_2)=
    \frac{\delta}{\delta x}(u_1;u_2)\overline{i}+\frac{\delta}{\delta y}(u_1;u_2)\overline{j}+\frac{\delta}{\delta z}(u_1;u_2)\overline{k}=\]\[
    (u_1 \frac{\delta u_2}{\delta x}+u_2\frac{\delta u_1}{\delta x})\overline{i} + (u_1 \frac{\delta u_2}{\delta y}+u_2\frac{\delta u_1}{\delta y})\overline{j} + (u_1\frac{\delta u_2}{\delta z}+u_2\frac{\delta u_1}{\delta z})\overline{k}=\]
  
    \[u_1(\frac{\delta u_2}{\delta x}\overline{i} +\frac{\delta u_2}{\delta y}\overline{j}+\frac{\delta u_2}{\delta z}\overline{k})+
  u_2(\frac{\delta u_1}{\delta x}\overline{i} +\frac{\delta u_1}{\delta y}\overline{j}+\frac{\delta u_1}{\delta z}\overline{k})=\
  \]\[u_1 \overline{grad u_2}+u_2 \overline{grad u_1}\]
  \end{adjustwidth}
  \underline{Замечание:}Направление вектора градиента совпадает с направлением вектора нормали к поверхности уровня $u(x;y;z)=const.$\\
  \underline{Замечание:}Скалярное поле $u(x;y;z)$ порождает векторное поле градиента $\overline{grad u}(x;y;z)$

  \subsection{Поток вектора через поверхность.}
  Пусть задано векторное поле $\overline{A}(M)$, т.е. заданы функциями
  \[\begin{matrix}
    A_x(x;y;z)=P(x;y;z)\\
    A_y(x;y;z)=Q(x;y;z)\\
    A_z(x;y;z)=R(x;y;z)
  \end{matrix} \overline{A}(P;Q;R)\]
  Возьмем некоторую поверхность S, выбрав определенную ее строку\\
  \underline{Обозначение} $\overline{\cos(\alpha);\cos(\beta);\cos(\gamma)}=\overline{n_0}$\\
  \underline{Определение: } Потоком векторного поля $\overline{A}$ через поверхность S в сторону единичной нормали\\
  $\overline{n}(\cos(\alpha);\cos(\beta);\cos(\gamma))$ называется 
  \[\iint_S (P\cos(\alpha)+Q\cos(\beta))+\cos(\gamma)dS=\iint_S \overline{A} \overline{n}dS = \oiint_S A_n dS=\text{ П}\]
  Рассмотрим движение жидкости в пространстве. Движение нестационарное. Вычислим количество жидкости, протекающее
  через поверхность S в направлении $\overline{n}$ за малый промежуток времени $dt$. За время $dt$ через $ds$ протечет
  количество жидкости, которое заполняет собой цилиндр с основанием $ds$ и высотой $V_n dt$.\\
  \begin{minipage}{0.45\textwidth}
    \includegraphics[scale=0.6]{9.2.png}
  \end{minipage}
  \hspace{1em}
  \begin{minipage}{0.55\textwidth}
    \[dm=\rho dV=\rho ds * V_n * dt\]
    \[m=dt \iint_{ds} V_n * \rho * ds\]
    \[\frac{m}{dt}=\iint_{ds} \rho V_n ds = Q\]
  \end{minipage}
  \vspace{1em}
  \par
  $Q$ - количество жидкости в единицу времени, поток вектора $\rho \overleftarrow{V_n}$ через поверхность S.\\
  Поток - скалярная величина, если угол между $\overline{A},\overline{n}$:
  \begin{itemize}
    \item острый $\Rightarrow$ П>0
    \item тупой $\Rightarrow$ П<0
    \item $\varphi = \frac{\pi}{2} \Rightarrow $ П=0
  \end{itemize}
  \[\text{П=} \iint_S (P\cos(\alpha)+Q\cos(\beta)+R\cos(\gamma))ds=\iiint_V
  (\frac{\delta P}{\delta x}+\frac{\delta Q}{\delta y}+\frac{\delta R}{\delta Z})dxdydz\]
  \underline{Определение: } Выражение 
  $\frac{\delta P}{\delta x}+\frac{\delta Q}{\delta y}+\frac{\delta R}{\delta Z}$ называется
  дивергенцией(расходимостью) векторного поля $\overline{A}$\\
  \underline{Обозначение:} $div \overline{A}$
  \[\text{П=} \iint_S \overline{A} \overline{n}dS=\iiint_V div \overline{A} dV\]
  \underline{Замечание:} Дивергенция - скаляр 
  \[\text{П=} \iiint_V div \overline{A} dV \underset{\hyperref[th:8.12.1]{\text{Th. о среднем}}}{=}
  \overline{A}\Big|_{(\xi,\eta,\chi)} \iiint_V dV= div\overline{A}\Big|_{(\xi,\eta,\chi)} *V \]
  \[div \overline{A} = \lim_{V \rightarrow (\xi,\eta,\chi)} \frac{\text{П}}{V}\]
  \subsection*{Физический смысл потока}
  Пусть $\overline{A}$ - векторное поле скоростей нестикаемой жидкости $(\rho = const)$ при наличии источников
  (стоков)
  \[\text{П} = \rho \iint_S \overline{V} \overline{n}dS\]
  \begin{itemize}
    \item П>0 - значит, что из области V, вытекает больше жидкости, чем втекает. Это означает, что внутри области
    V есть источники.
    \item П<0 - значит, что из области V вытекает меньше жидкости, чем втекает. Это означает что внутри области V есть стоки.
    \item П=0 - сколько жидкости втекает, столько и вытекает.
  \end{itemize}
  \subsection*{Физический смысл дивергенции:}
  Если источники(стоки) распределены непрерывно, то вводится понятие плотности источника(стока).
  \[div \overline{A}\Big|_M = \lim_{V \to M} \frac{\text{П}}{V} \text{- плотность источников(стоков)}\]
  \begin{itemize}
    \item $div \overline{A}\Big|_M >0$ - (.) M - источник.
    \item $div \overline{A}<0$ - (.) M - сток.
    \item $div\overline{A}\Big|_M = 0$ - (.) M ни источник, ни сток.
  \end{itemize}
  \subsection*{Свойства дивергенции:}
  \begin{enumerate}
    \item $div(\overline{a}+\overline{b})=div\overline{a}+div\overline{b}(\overline{a},\overline{b} - \text{векторные поля})$
    \item $div \overline{c} = 0, \overline{c} = const$
    \item $div(f\overline{a}) = f div\overline{a}+\overline{a}*\overline{gradf},$ f - скалярная функция $f(x,y,z)$
  \end{enumerate}
  \underline{Доказательство:}
  \begin{adjustwidth}{1.5em}{1.5em}
    \[div(f\overline{a})=\frac{\delta fP}{\delta X}+\frac{\delta f Q}{\delta y}+\frac{\delta f R}{\delta Z}=
    f\frac{\delta P}{\delta x}+P\frac{\delta f}{\delta x}+f \frac{\delta Q}{\delta y}+Q\frac{\delta f}{\delta y}+
    f\frac{\delta R}{\delta Z}+R\frac{\delta f}{\delta z}=\]
    \[f div \overline{a}+\overline{a}*\overline{gradf}\]
  \end{adjustwidth}
  \begin{center}
    \textbf{Ч.т.д.}
  \end{center}
  \underline{Замечание:} 
  \[\overline{\bigtriangledown} = \Big(\overline{\frac{\delta \underbracket{\hspace{10pt}}}{\delta x}
  ;\frac{\delta \underbracket{\hspace{10pt}}}{\delta y};\frac{\delta \underbracket{\hspace{10pt}}}{\delta z}}\Big)
  \text{ - оператор Гамильтона(оператор Набла)}\]
  \[div \overline{A} = \overline{\bigtriangledown}*\overline{A}\]
  \[\overline{\bigtriangledown} = \frac{\delta}{\delta x}\overline{i}+
  \frac{\delta}{\delta y}\overline{j}+\frac{\delta}{\delta z}\overline{k}\]

  \subsection{Циркуляция вектора}
  Пусть задано векторное поле $\overline{A}(P;Q;R)$ и замкнутая кривая $l$ в пределах рассматриваемой
  области.\\
  \underline{Определение: }$\int_{(l)} Pdx+Qdy+Rdz$ - называется \underline{циркуляцией} векторного поля
  $\overline{A}$ вдоль кривой l.\\
  \underline{Обозначение:} 
  \[\text{ц} = \oint_{(l)} Pdx+Qdy+Rdz= \oint_{(l)}[P\cos(\alpha)+Q\cos(\beta)+R\cos(\gamma)]dl=\]
  \[\oint_{(l)}(\overline{P;Q;R})(\overline{\cos(\alpha);\cos(\beta);\cos(\gamma)}dl=
  \oint_{(l)})\overline{A}*\overline{e}\]
  $\overline{e}$ - вектор, касательный к кривой в каждой(.)\\
  \underline{Замечание:} Если $\overline{A}$ - силовое поле, то Ц равна работе сил этого поля по
  перемещению (.) по кривой (l).
  \[\text{Ц=} \oint_{(l)} \overline{A}*\overline{e}dl \underset{\text{Формула Стокса}}{=}
  \iint_S \Big[\Big(\frac{\delta R}{\delta y}-\frac{\delta Q}{\delta z}\cos(\alpha)\Big)
  + \Big(\frac{\delta P}{\delta Z} - \frac{\delta R}{\delta x}\Big)\cos(\beta)
  + \Big(\frac{\delta Q}{\delta x}-\frac{\delta P}{\delta y}\cos(\gamma)\Big)\Big]dS\]
  \[\overline(n)(\cos(\alpha);\cos(\beta);\cos(\gamma)) - \text{нормаль к поверхности}\]
  \underline{Определение: } Вектор с координатами $\Big(\frac{\delta R}{\delta y}-\frac{\delta Q}{\delta z};
  \frac{\delta P}{\delta z}-\frac{\delta R}{\delta x};\frac{\delta Q}{\delta x} - \frac{\delta P}{\delta y}\Big)$
  называется ротором(вихрем) векторного поля $\overline{A}$.\\
  \underline{Обозначение: } $\overline{rotA}$
  \[\oint_{(l)}\overline{A}\overline{e}dl = \iint_S \overline{rotA}*\overline{n_0}dS = \text{ц}\]
  То есть циркуляция векторного поля $\overline{A}$ вдоль замкнутого контура l равна потоку ротора этого поля
  $\overline{rotA}$ через поверхность S, границей которого является l.
  \[\overline{rotA} = \overline{\bigtriangledown} X \overline{A} = 
  \begin{vmatrix}
    \overline{i} & \overline{j} &\overline{k}\\
    \frac{\delta}{\delta x} & \frac{\delta }{\delta y} & \frac{\delta }{\delta z}\\
    P & Q & R
  \end{vmatrix}\]
  \subsection*{Свойство ротора:}
  \begin{enumerate}
    \item $\overline{rot}(\overline{a}+\overline{b}) = \overline{rota}+\overline{rotb}$
    \item $\overline{rot}(f\overline{a}) = f\overline{rota}+\overline{gradf}X\overline{a}$
  \end{enumerate}
  \pagebreak
  \underline{Доказательство:}
  \begin{adjustwidth}{1.5em}{1.5em}
    \[\overline{rot}(f\overline{a}) = 
    \begin{vmatrix}
      \overline{i} & \overline{j} &\overline{k}\\
      \frac{\delta}{\delta x} & \frac{\delta }{\delta y} & \frac{\delta }{\delta z}\\
      fP & fQ & fR
    \end{vmatrix} = 
    i\Big(\frac{\delta fR}{\delta y} - \frac{\delta f Q}{\delta z}\Big) 
   -j\Big(\frac{\delta fR}{\delta x} - \frac{\delta f P}{\delta z}\Big)
   +k\Big(\frac{\delta fQ}{\delta x} - \frac{\delta f P}{\delta y}\Big)=\]
    \[=i\Big(\frac{\delta f}{\delta y}R + f\frac{\delta R}{\delta y} - \frac{\delta f}{\delta z}Q-f\frac{\delta Q}{\delta z}\Big)-
    j\Big(\frac{\delta f}{\delta x}R + \frac{\delta R}{\delta x}f - \frac{\delta f}{\delta z}P - f\frac{\delta P}{\delta z}\Big)+
    k\Big(\frac{\delta f}{\delta x}Q+\frac{\delta Q}{\delta x}P - \frac{\delta f}{\delta y}P - f\frac{\delta P}{\delta y}\Big)=\]
    \[f\Big(\frac{\delta R}{\delta y}-\frac{\delta Q}{\delta z};
        \frac{\delta P}{\delta z}-\frac{\delta R}{\delta x};
        \frac{\delta Q}{\delta x}-\frac{\delta P}{\delta y}\Big)=
    \begin{vmatrix}
      \overline{i} & \overline{j} &\overline{k}\\
      \frac{\delta}{\delta x} & \frac{\delta }{\delta y} & \frac{\delta }{\delta z}\\
      P & Q & R
    \end{vmatrix}=\]
    \[f * \overline{rot}\overline{a}+\overline{grad}f x \overline{a}\]
  \end{adjustwidth}
  \begin{enumerate}
    \setcounter{enumi}{2}
    \item Если $\overline{c}=const,\overline{rot} \overline{c}=\overline{0}$
    \item Если $\overline{r}=x \overline{i}+y\overline{j}+z\overline{k},\overline{rot}\overline{r} = \overline{0}$
  \end{enumerate}
  \subsection*{Формула Стокса:}
  \[\int_{(l)} = \overline{A}\overline{e}dl 
  = \iint_S \overline{rot}\overline{A}*\overline{n_0} dS 
  = \iint_S rot A_n dS
  = rot A_n \Big|_{(\xi,\eta,\chi)} \iint_S dS\]
  \[rot A_n = \lim_{(S) \to M} \frac{\int_{(l)} \overline{A}\overline{e}dl}{S}\]
  Ротор - это вектор, проекция которого на каждое направление равна пределу отношения циркуляции векторного
  поля по кривой l плоской площади, перпендикулярной к этому направлению, к площади этой площади, 
  когда площадка стягивается в точку.\\
  \underline{Замечание:} Если $\overline{A}$ - векторное поле скоростей, то 
  $\overline{rot}\overline{A} = \overline{\omega}$ - угловая скорость.
  \subsection*{Дифференциальные операции второго порядка:}
  \begin{enumerate}
    \item $div \overline{grad}u= \overline{\bigtriangledown}(\overline{\bigtriangledown} u) = 
    \frac{\delta }{\delta x}*\frac{\delta u}{\delta x}+
    \frac{\delta}{\delta y}*\frac{\delta  u}{\delta y}+
    \frac{\delta}{\delta z}*\frac{\delta u}{\delta z}=
    \frac{\delta^2 u}{\delta x^2}+\frac{\delta^2 u}{\delta y^2}+\frac{\delta^2 u}{\delta z^2}$
    \[\text{Оператор }\bigtriangleup = 
    \frac{\delta^2}{\delta x^2 }+\frac{\delta^2}{\delta y^2}+\frac{\delta^2}{\delta z^2} \text{ - оператор Лапласа}\]
    \[div\overline{grad}u=\bigtriangleup u\]

    \item $\overline{rot} \overline{gradu} = 
    \begin{vmatrix}
      i & j & k\\
      \frac{\delta}{\delta x} & \frac{\delta}{\delta y} & \frac{\delta}{\delta z}\\
      \frac{\delta u}{\delta x} & \frac{\delta u}{\delta y} & \frac{\delta u}{\delta z}
    \end{vmatrix}
    =i\Big(\frac{\delta^2 u}{\delta y \delta z}-\frac{\delta^2 u}{\delta y \delta z}\Big)
    -j\Big(\frac{\delta^2 u}{\delta x \delta z}-\frac{\delta^2 u}{\delta x \delta z}\Big)
    +k\Big(\frac{\delta^2 u}{\delta x \delta y}-\frac{\delta^2 u}{\delta x \delta y}\Big)=
    \overline{o}$

    \item $\overline{grad}div \overline{A}=\overline{\bigtriangledown}(\bigtriangledown A)=
    \frac{\delta}{\delta x}\Big(\frac{\delta P}{\delta x}+\frac{\delta Q}{\delta y}+\frac{\delta R}{\delta z}\Big)\overline{i}
  + \frac{\delta}{\delta y}\Big(\frac{\delta P}{\delta x}+\frac{\delta Q}{\delta y}+\frac{\delta R}{\delta z}\Big)\overline{j}
  + \frac{\delta}{\delta z}\Big(\frac{\delta P}{\delta x}+\frac{\delta Q}{\delta y}+\frac{\delta R}{\delta z}\Big)\overline{k}
  = \overline{i}\Big(\frac{\delta^2 P}{\delta x^2}+\frac{\delta^2 Q}{\delta x \delta y}+\frac{\delta^2 R}{\delta x \delta z}\Big)
  + \overline{j}\Big(\frac{\delta^2 P}{\delta x \delta y}+\frac{\delta^2 Q}{\delta y^2}+\frac{\delta^2 R}{\delta x \delta z}\Big)
  + \overline{k}\Big(\frac{\delta^2 P}{\delta x \delta y}+\frac{\delta^2 Q}{\delta y \delta z}+\frac{\delta^2 R}{\delta z^2}\Big)$
    
    \item $div \overline{rot}\overline{A} = \frac{\delta}{\delta x}\Big(\frac{\delta R}{\delta y}-\frac{\delta Q}{\delta z}\Big)
    + \frac{\delta}{\delta y}\Big(\frac{\delta P}{\delta z} - \frac{\delta R}{\delta x}\Big)+
    \frac{\delta}{\delta z }\Big(\frac{\delta Q}{\delta x} - \frac{\delta P}{\delta y}\Big)=0$

    \item $\overline{rot}\overline{rotA} =
    \begin{vmatrix}
      \overline{i} & \overline{j} & \overline{k}\\
      \frac{\delta}{\delta x} & \frac{\delta}{\delta y} & \frac{\delta}{\delta z}\\
      \frac{\delta R}{\delta y} - \frac{\delta Q}{\delta z} & 
      \frac{\delta P}{\delta z} - \frac{\delta R}{\delta x} &
      \frac{\delta Q}{\delta x} - \frac{\delta P}{\delta y}
    \end{vmatrix}
    =\overline{i}\Big(\frac{\delta^2 Q}{\delta x \delta y} - \frac{\delta^2 R}{\delta y^2}-
    \frac{\delta^2 P}{\delta z^2} + \frac{\delta^2 R}{\delta x \delta z}-\frac{\delta^2 P}{\delta x^2}+\frac{\delta^2 P}{\delta x^2}\Big)-
    \overline{j}\Big(\frac{\delta^2 Q}{\delta x^2} - \frac{\delta^2 P}{\delta x \delta y} - \frac{\delta^2 R}{\delta y \delta z} +
    \frac{\delta^2 Q}{\delta z^2}+\frac{\delta^2 Q}{\delta y^2}-\frac{\delta^2 Q}{\delta y^2}\Big)+
    \overline{k}\Big(\frac{\delta^2 P}{\delta x \delta z} - \frac{\delta^2 R}{\delta x^2} - \frac{\delta^2 R}{\delta y^2}+
    \frac{\delta^2 Q}{\delta y \delta z} - \frac{\delta^2 R}{\delta z^2}+ \frac{\delta^2 R}{\delta z^2}\Big)=
    \overline{i}\Big(\frac{\delta^2 P}{\delta x^2}+\frac{\delta^2 Q}{\delta x \delta y}+\frac{\delta^2 R}{\delta x \delta z}\Big)+
    \overline{j}\Big(\frac{\delta^2 Q}{\delta y^2}+\frac{\delta^2 P}{\delta x \delta y}+ \frac{\delta^2 R}{\delta y \delta z}\Big)+
    \overline{k}\Big(\frac{\delta^2 R}{\delta z^2}+\frac{\delta^2 P}{\delta x \delta z} + \frac{\delta^2 Q}{\delta y \delta z}\Big)-
    \overline{i}\Big(\frac{\delta^2 P}{\delta x^2}+\frac{\delta^2 P}{\delta y^2}+\frac{\delta^2 P}{\delta z^2}\Big)-
    \overline{j}\Big(\frac{\delta^2 Q}{\delta x^2}+\frac{\delta^2 Q}{\delta y^2}+\frac{\delta^2 Q}{\delta z^2}\Big)-
    \overline{j}\Big(\frac{\delta^2 R }{\delta x^2}+\frac{\delta^2 R}{\delta y^2}+\frac{\delta^2 R}{\delta z^2}\Big)=
    \overline{grad}div \overline{A}=\overline{\bigtriangleup A}$
    \[\overline{\bigtriangleup A}(\Delta P;\Delta Q;\Delta R)\]
\end{enumerate}
\subsection{Виды векторных полей}
\begin{enumerate}
  \item Потенциальное поле\\
  \underline{Определение: } Векторное поле $\overline{A}$ называется потенциальным, если существует скалярная функция,
  для которой $\overline{A}$ является градиентом.
  \[\overline{A}=\overline{gradu},\text{т. е.}\]
  \[P=\frac{\delta u}{\delta x};Q=\frac{\delta u}{\delta y}; R= \frac{\delta u}{\delta z}\]
  \[Pdx+Qdy+Rdz=du\]
  \underline{Определение: } Функция $u(x;y;z)$ называется потенциальной функцией поля $\overline{A}$.\\
  \underline{Утверждение: } Для того чтобы поле $\overline{A}$ было потенциальным $\Leftrightarrow$ чтобы 
  \[\frac{\delta R}{\delta y}=\frac{\delta Q}{\delta z};\frac{\delta P}{\delta z}=\frac{\delta R}{\delta x};
  \frac{\delta Q}{\delta x}=\frac{\delta P}{\delta y} \Rightarrow \boxed{\overline{rot}\overline{A}=\overline{o}}\]
  Потенциальное поля называют безвихревым полем.\\
  Для потенциального поля справедливо:
  \begin{itemize}
    \item Циркуляция по простому замкнутому контуру всегда равна 0.
    \item Интеграл по кривой, соединяющий две (.) поля, не зависит от формы кривой(не зависит от пути интегрирования).
    \item  Скалярная функция u определяется с точностью до постоянной.
  \end{itemize}
  \item Соленоидальное поле\\
  \underline{Определение: } Векторное поле А называется соленоидальным(трубчатым), если существует
  функция $\overline{B}$, для которой А служит ротором.
  \[\overline{A}=\overline{rot}\overline{B}; 
  A_x=\frac{\delta B_z}{\delta y}-\frac{\delta B_y}{\delta z};
  A_y=\frac{\delta B_x}{\delta z}-\frac{\delta B_z}{\delta x};
  A_z=\frac{\delta B_y}{\delta x}-\frac{\delta B_x}{\delta y}\]
  $\overline{B}$ называется векторным потенциалом поля $\overline{A}$.\\
  \underline{Утверждение: } Для того чтобы векторное поле было соленоидальным $\Leftrightarrow \boxed{\overline{div}\overline{A}=0}$\\
  \pagebreak
  \underline{Доказательство:}
  \begin{adjustwidth}{1.5em}{1.5em}
    \circled{$\Rightarrow$} Поле соленоидальное, тогда по определению существует $\overline{B}:A=\overline{rot}\overline{B}$. 
    Вычислим $div\overline{A}=div(\overline{rot}\overline{B})=0$\\
    \circled{$\Leftarrow$} Пусть $div \overline{A}=0.$ Надо показать, что существует $\overline{B}:A=\overline{rot}\overline{B}$
    \[(\overline(A_x;A_y;A_z)) \text{ - известно} \hspace{20pt} 
    (\overline{B_x;B_y;B_z}) \text{ - неизвестно}\]
    Достаточно найти какое-нибудь одно частное решение.
    Пусть $B_z=0$
    \[A_x = - \frac{\delta B_y}{\delta B_z}; \hspace{10pt}
      A_y = \frac{\delta B_x}{\delta B_z}; \hspace{20pt}
      A_z = \frac{\delta B_y}{\delta x} - \frac{\delta B_x}{\delta y}\]
    \[B_y = - \int_{z_0}^{z}A_x(x;y;z)dz+p(x;y) ; \hspace{20pt}
      B_x = \int_{z_0}^{z}A_y(x;y;z)dz\]
      Найдем $\varphi(x;y)$
      \[\frac{\delta B_y}{\delta x}= - \int_{z_0}^{z}\frac{\delta A_x}{\delta x}dz+\varphi'_x(x;y) \hspace{20pt}
      \frac{\delta B_x}{\delta y} = \int_{z_0}^{z}\frac{\delta A_y}{\delta y}dz\]
      Тогда
      \[A_z = - \int_{z_0}^{z}\Big[\frac{\delta A_x}{\delta x}+ \frac{\delta A_y}{\delta y}\Big]dz
      +\varphi'_x(x;y)=
      \begin{vmatrix}
        \frac{\delta A_x}{\delta x}+\frac{\delta A_y}{\delta y}+\frac{\delta A_z}{\delta z}=0\\
        \frac{\delta A_x}{\delta x }+\frac{\delta A_y}{\delta y}= -\frac{\delta A_z}{\delta z}
      \end{vmatrix}=\]
      \[=\int_{z_0}^{z}\frac{\delta A_z}{\delta z}dz + \varphi'_x(x;y)=A_z(x;y;z)-A_z(x;y;z_0)+\varphi'_x(x;y) \Rightarrow\]
      \[\varphi'_x(x;y)=A_z(x;y;z_0) \hspace{20pt} \varphi(x;y)=\int_{x_0}^{x}A_z(x;y;z_0)dx+\psi(y)\]
      \[B_x=\int_{z_0}^{z}A_y(x;y;z)dz \hspace{20pt} 
      B_y = - \int_{z_0}^{z}A_x(x;y;z)dz+\int_{x_0}^{x}A_z(x;y;z_0)dx+\psi(y) \hspace{20pt}
      B_z =0\]
    \end{adjustwidth}
    \begin{center}
      \textbf{Ч.т.д.}
    \end{center}
    \item Разложение произвольного векторного поля.\label{punkt3} \\
    Пусть A - произвольное векторное поле.
    \begin{itemize}
      \item $\overline{A}=\overline{A}'+\overline{A}''$
      \item $\overline{A}'$ - потенциальная составляющая($\overline{rot}\overline{A}=0$)
      \item $\overline{A}''$ - соленоидальная составляющая ($div \overline{A}'=0$)
    \end{itemize}
    Пусть $F$ - скалярная функция. \hspace{20pt} $F-?$ \hspace{10pt} Пусть $\overline{A}' = \overline{gradF} \hspace{20pt} rotgradF=0$\\
    Тогда $\overline{A}'' = \overline{A}-\overline{A}'$.
    \[div \overline{A}'' = div\overline{A}' = div \overline{A}-divgradF=div \overline{A}-\Delta F=0 \hspace{10pt} \Delta F=divA\]
    \clearpage
    \underline{Алгоритм:}
    \begin{enumerate}
      \item[1.] $\Delta F = divA \Rightarrow F$
      \item[2.] $\overline{A}' = gradF$
      \item[3.] $\overline{A}'' = \overline{A} - \overline{A}'$
    \end{enumerate}
    \item Гармоническое поле\\
    \underline{Определение: } Векторное поле $\overline{A}$ называется гармоническим, если оно и потенциальное и
    соленоидальное одновременное; т.е.
    \[div \overline{A} = 0 \hspace{20pt} \overline{rot}\overline{A}=0\]
    Пусть $F$ - скалярная функция \hspace{20pt} $F-?$\\
    Пусть $\overline{A} = \overline{grad}F(\overline{rot}\overline{A}=rot\overline{grad}F=0)$
    \[div \overline{A} = div\overline{grad}F=0 \hspace{20pt} \Delta F=0\]
    Алгоритм
    \[\frac{\delta^2 F}{\delta x^2}+\frac{\delta^2 F}{\delta y^2}+\frac{\delta^2 F}{\delta z^2} = 0 \Rightarrow F\]
    \[\overline{A}=\overline{grad}F\]
    \item Обратная задача векторного анализа\\
    \underline{Задача}: Найти векторное поле $A$, если известно
    \[div \overline{A} = F \hspace{20pt} \overline{rot}\overline{A}=\overline{B}\]
    \underline{Решение}
    \begin{center}
      Решение ищем в виде: $\overline{A}=\overline{A'}+\overline{A''}$
    \end{center}
    \[(1):\begin{matrix}
      \overline{rot}\overline{A'}=0\\
      div \overline{A'}=F
    \end{matrix}
    \hspace{20pt} 
    (2):\begin{matrix}
      \overline{rot} \overline{A''}=\overline{B}\\
      div \; A'' = 0
    \end{matrix}\]
    \begin{enumerate}
      \item[(1):] Пусть Ф - скалярная функция. Тогда $\overline{A'}$ можно представить в виде
      \[\overline{A'}=\overline{\text{gradФ}},\text{т.к. } \overline{rot}\overline{grad}\text{Ф}=0\]
      Тогда $div \overline{A'}=div\overline{grad}\text{Ф}=\Delta \text{Ф}=F$\\
      \underline{Алгоритм:}
      \begin{enumerate}
        \item[1.] $\Delta \text{Ф}=F \hspace{20pt} \text{Находим Ф}$
        \item[2.] $\overline{A'}=\overline{grad}\text{Ф}$ 
      \end{enumerate}
      \item[(2):] $\overline{rot}\overline{A''}=\overline{B}$\\
      $\overline{B}$ известно. Тогда из \hyperref[punkt3]{пункта 3} можно найти частное решение
      $\overline{A_0''}$. Общее решение ищем в виде:
      \[\overline{A''}=\overline{A_0''}+\overline{C},\text{где } 
      \overline{C}=\overline{grad\tilde{\text{Ф}}}(\text{нужно, чтобы }
      \overline{rot}\overline{C}=0)\]
      \underline{Алгоритм:}
      \begin{enumerate}
        \item[1)] Решаем уравнение $\overline{rot}\overline{A''}=B$
        \item[2)] Решаем $\Delta \tilde{\text{Ф}}=-div\overline{A_0''}$. Находим $\tilde{\text{Ф}}$
        \item[3)] Находим $\overline{C}=grad \tilde{\text{Ф}}$
        \item[4)] $\overline{A''}=\overline{A_0''}+\overline{C}$
      \end{enumerate}
    \end{enumerate}
  \end{enumerate}
  %КОНЕЦ 3 СЕМЕСТРА ЛЮБИМОГО МАТ АНАЛИЗА ААААААААААААААААААААААА
  \section{Числовые и функциональные ряды. Ряды Фурье}
  \subsection{Выраженные виды}
  \[  U_1+U_2+\dots+U_N+\dots=\sum_{n=1}^{\infty} U_n\]
  Называется бесконечной суммой или числовым рядом.\\
  $\{ U_n\}$ - числа члены ряда, в общем случае $U_n \in \mathcal{C}$\\
  $U_n$ - общий член ряда\\
  \underline{Определение: } Сумму $\begin{matrix}
    S_1=U_1\\
    S_2=U_1+U_2\\
    S_3=U_+U_2+U_3\\
    \dots\\
    S_n=U_1+\dots+U_n
  \end{matrix}$
  называется частичными суммами ряда $\sum_{n=1}^{\infty} U_n$\\
  
  \underline{Определение: } Если $\exists$ конечный предел $\lim_{n \to \infty} S_n =S$, то он называется
  суммой ряда. В этом случае ряд называется сходящимся. В противном случае ряд расходится.\\

  \underline{Замечание:} Пусть известна $\{S_n\}$
  \begin{center}
    $u_1=S-1$\\
    $u_2=S_2-S_1$
    $\vdots$\\
    $u_n=S_n-S_{n-1}$
  \end{center}
  Каждая из последовательностей $\{u_n\},\{S_n\}$ однозначно определяет другую.\\

  \underline{Замечание:} Нумерация ряда может начинаться с любого натурального числа, а также с нуля ($\varnothing$).

  \underline{Пример:} Бесконечно убывающая геометрическая прогрессия (Б. у. геом. прогрессия)
  \begin{center}
    $1+q+q^2+\dots+q^n+\dots=\sum^{\infty}_{n=0} q^n$\\
    $S_n=1+\dots+q^n$\\
    $S_n q=q+\dots+q^{n+1}$\\
    $S_n(1-q)=1-q^{n+1}\Rightarrow S_n = \frac{1-q^{n+1}}{1-q}$
  \end{center}

  \[\lim_{n \to \infty} S_n =\Bigg\{ 
    \begin{matrix}
      \frac{1}{1-q}, & |q|<1\\
      \infty, & |q|>1\\
      \infty, & |q|=1\\
      \not \exists, & |q|=-1
    \end{matrix}\]



  \subsection{Свойства сходящихся рядов}
  \subsubsection*{Теорема 10.2.1(Необходимое условие сходимости рядов)}\label{th:10.2.1}
  \par\noindent
  Если ряд $\sum_{n=1}^{\infty} U_n$ сходится, то $\lim_{n \to \infty} U_n =0$\\
  \underline{Доказательство:}
  \begin{adjustwidth}{1.5em}{1.5em}
    Пусть ряд $\sum_{n=1}^{\infty} U_n$ сходится $\Rightarrow \exists$ конечный $\lim_{n \to \infty} S_n =S$.\\
     $U_n=S_n-S_{n-1}$
    \[\lim_{n \to \infty} U_n = \lim_{n \to \infty} (S_n - S_{n-1}) = \lim_{n \to \infty} S_n - \lim_{n \to \infty} S_{n-1} = S-S = 0\]
  \end{adjustwidth}
  \begin{center}
    \textbf{Ч.т.д.}
  \end{center}
  \subsubsection*{Теорема 10.2.2}\label{th:10.2.2}
  \par\noindent
  Если ряды $\sum_{n=1}^{\infty} U_n' $ и $\sum_{n=1}^{\infty} U_n''$ сходятся, причём их суммы собственно равны S' и S'',
  то $\forall \lambda',\lambda'' \in R$. Ряд $\sum^{\infty}_{n=1}(\lambda' u'_0+\lambda''u''_n)$ сходится и его
  сумма $S=\lambda'S'+\lambda''S''$\\
  \underline{Доказательство:}
  \begin{adjustwidth}{1.5em}{1.5em}
    Рассмотрим частичные суммы рядов:
    \[\sum^{\infty}_{n=1} u'_n, \sum_{n=1}^{\infty}u''_n,\sum_{n=1}^{\infty}(\lambda'u'_n+\lambda''u''_n)\]
    \[\text{Рассмотрим } S_n=
    \sum_{k=1}^{n}(\lambda'u'_k+\lambda'u''_k)=
    \lambda'\sum_{k=1}^{n}u'_k+\lambda''\sum_{k=1}^{n}u''_k=\lambda'S_n'+\lambda''S''_n\]
    \[S=\lim_{n\to \infty} S_n=
    \lim_{n \to \infty}(\lambda'S'_n+\lambda''S''_n)=
    \lambda'\lim_{n\to \infty}S'_n+\lambda''\lim_{n \to \infty}S''_n=\lambda'S'+\lambda''S''\]
  \end{adjustwidth}
  \begin{center}
    \textbf{Ч.т.д.}
  \end{center}

  \underline{Определение: } Для ряда $\sum_{n=1}^{\infty}u_n$ ряд $\sum_{k=1}^{\infty}u_{n+k}$ называется n-ым остатком ряда.\\
  \subsubsection*{Теорема 10.2.3}\label{th:10.2.3.}
  \par\noindent
  Если ряд сходится, то $\forall$ его остаток сходится. Верно и обратное утверждение.\\

  \underline{Доказательство:}
  \begin{adjustwidth}{1.5em}{1.5em}
    Рассмотрим $S_i^{(n)}=u_{n+1}+u_{n+2}+\dots+u_{n+i}$ i-ая частичная сумма n-ого остатка ряда.\\
    Зафиксируем n(n-конечное).
    Пусть $m=n+1$\\
    $S_m=S_n+S_i^{(n)}$\\
    При фиксированным n и $m \to \infty(i \to \infty)$ одновременно $\exists$ или $\not \exists$ конечные
    пределы $\lim_{m\to \infty}S_m=S\lim_{i\to \infty}S_i^{(n)}=r_n$, т.е. $S=S_n+r_n$
  \end{adjustwidth}
  \begin{center}
    \textbf{Ч.т.д.}
  \end{center}
  \underline{Замечание:} Если ряд $\sum_{n=1}^{\infty}$ сходится, то $\lim_{n \to \infty}r_n=0$\\
  \underline{Доказательство:}
  \begin{adjustwidth}{1.5em}{1.5em}
    \begin{center}
      $S=S_n+r_n$\\
      $r_n=S-S_n$\\
      $\lim_{n\to \infty}r_n=\lim_{n \to \infty}(S-S_n)=\lim_{n \to \infty}S-\lim_{n\to \infty}S_n=S-S=0$
    \end{center}
  \end{adjustwidth}
  \begin{center}
    \textbf{Ч.т.д.}
  \end{center}
  
  \subsection{Критерий Коши сходимости числового ряда}
  \subsubsection*{Теорема 10.3.1}\label{th:10.3.1}
  \par\noindent
  Для того, чтобы числовой ряд $\sum_{n=1}^{\infty}$ сходился $\Leftrightarrow$
  \[\forall \varepsilon > 0 \exists n_0=n_0(\varepsilon):\forall n>n_0 \text{ и } \forall p \geq 0 |u_n+\dots+u_{n+p}|<\varepsilon\]
  \underline{Доказательство:}
  \begin{adjustwidth}{1.5em}{1.5em}
    Ряд сходится, если сходится $\{S_n\}$. Если $\{S_n\}$ фундаментальная $\Rightarrow$ она сходится.
    Покажем, что $\{S_n\}$ фундамент. Последовательность фундамент, если $\forall \varepsilon >0 \exists
    n_0=n_0(\varepsilon):\forall m,k>n_0 \Rightarrow |S_m-S_k|<\varepsilon$. Пусть $m=n+p,k=n-1:m,k>n_0=
    |S_{n+p}-S_{n-1}|<\varepsilon |u_n+\dots+u_{n+p}|<\varepsilon$
  \end{adjustwidth}
  \begin{center}
    \textbf{Ч.т.д.}
  \end{center}

  \underline{Пример:} Покажем, что гармоничный ряд является расходящимся\\
  \[1+\frac{1}{2}+\dots+\frac{1}{n}+\dots\]
  \[\text{Пусть } p=n. 
  \text{ Рассмотрим} 
  \underbracket{\frac{1}{n}+\frac{1}{n+1}+\dots+\frac{1}{2n-1}}_{n \text{ слагаемых}}>
  \frac{1}{2n}+\frac{1}{2n}+\dots+\frac{1}{2n}=\frac{1}{2}\]
  Для $\varepsilon:0<\varepsilon<\frac{1}{2}$ не $\exists n_0=n_0(\varepsilon)$: выполняется критерий коши 
  $\Rightarrow$ ряд расходится.\\
  Рассмотрим $\sum_{n=1}^{\infty}\frac{1}{n^2}$\\
  \[\underbracket{\frac{1}{n^2}+\dots+\frac{1}{(2n-1)^2}}_{n\text{ слагаемых}}
  <\frac{1}{n^2}+\dots+\frac{1}{n^2}=\frac{1}{n}\]
  \[\exists n_0=n_0(\varepsilon):\forall n>n_0 \frac{1}{n}<\varepsilon\]

  \subsection{Признак сходимости рядов с неотрицательными членами}
  \[\sum_{n=1}^{\infty} u_n, \text{ где } u_n\geq 0\]
  
  \textbf{Лемма :} Если члены ряда $u_n \geq 0,$ то он сходится $\Leftrightarrow$ когда $\{S_n\}$ ограничена сверху.

  \underline{Доказательство:}
  \begin{adjustwidth}{1.5em}{1.5em}
    Рассмотрим $\sum_{n=1}^{\infty} u_n, u_n \geq 0$\\
    Рассмотрим $S_{n+1}=S_n+u_{n+1}\geq S_n.$ Тогда $\{S_n\}$ неубывающая. Тогда она будет сходится $\Leftrightarrow$ когда она ограничена сверху.
  \end{adjustwidth}
  \begin{center}
    \textbf{Ч.т.д.}
  \end{center}

  \subsubsection*{Теорема 10.4.1(интегральный признак Коши сходимости числового ряда)}\label{th:10.4.1}
  \par\noindent
  Пусть функция $f(x)$ неотрицательная и невозрастающая $\forall x \geq 1$. 
  Тогда, для того чтобы ряд $\sum_{n=1}^{\infty} f(n)$ сходился $\Leftrightarrow$ чтобы сходился 
  $\int^\infty_1 f(x)dx$.

  \underline{Доказательство:}
  \begin{adjustwidth}{1.5em}{1.5em}
    \begin{minipage}{0.45\textwidth}
      \includegraphics[scale=0.45]{10.4.1.png}
    \end{minipage}
    \hspace{1em}
    \begin{minipage}{0.35\textwidth}
      Так как по условию $f(x)$ монотонна $\forall x \geq 1 \Rightarrow$ она была интегрируема $\forall [1;b] b \in [1;+\infty]$.
      Т.е. имеет смысл несобственный интеграл I рода.
    \end{minipage}
    \vspace{1em}
    \par
    Рассмотрим $k \leq x \leq k+1$. Так как $f(x)$ невозрастающая, то 
    $f(k)\geq f(x)\geq f(k+1)$. Проинтегрируем это неравенство на $[k;k+1]$.
    \[\int_{k}^{k+1} f(k)dx \geq \int_{k}^{k+1} f(x)dx \geq \int_{k}^{k+1} f(k+1 dx)\]
    \[f(k) \int_{k}^{k+1} dx \geq \int_{k}^{k+1} f(x)dx \geq f(k+1) \int_{k}^{k+1} dx\]
    \[f(k)\geq \int_{k}^{k+1}f(x)dx \geq f(k+1)\]
    Просуммируем по $k=\overline{1,n}.$
    \[\sum_{k=1}^{n}f(k+1)\leq \sum_{k=1}^{n}f(x)dx \leq \sum_{k=1}^{n}f(k)\]
    \[f(2)+f(3)+\dots+f(n+1)\leq \int_{1}^{n+1} f(x)dx \leq f(1)+f(2)+\dots+f(n)\]
    \[S_{n+1}-f(1)\leq\int_{1}^{n+1}f(x)dx \leq S_n\]
    
    \boxed{\Rightarrow} Пусть $\sum_{n=1}^{\infty}f(n)$ сходится. \\
    Тогда его сумма равна S. По лемме $S_n \leq S \forall n$. Тогда $\int_{1}^{n+1} f(x)dx \leq S_n \leq S.$
    Выберем $b \leq n+1$. Тогда $\int_{1}^{b} f(x)dx \leq \int_{1}^{n+1}f(x)dx \leq S_n \leq S$.
    Т.е. множество интегралов $\int_{1}^{b} f(x)dx, b\geq 1$ ограничены сверху $\Rightarrow \int_{1}^{\infty}f(x)dx$ сходится.\\
    
    \boxed{\Leftarrow} Пусть $\int_{1}^{\infty} f(x)dx $ сходится.\\
    Тогда $S_{n+1} \leq f(1) + \int_{1}^{n+1}f(x)dx \leq f(1)+\int_{1}^{\infty}f(x)dx$\\
    Т.е. $\{S_{n+1}\}$ ограничена сверху $\Rightarrow \exists \lim_{n \to \infty} S_{n+1}=S.$ А это значит $\sum_{n=1}^{\infty} f(x)$ сходится.
  \end{adjustwidth}
  \begin{center}
    \textbf{Ч.т.д.}
  \end{center}

  \underline{Пример:} Рассмотрим $\sum_{n=1}^{\infty} \frac{1}{n^p}$
  \[\text{Рассмотрим } \int_{1}^{\infty} \frac{dx}{x^p} = \Big| p \not = 1 \Big|=
  \lim_{A \to \infty}\int_{1}^{A} \frac{dx}{x^p} 
  = \lim_{A \to \infty} \Bigg[ \frac{x^{-p+1}}{-p+1}\Bigg|^A_1 \Bigg] =
  \lim_{A \to \infty} \Bigg[ \frac{A^{1-p}}{1-p} - \frac{1}{1-p} \Bigg] =
  \Bigg\{ \begin{matrix}
    \frac{1}{p-1}, p>1\\
    \infty, p<1
  \end{matrix}\]
  \[\text{Рассмотрим } \int_{1}^{\infty} \frac{dx}{x} = 
  \lim_{A \to \infty} \int_{1}^{A} \frac{dx}{x} = 
  \lim_{A \to \infty} \Big[ ln|x| \Big|^A_1 \Big] =
  \lim_{A \to \infty} \Big[ ln A - ln1 \Big] = \infty\] 
  \[\text{Вывод: } \sum_{n=1}^{\infty} \frac{1}{n^p} \hspace{10pt} \setlength{\arraycolsep}{1pt}\begin{matrix}
    & \text{ сходится } p>1\\
    \nearrow  \\
    \searrow  \\
    & \text{ расходится } p\leq 1
  \end{matrix}\]

  \subsubsection*{Теорема 10.4.2(I признак сравнения)}\label{th:10.4.2}
  \par\noindent
  Пусть $0\leq u_n \leq U_n$. Тогда 
  \begin{enumerate}
    \item Если $\sum_{n=1}^{\infty} U_n$ сходится $\Rightarrow \sum_{n=1}^{\infty} u_n$ сходится.
    \item Если $\sum_{n=1}^{\infty} u_n$ расходится $\Rightarrow \sum_{n=1}^{\infty} U_n$ расходится.
  \end{enumerate}

  \underline{Доказательство:}
  \begin{adjustwidth}{1.5em}{1.5em}
    \begin{enumerate}
      \item Пусть $\sum_{n=1}^{\infty} U_n$ сходится. Тогда $\sigma = \sum_{n=1}^{\infty}U_n$\\
      Рассмотрим $\sigma_n = \sum_{k=1}^{\infty} U_k$. Так как $U_n \geq 0 \Rightarrow \sigma_n \leq \sigma$\\
      Рассмотрим $S_n = \sum_{k=1}^{\infty} u_k \leq \sum_{k=1}^{\infty} U_k = \sigma_n \leq \sigma$\\
      $\{S_n\}$ невозрастающая и ограничена сверху $\Rightarrow$ по лемме $\sum_{n=1}^{\infty}u_n$ сходится.
      \item Пусть $\sum_{n=1}^{\infty} u_n$ расходится. Тогда $\sum_{n=1}^{\infty} U_n$ расходится, потому что если бы он 
      сходился, тогда по (1) $\sum_{n=1}^{\infty} u_n$ был бы сходящимся, а  это не так.
    \end{enumerate}
  \end{adjustwidth}
  \begin{center}
    \textbf{Ч.т.д.}
  \end{center}

  \subsubsection*{Теорема 10.4.3(Следствие признака сравнения. II признак сравнения)}\label{th:10.4.3}
  \par\noindent
  Пусть $u_n \geq 0, U_n >0 \; \forall n$ и $\lim_{n \to \infty} \frac{u_n}{U_n} = l$.
  Тогда:
  \begin{enumerate}
    \item Если $\sum_{n=1}^{\infty} U_n$ сходится и $0\leq l \leq \infty,$ то $\sum_{n=1}^{\infty} u_n$ сходится\\
    \item Если $\sum_{n=1}^{\infty} U_n$ расходится и $0\leq l \leq \infty$, то $\sum_{n=1}^{\infty}u_n$ расходится
  \end{enumerate}

  \underline{Доказательство:}
  \begin{adjustwidth}{1.5em}{1.5em}
    \begin{enumerate}
      \item Пусть $0\leq l \leq \infty $ и $\lim_{n \to \infty} \frac{u_n}{U_n}=l. $\[\exists N=N(\varepsilon): \forall n>N
      \Big| \frac{u_n}{U_n}-l \Big| < \varepsilon, \frac{u_n}{U_n} \in U_\varepsilon (l)\]
      А это значит $\forall n > N \frac{u_n}{U_n}<l+1$.\\
      Рассмотрим $\sum_{A=N}^{\infty}U_n$ сходится $\Rightarrow \sum_{n=N}^{\infty}(l+1)U_n$ сходится $\Rightarrow \sum_{k=1}^{\infty} u_{N+k}$
      сходится по I признаку сравнения.\\
      Т.е. сходится исходный ряд $\sum_{n=1}^{\infty} u_n$, так как сходится его остаток.
      
      \item Пусть $0<l \leq \infty$
      \[\exists l: 0 < l' < l. \exists n_0=n_0(\varepsilon): \forall n>n_0 \frac{u_n}{U_n} \in \varepsilon \text{ окрестности } (.) l\]
      А это значит $\forall n>n_0 \frac{u_n}{U_n}> l' \Rightarrow u_n > l'U_n$.\\
      Так как $\sum_{n=1}^{\infty}U_n$ расходится $\Rightarrow \sum_{n=1}^{\infty} l'U_n$ расходится 
      $\Rightarrow \sum_{k=1}^{\infty} u_{n_0+k}$ расходится по I признаку сравнения.
    \end{enumerate}
    Тогда исходный $\sum_{n=1}^{\infty}u_n$ расходится, так как расходится его $n_0$-ый остаток
  \end{adjustwidth}
  
  \begin{center}
    \textbf{Ч.т.д.}
  \end{center}

  \underline{Примеры:}
  \begin{enumerate}
    \item $\sum_{n=1}^{\infty} \frac{sin^2 \frac{n^2 \alpha^3}{\sqrt[3]{e^{n^2+ln n^2}}}}{n^2}$.
    \\ Рассмотрим $\frac{sin^2 \frac{n^2 \alpha^3}{\sqrt[3]{e^{n^2+ln n^2}}}}{n^2} \leq \frac{1}{n^2}$\\
    Рассмотрим $\sum_{n=1}^{\infty} \frac{1}{n^2}$ сходится (так как $m=2$) $\Rightarrow$ исходный ряд сходится по I признаку сравнения

    \item $\sum_{n=1}^{\infty} \frac{1}{\sqrt{n}+1}$\\
    Рассмотрим $\sum_{n=1}^{\infty}\frac{1}{\sqrt{n}}$-расходится($m=\frac{1}{2}$)\\
    Рассмотрим $\lim_{n \to \infty} \frac{\frac{1}{\sqrt{n}+1}}{\frac{1}{\sqrt{n}}}=1\not=0 \Rightarrow$ исходный ряд расходится по II признаку сравнения.
  \end{enumerate}

  \subsubsection*{Теорема 10.4.4(Признак Даламбера)}\label{th:10.4.4}
  \par\noindent
  Если для ряда $\sum_{n=1}^{\infty} u_n \; u_n>0 \exists \lim_{n\to \infty} \frac{u_{n+1}}{u_n}=l$\\
  Если $l<1 \Rightarrow \sum_{n=1}^{\infty} u_n$ сх\\
  Если $l>1 \Rightarrow \sum_{n=1}^{\infty} u_n$ рсх\\
  Если $l=1 \Rightarrow$ признак Даламбера не отвечает на вопрос о сходимости или расходимости исходного ряда

  \underline{Доказательство:}
  \begin{adjustwidth}{1.5em}{1.5em}
    \begin{minipage}{0.2\textwidth}
      $1) l<1$
    \end{minipage}
    \hspace{1em}
    \begin{minipage}{0.55\textwidth}
      \includegraphics[scale=0.3]{10.4.2.png}
    \end{minipage}
    \vspace{1em}
    \par

    Пусть $l<q<1$. Тогда $\exists n_0=n_0(\varepsilon):\forall n>n_0 \frac{u_{n+1}}{u_n}\in \varepsilon$ окрестности (.) 
    $l \Rightarrow \forall n>n_0 \frac{u_{n+1}}{u_n}<q \Rightarrow u_{n+1} < q*u_n$\\
    \[u_{n_0+k}<qu_{n_0+k-1}<q^2 u_{n_0+k-2} < \dots < q^{k-1}u_{n_0+1}\]

    Рассмотрим $\sum_{n=1}^{\infty} q^{n-1} u_{n_0+1}=u_{n_0+1}\sum_{n=1}^{\infty}q^{n-1},$ ряд
    $\sum_{n=1}^{\infty}q^{n-1}$ сх как бесконечно убывающая геометрическая прогрессия.\\
    Тогда ряд $u_{n_0+1}\sum_{n=1}^{\infty}q^{n-1}$ сх.\\
    Тогда ряд $\sum_{n=1}^{\infty} u_{n_0+k}$- сх по первому признаку сравнения.\\
    Тогда $\sum_{n=1}^{\infty} u_n$ сх, так как сходятся его $n_0$-ый остаток
    \begin{center}
      \textbf{Ч.т.д.}
    \end{center}
    
    \begin{minipage}{0.1\textwidth}
      2)
    \end{minipage}
    \hspace{1em}
    \begin{minipage}{0.55\textwidth}
      \includegraphics[scale=0.6]{10.4.3.png}      
    \end{minipage}
    \vspace{1em}
    \par

    $\exists n_0=n_0(\varepsilon): \forall n>n_0 \frac{u_{n+1}}{u_n} \in \varepsilon$ - окрестности (.) $l$.\\
    Тогда $\forall n > n_0 \frac{u_{n+1}}{u_n}>1$\\
    \[u_{n+1}=u_n \text{ для } n>n_0\]
    То есть: 
    \[\begin{matrix}
      u_{n_0+2}>u_{n_0+1}\\
    u_{n_0+3}>u_{n_0+2}>u_{n_0+1}\\
    u_{n_0+k}>\dots>u_{n_0+1}\\
    \end{matrix}\]

    Последовательность членов ряда не стремится к нулю.\\
    Необходимое условие не выполняется, ряд расходится.\\
    \begin{center}
      \textbf{Ч.т.д.}
    \end{center}
  \end{adjustwidth}

  \subsubsection*{Теорема 4.5 (Радикальный признак Коши)}\label{th:10.4.5}
  \par\noindent
  Пусть для ряда $\sum_{n=1}^{\infty} u_n \; u_n>0$\\
  $\exists \lim_{n \to \infty} \sqrt[n]{u_n}=l$\\
  Тогда:
  \begin{enumerate}
    \item Если $l<1 \Rightarrow \sum_{n=1}^{\infty} u_n$ сх\\
    \item Если $l>1 \Rightarrow \sum_{n=1}^{\infty} u_n$ рсх\\
    \item Если $l=1$, то радикальный признак Коши не отвечает вопрос о сходимости/расходимости ряда. 
  \end{enumerate}
  
  \pagebreak

  \underline{Доказательство:}
  \begin{adjustwidth}{1.5em}{1.5em}
    \begin{enumerate}
    \item $\sqsupset l<1$ \includegraphics[scale=0.6]{10.4.4.png}. $\exists q:l<q<1$\\

    \[ \exists n_0=n_0(\varepsilon):\forall n>n_0: \sqrt[n]{u_n} \in \varepsilon \text{- окрестности (.)} \]    
    Тогда $\forall n>n_0 \sqrt[3]{u_n}<9.$ Тогда\\
    $
      u_n>n_0 \hspace{20pt} u_n<q\\
      u_{n_0+1}<q^{\overline{n_0+1}}\\
      u_{n_0+2}<q^{n_0+2}\\
      u_{n_0+k}<q^{n_0+k}
    $\\
    Рассмотрим $\sum_{k=1}^{\infty} q^{n_0+k}$ сходится как бесконечно убывающая геометрическая прогрессия. Тогда
    $\sum_{k=1}^{\infty} u_{n_0+k}$ сходится по I признаку сравнения. Тогда исходный ряд сходится, так как сходится его 
    $n_0$ остаток

    \item $\sqsupset l>1$ \includegraphics[scale=0.2]{10.4.5.png}\\
    $\exists n_0=n_0(\varepsilon):\forall n>n_0 
    \begin{matrix}
      \sqrt[n]{u_n}\\
      \sqrt{u_n}
    \end{matrix}
    \hspace{20pt}
    \begin{matrix}
      u_{n_0+1}>1\\
      u_{n_0+2}>1\\
      \vdots\\
      u_{n_0+k}\\
    \end{matrix}
    $\\
    Последовательность членов ряда не стремится к нулю. Ряд $\sum_{k=1}^{\infty} u_{n_0+k} \Rightarrow
    \sum_{n=1}^{\infty} u_n$ расходится, так как расходится его $n_0$-ый остаток.
    \end{enumerate}
  \end{adjustwidth}
  \begin{center}
    \textbf{Ч.т.д.}
  \end{center}

  \underline{Замечание:} 
  \begin{enumerate}
    \item Если общий член ряда содержит факториал, рекомендуется использовать признак Даламбера:\\
    Рассмотрим $\sum_{n=1}^{\infty} \dfrac{1}{n!}$
    \[u_n = \frac{1}{n!} \hspace{20pt} u_{n+1}= \frac{1}{(n+1)!} \hspace{20pt} 
    \lim_{n \to \infty} = \frac{u_{n+1}}{u_n}=\lim_{n \to \infty} \frac{1}{n+1} =0 <1\]
    \begin{center}
      \text{Исходный ряд сходится по Даламберу}
    \end{center} 

    \item Если общий член ряда возводится в $n$-ую степень, то рекомендуется использовать 
    радикальный признак.\\
    Рассмотрим $\sum_{n=1}^{\infty} \frac{1}{n^n}$
    \[\lim_{n \to \infty} \sqrt[n]{u_n} = \lim_{n \to infty} \frac{1}{n}=0<1 \]
    \begin{center}
      Исходный ряд сходится по радикальному признаку Коши.
    \end{center}

    \pagebreak

    \item Если общий член ряда-дробно-рациональная функция, то ни признак Даламбера, ни 
    рациональный признак Коши не даст ответ на вопрос о сходимости(расходимости) ряда.\\
    $\sum_{n=1}^{\infty} \frac{n+1}{n(n+3)}$
    \[\lim_{n \to \infty} \frac{u_{n+1}}{u_n}=\lim_{n \to \infty}\frac{(n+2)n(n+3)}{(n+1)(n+4)(n+1)}=1\]
    
    \item Если один из признаков (Даламбер/Коши) не дал ответа на вопрос о 
    сходимости (расходимости) ряда, не рекомендуется использовать другой признак (Коши/Даламбер).
  \end{enumerate}

  \subsection{Знакочередующиеся ряды}
  \underline{Определение: } Ряд вида: $u_1-u_2+u_3-u_4+\dots+(-1)^{n+1} u_n =
  \sum_{n=1}^{\infty} (-1)^{n+1} u_n $ называется знакочередующимся рядом.\\

  \subsubsection*{Теорема 10.5.1(признак Лейбница)}\label{th:10.5.1}
  \par\noindent
  \begin{enumerate}
    \item Если для знакочередующегося ряда $\sum_{n=1}^{\infty} (-1)^{n+1} u_n \hspace{10pt} u_n \geq 0$
    последовательность $\{u_n\}$ убывающая или не возрастающая и $\lim_{n \to \infty} u_n =0$, то ряд сходится.

    \item Если $S_n = \sum_{k=1}^{n} (-1)^{k+1}u_k, S = \sum_{n=1}^{\infty} u_n$, то $\forall n \in N$ 
    выполняется:
    \[|S_n-S|\leq u_{n+1}\]
  \end{enumerate}

  \underline{Доказательство:}
  \begin{adjustwidth}{1.5em}{1.5em}
    \begin{enumerate}
      \item
      $\begin{matrix}
        S_1=u_1 && S_2=u_1-u_2\\
        S_3=u_1-u_2+u_3 && S_4 = u_1-u_2+u_3-u_4 \\
        \dots && S_{2k} = (u_1-u_2)+(u_3-u_4)+\dots + (u_{2k-1}-u_{2k})
      \end{matrix}$\\

      \begin{center}
              Так как по условию $\{u_n\}$ убывающая или невозрастающая,
      \end{center}
      \[u_1-u_2 > 0 (\geq 0)\dots  (u_{2k-1}-u_{2k}) > 0 (\geq 0)\] 
      \begin{center}
        Тогда
      \end{center}
      \[S_{2k+2}=S_{2k}+(u_{2k+1}-u_{2k+2})\geq S_{2k}\]
      \begin{center}
        То есть $\{S_{2k}\}$ неубывающая(возрастающая).
      \end{center}
      \begin{center}
        С другой стороны
      \end{center} 
      \[s_{2k}=u_1-(u_2-u_3)-(u_4-u_5)-\dots-(u_{2k-2}-u_{2k-1})-u_{2k}\leq u_1\]
      Получим $\{S_{2k}\}$ неубывающая(возрастающая) и ограничена сверху. \\
      То есть 
      $\exists \lim_{k \to \infty}S_{2k}=S$.\\



      Рассмотрим последовательность с нечётными номерами
      \[S_{2k+1}=S_{2k}+u_{2j+1}\]
      \[\lim_{k \to \infty} S_{2k+1}=\lim_{k \to \infty} S_{2k}+\lim_{k \to \infty} u_{2k+1} = S\]\\
      То есть $\forall n \lim_{n \to \infty} S_n =S$


      \item Докажем, что $|S_n-S|\leq u_{n+1}$.\\
      $\{S_{2n}\}$ неубывающая(возрастающая) $S_{2n}\leq S$\\
      Рассмотрим $\{S_{2n+1}\}$ 
      \[S_{2n+1}=S_{2n-1}-u_{2n}+u_{2n+1}=S_{2n-1}-(u_{2n}-u_{2n+1})\leq S_{2n-1}\]
      То есть $\{S_{2n+1}\}$ невозрастающая (убывающая)
      \[S_{2n+1}\geq S\]
      Получим что $S_{2n}\leq S_{2n+2}\leq S\leq S_{2n+1}(-S_{2n})$
      \[a) 0\leq S-S_{2n}\leq S_{2n-1}-S{2n}\]
      \[0\leq S-S_{2n}\leq u_{2n+1}\]
      \[b) S_{2n+2}-S_{2n+1}\leq S-S_{2n+1}\leq | \cdot (-1)\]
      \[0\leq S_{2n+1}-S \leq S_{2n+1} - S_{2n+1}\]
      \[0\leq S_{2n+1}-S \leq u_{2n+2}\]
      $\forall n \hspace{10pt} 0\leq |S-S_n|\leq u_{n+1}$ 
    \end{enumerate}
  \end{adjustwidth}
  \begin{center}
    \textbf{Ч.т.д.}
  \end{center}

  \subsection{Абсолютно сходящиеся ряды}
  \underline{Замечание:} В этом параграфе ряд $\sum_{n=1}^{\infty} u_n$ - знакочередующийся

  \underline{Определение: } Ряд $\sum_{n=1}^{\infty} u_n$ сходится абсолютно, если сходится ряд $\sum_{n=1}^{\infty} |u_n|$

  \subsubsection*{Теорема 10.6.1(Критерий Коши абсолютной сходимости знакочередующегося ряда)}\label{th:10.6.1}
  \par\noindent
  Для того, чтобы ряд $\sum_{n=1}^{\infty} u_n$ абсолютно сходился необходимо и достаточно чтобы 
  \[ \forall \varepsilon > 0 \exists n_0 = n_0(\varepsilon): \forall n > n_0 \forall p \geq 0 \Rightarrow \sum_{k=0}^{p} |u_{n+k}| < \varepsilon\]

  \underline{Доказательство:}
  \begin{adjustwidth}{1.5em}{1.5em}
    Следует из \hyperref[th:10.3.1]{критерия Коши сходимости числового ряда} и определения абсолютной 
    сходимости ряда.
  \end{adjustwidth}
  
  \subsubsection*{Теорема 10.6.2}\label{th:10.6.2}
  \par\noindent
  Если ряд абсолютно сходится, то он сходится.

  \underline{Доказательство:}
  \begin{adjustwidth}{1.5em}{1.5em}
    Так как ряд сходится абсолютно, то согласно \hyperref[th:10.6.1]{Теореме 10.6.1}
    \[\forall \varepsilon > 0 \exists n_0 = n_0(\varepsilon): \forall n>n_0 \text{ и } \forall p \geq 0
    \sum_{k=0}^{p} |u_{n+k}| < \varepsilon\]
    \[\sum_{k=0}^{p} u_{n+k} \leq \sum_{k=0}^{p} |u_{n+k}| < \varepsilon\]
    Тогда ряд сходится по \hyperref[th:10.3.1]{Теореме 10.3.1}
  \end{adjustwidth}

  \subsubsection*{Теорема 10.6.3}\label{th:10.6.3}
  \par\noindent
  Если ряд $\sum_{n=1}^{\infty} u_n$ абсолютно сходится, то ряд $\sum_{m=1}^{\infty} \overset{*}{u_m}$,
  составленный из тех же членов, что и исходный ряд, но взятых в другом порядке, также абсолютно сходится и 
  имеет ту же сумму.
  \[\sum_{n=1}^{\infty} u_n = \sum_{m=1}^{\infty} \overset{*}{u_m}\]

  \subsubsection*{Теорема 10.6.4}\label{th:10.6.4}
  \par\noindent
  Если ряд $\sum_{n=1}^{\infty} u_n$ и $\sum_{n=1}^{\infty} V_n$ абсолютно сходятся, то ряд, составленный
  из всевозможных произведений $u_n V_n$ членов этого ряда, также абсолютно сходятся.
  Причем, если $S' = \sum_{n=1}^{\infty}u_n, S''=\sum_{n=1}^{\infty} V_n$, то сумма ряда, составленного
  из всевозможных произведений $u_n V_n$, будет равна $S=S'\cdot S''$.

  \begin{enumerate}
    \item[(1)] $\underset{\text{сх по \hyperref[th:10.6.1]{Th. 10.6.1}}}{\sum_{n=1}^{\infty}} (-1)^n \cdot \frac{1}{n^2}$
    Рассмотрим $\sum_{n=1}^{\infty} \frac{1}{n^2}$ - (2) сходится (m=2)\\
    Тогда  (1) сходится абсолютно $\Rightarrow$ сходится по \hyperref[th:10.6.2]{Th. 10.6.2}
  \end{enumerate}

  \subsection{Условно сходящиеся ряды}
  \underline{Определение: } Сходящийся, но не абсолютно ряд, называется условно сходящимся рядом.

  \underline{Пример:}\\
  \begin{minipage}{0.45\textwidth}
    \[\sum_{n=1}^{\infty} (-1)^n \frac{1}{n}\]
  \end{minipage}
  \hspace{1em}
  \begin{minipage}{0.55\textwidth}
    \begin{enumerate}
      \item $u_n>u_{n+1}$\\
      \text{ } \hspace{60pt} Ряд сходится по Th. Лейбница\\
      $\frac{1}{n}>\frac{1}{n+1}$
      \item $\lim_{n \to \infty} u_n =0$ \hspace{40pt} $\measuredangle$ $\sum_{n=1}^{\infty} \frac{1}{n}$рсх(гармон.)\\
      $\lim_{n \to \infty} \frac{1}{n}=0$ \hspace{40pt} Тогда исх ряд сх-ся условно
    \end{enumerate}
    
  \end{minipage}
  \vspace{1em}
  \par
  
  Рассмотрим $\sum_{n=1}^{\infty} u_n$ - знакочередующийся. \\
  Обозначим $
  \begin{matrix}
    \overset{+}{u_n} \text{-  неотрицательные слагаемые исходного ряда}\\
    \overset{-}{u_n} \text{- отрицательные слагаемые}
  \end{matrix}$

  \underline{Замечание:} $\overset{+}{u_n},\overset{-}{u_n}$ взяты в том же порядке, в котором они расположены
  в исходном ряде.

  Рассмотрим ряды $\sum_{n=1}^{\infty} \overset{+}{u_0}(1) \hspace{10pt} \sum_{n=1}^{\infty}\overset{-}{u_n}(2)$
  
  \textbf{Лемма :} Если ряд $\sum_{n=1}^{\infty} u_n$ условно сходится, то (1) и (2) расходятся.

  \underline{Доказательство:}
  \begin{adjustwidth}{1.5em}{1.5em}
    $\sqsupset S_n = \sum_{k=1}^{n} |u_k| -$ n-ая частичная сумма ряда из модулей\\
    $\overset{+}{S_n} = \sum_{k=1}^{n} \overset{+}{u_n}$ - n-ая частичная сумма ряда (1)\\
    $\overset{-}{S_n} = \sum_{k=1}^{n} \overset{-}{u_n}$ - n-ая частичная сумма ряда (2)\\
    Суммы $S_n',\overset{+}{S_n}, \overset{-}{S_n}$ неотрицательны\\
    Рассмотрим \Bigg\{
    $
    \begin{matrix}
      S_n = \overset{+}{S_m} - \overset{-}{S_k}\\
      S_n'=\overset{+}{S_m} + \overset{-}{S_k}\\
      \overset{+}{S_m}, \overset{-}{S_k}
    \end{matrix}
    \hspace{20pt}  n=m+k$\\
    Решение: 
    \[\overset{+}{S_m} = \frac{S_n+S_n'}{2}, \overset{-}{S_k} = \frac{S_m'-S_m}{2}\]
    \[\lim_{m \to \infty} \overset{+}{S_m} = \infty \Rightarrow \sum_{n=1}^{\infty} \overset{+}{u_n} \text{ расходится}\]
    \[\lim_{k \to \infty} \overset{-}{S_k} = \infty \Rightarrow  \sum_{n=1}^{\infty} \overset{-}{u_n} \text{расходится}\]
  \end{adjustwidth}
  \begin{center}
    \textbf{Ч.т.д.}
  \end{center}

  \subsubsection*{Теорема 10.7.1(Теорема Римана)}\label{th:10.7.1}
  \par\noindent
  Если ряд $\sum_{n=1}^{\infty} (-1)^n u_n$ условно сходится, то каково бы не было действительного числа
  S можно так представить члены этого ряда, что сумма получившегося ряда будет равна S.

  \underline{Доказательство:}
    \begin{adjustwidth}{1.5em}{1.5em}
      \begin{tikzpicture}[scale=1.2]
        \draw[->] (-4.5,0) -- (4.5,0);
    
        \filldraw (-2,0) circle (1pt) node[above] {$\overset{+}{u_1} + \dots + \overset{+}{u_{n_1}} - \overset{-}{u_1} - \dots - \overset{-}{u_{n_2}}$};
        \filldraw (0,0) circle (1pt) node[below=2pt] {$S$};
        \filldraw (1.5,0) circle (1pt) node[below] {$\overset{+}{u_1} + \dots + \overset{+}{u_{n_1}}$};
    \end{tikzpicture}

    Рассмотрим ряды
    $
    \begin{matrix}
      \sum_{n=1}^{\infty} \overset{+}{u_n}(1)\\
      \sum_{n=1}^{\infty} \overset{-}{u_n}(2)
    \end{matrix}
    $(1),(2) - расходятся\\
    Выберем из ряда (1) подряд число слагаемых так, чтобы их сумма была $> S$, а сумма меньшего числа
    этих слагаемых была не больше $S$. Обозначим через $n_1$ наименьшее натуральное число при котором
    выполняется условие:
    \[\overset{+}{u_1}+\overset{+}{u_2}+\dots+\overset{+}{u_{n_1}} > S\]
    \[\overset{+}{u_1}+\overset{+}{u_2}+\dots+\overset{+}{u_{n_1-1}} \leq S\]

    Выберем из ряда (2) подряд столько слагаемых, чтобы вычтя их сумму из суммы уже набранных из ряда
    (1) слагаемых, получить значение $<S$ и чтобы меньшее число указанных слагаемых не обладало
    этим свойством. Обозначим наименьшее натуральное число, обладающее этим свойством, $n_2$.

    \[\overset{+}{u_1}+\overset{+}{u_2}+\dots+\overset{+}{u_{n_1}}-\overset{-}{u_1}-\overset{-}{u_2}-\dots-\overset{-}{u_{n_2}}<S\]

    \[\overset{+}{u_1}+\overset{+}{u_2}+\dots+\overset{+}{u_{n_1}}-\overset{-}{u_1}-\overset{-}{u_2}-\dots-\overset{-}{u_{n_2-1}} \geq S\]

    Обозначим через $n_3$ наименьшее натуральное число m котором обладает свойством:
    \[\overset{+}{u_1}+\dots\overset{+}{u_{n_1}}-\overset{-}{u_1}-\dots -\overset{-}{u_{n_2}}+\overset{+}{u_{n_1+1}}+\dots+\overset{+}{u_{n_3}} > S\]

    \[\overset{+}{u_1}+\dots\overset{+}{u_{n_1}}-\overset{-}{u_1}-\dots -\overset{-}{u_{n_2}}+\overset{+}{u_{n_1+1}}+\dots+\overset{+}{u_{n_3-1}} \leq S\]

    Обозначим через $n_4$ наименьшее натуральное число, обладающее свойством:

    \[\overset{+}{u_1}+\dots+\overset{+}{u_{n_1}}-\overset{-}{u_1}-\dots-\overset{-}{u_{n_2}}+\overset{+}{u_{n_1+1}}+\dots+\overset{+}{u_{n_3}}-\overset{-}{u_{n_2+1}}-\dots-\overset{-}{u_{n_4}} < S\]

    \[\overset{+}{u_1}+\dots+\overset{+}{u_{n_1}}-\overset{-}{u_1}-\dots-\overset{-}{u_{n_2}}+\overset{+}{u_{n_1+1}}+\dots+\overset{+}{u_{n_3}}-\overset{-}{u_{n_2+1}}-\dots-\overset{-}{u_{n_4-1}} \geq S\]
    

    Продолжая этот процесс, получим ряд:

    \[\overset{+}{u_1}+\dots+\overset{+}{u_{n_1}}-\overset{-}{u_1}-\dots-\overset{-}{u_{n_2}}+\overset{+}{u_{n_1+1}}+\dots+\overset{+}{u_{n_3}}-\overset{-}{u_{n_2+1}}-\dots-\overset{-}{u_{n_4}}+\dots\]
  
    Обозначим $S_n$ - частичные суммы этого ряда

    \[\begin{matrix}
      S_{n_1}>S & S_{n_1-1}\leq S\\
      S_{n_1+n_2}<S & S_{n_1+n_2-1}\geq S\\
      S_{n_1+n_2+n_3} > S & S_{n_1+n_2+n_3-1} \leq S\\
      S_{n_1+n_2+n_3+n_4} < S & S_{n_1+n_2+n_3+n_4-1} \geq S
    \end{matrix}\]
    Получим:
    \[\begin{matrix}
      S_{n_1+n_2+\dots+n_{2k-1}} > S & & S_{n_1+n_2+\dots+n_{2k-1}-1} \leq S\\
      S_{n_1+n_2+\dots+n_{2k}} < S & &  S_{n_1+n_2+\dots+n_{2k-1}} \geq S\\
      S_{n_1+n_2+n_{2k-1}-1} \leq & S & < s_{n_1+n_2+\dots+n_{2k-1}}(3)\\
      S_{n_1+n_2+n_{2k}} < & S & \leq s_{n_1+n_2+\dots+n_{2k-1}-1}(4)\\
    \end{matrix}\]
    Рассмотрим (3). Вычтем из каждой части (3) $S_{n_1+\dots+n_{2k-1}-1}$\\
    Получим $0\leq S - S_{n_1+\dots+n_{2k-1}-1} < S_{n_1+\dots+n_{2k-1}} - S_{n_1+\dots+n_{2k-1}-1}$
    \[\lim_{k \to \infty} S_{n_1+\dots+n_{2k-1}-1} = S\]
    Аналогично, рассмотрев (4) можно показать, что:
    \[\forall k \lim_{k \to \infty} S_k =S \text{ (удовлетворяет условиям алгоритма)}\]
    То есть $S$ - сумма ряда из перестановок
  \end{adjustwidth}

  \begin{center}
    \textbf{Ч.т.д.}
  \end{center}

  \underline{Замечание:} Теорема Римана показывает, что одно из основных свойств конечных сумм
  (независимость суммы от порядка суммирования, коммунитативность сложения) не переносится на бесконечные суммы:
  если ряд сходится условно, то его сумма зависит от порядка слагаемых.

  \subsection{Признаки сходимости рядов Дирихле и Абеля}
  Рассмотрим конечную сумму
  \[\sum_{i=1}^{\infty} a_i b_i \boxed{=} \]
  Обозначим $
  \begin{matrix}
    B_i=b_1+\dots+b_i, i=1,\dots,a\\
    b_1=B_1 \hspace{10pt} b_2=B_2 \hspace{10pt} b_i=B_i \hspace{10pt}
  \end{matrix}\\
  \boxed{=} a_1b_1+a_2b_2+\dots+a_nb_n=a_1B_1-a_2(B_2-B_1)+a_3(B_3-B_2)+\dots+a_n(B_n-B_{n-1})=
  $
  
  $
  B_1(a_1-a_2)+B_2(a_2-a_3)+\dots+B_{n-1}(a_{N-1}-a_n)+a_n B_n=  $

  \[\sum_{i=1}^{n} a_ib_i = \sum_{i=1}^{n-1} (a_i-a_{i+1})B_i+a_nb_n\]
  - преобразование Абеля конечной суммы $\sum_{i=1}^{n}a_ib_i$
  \[a_1b_1+\sum_{i=2}^{n} a_ib_i=-\sum_{i=1}^{n-1}(a_{i+1}-a_i)B_i+a_nB_n\]

  \[\sum_{i=2}^{n} a_ib_i=a_nB_n-a_1B_1-\sum_{i=1}^{n-1}(a_{i+1}-a_i)B_i\]
  - дискретный аналог формулы интегрирования по частям

  \textbf{Лемма :} Если $\forall i = \overline{1,n-1}$ выполняется $a_i\leq a_{i+1}$ или $a_i\geq a_{i+1}$,
  а для $\forall k = \overline{1,n} \hspace{20pt}|b_1+\dots+b_k|\leq B$

  \begin{center}
    Тогда $|\sum_{i=1}^{n}a_ib_i| \leq B(|a_1|+2|a_n|)$
  \end{center}

  \underline{Доказательство:}
  \begin{adjustwidth}{1.5em}{1.5em}
    Рассмотрим $|\sum_{i=1}^{n} a_ib_i|\leq \sum_{i=1}^{n} |a_ib_i| = \sum_{i=1}^{n-1} |a_i-a_{i+1}||B_i|+|a_nb_n| \leq 
    B(\sum_{i=1}^{n-1} |a_i-a_{i+1}|+|a_n|)=$

    $B(|a_1-a_n|+|a_n|)\leq \Bigg|
    \begin{matrix}
      1) a_1-a_n+a_n\\
      2)a_n-a_1+a_n\\
      a_i\geq 0
    \end{matrix} \Bigg| \leq B(|a_1|+2|a_n|)$
  \end{adjustwidth}

  \begin{center}
    \textbf{Ч.т.д.}
  \end{center}

  \subsubsection*{Теорема 10.8.1 Признак Дирихле}\label{th:10.8.1}
  \par\noindent
  Если $\{a_n\}$ - монотонная и $\lim_{n \to \infty} a_n = 0,$ а последовательность $B_n=b_1+\dots+b_n$
  ограничена, то ряд $\sum_{n=1}^{\infty} a_nb_n$ - сх

  \underline{Доказательство:}
  \begin{adjustwidth}{1.5em}{1.5em}
    $\sqsupset |B_n| \leq B$ по условию
    
    Рассмотрим $|\sum_{k=0}^{p} b_{n+k}| = |B_{n+p}-B_{n-1}| \leq |B_{n+p}| + |B_{n-1}| \leq 2B$
    
    Рассмотрим $\lim_{n \to \infty} a_n =0 \Rightarrow \forall \varepsilon > 0 \exists n_0=n_0(\varepsilon): \forall n>n_0 \Rightarrow |a_n| < \frac{\varepsilon}{6B}$

    Рассмотрим $|\sum_{k=0}^{p} a_{n+k}+b_{n+k}| \leq 2B(|a_n|+2|B_{n+p}|) < 2B(\frac{\varepsilon}{6B}+\frac{2\varepsilon}{6B})=\varepsilon$

    Ряд сходится по критерию Коши
  \end{adjustwidth}

  \subsubsection*{Теорема 10.8.2 Признак Абеля}\label{th:10.8.2}
  \par\noindent
  Если $\{a_n\}$ монотонна и ограничена, а ряд $\sum_{n=1}^{\infty} b_n$ сходится, то ряд $\sum_{n=1}^{\infty} a_n b_n$ сходится.

  \underline{Доказательство:}
  \begin{adjustwidth}{1.5em}{1.5em}
    Рассмотрим $\{a_n\}$. Она ограниченна и монотонна $\Rightarrow \{a_n\}$ сходится,
    т.е. $\exists \lim_{n \to \infty} a_n = a \Rightarrow a_n = a+\alpha_n$, где $\lim_{n \to \infty} \alpha_n = 0$
    Так как $\{a_n\}$ монотонна $\Rightarrow \{\alpha_n\}$ - монотонна

    Пусть $\sum_{n=1}^{\infty} b_n$ сходится $\Rightarrow \{B_n\}$ - ограниченна
    
    Рассмотрим $\sum_{n=1}^{\infty} a_n b_n = \sum_{n=1}^{\infty} (a_n +\alpha_n)b_n=a \sum_{n=1}^{\infty} b_n + \sum_{n=1}^{\infty}\alpha_n b_n$ - сходится
  \end{adjustwidth}

  \begin{center}
    \textbf{Ч.т.д.}
  \end{center}

  \subsection{Функциональные последовательности и ряды. Основные понятия.}

  Пусть на некотором $\chi$ задана последовательность функций $f_n(x)$ где $n=1,2,\dots$, принимающая
  $\forall x \in \chi$ числовые значения(в общем случае Комплексные). Тогда говорят, что на $\chi$ задана 
  функциональная последовательность.
  
  \underline{Определение: } Последовательность $\{f_a(x)\}$ называется ограниченной на $\chi$, если 
  \[\exists C>0 \forall n \in N u \forall (\cdot) x \in \chi |f_n(x)| \leq C\]

  \underline{Определение: } Последовательность $\{f_n(x)\}$ называется сходящейся на множестве $\chi$, если
  при $\forall$ фиксируемом $x \in \chi$ числовая последовательность $\{f_n(x)\}$ сходится.

  \underline{Определение: }Если последовательность $\{f_n(x)\}$ сходится на множестве $\chi$, то функция $f(x)$
  определенная $\forall x \in \chi$, равная $f(x)=\lim_{n \to \infty} f_n(x)$ называется пределом последовательности
  $\{f_n(x)\}$

  Пусть на множестве $\chi$ задана последовательность $\{u_n(x)\}$

  \underline{Определение: } Множество всех числовых рядов $\sum_{n=1}^{\infty} u_n(x)$, в каждом из которых
  точка $x \in \chi$ фиксирована, называется функциональным рядом на множестве $\chi$

  $S_n(x)=\sum_{n=1}^{\infty}u_n(x)$-n-ая частичная сумма функционального ряда

  $\sum_{k=1}^{\infty} u_{n+k}(x)$-n-ый остаток функционального ряда.

  \underline{Определение: } Функциональный ряд называется сходящимся на множестве $\chi$, если
  последовательность $\{S_n(x)\}$ сходится на этом множестве, т.е. $\exists \lim_{n \to \infty} S_n(x)=S(x)$.\\
  $S(x)$- сумма ряда

  В этом случае $S(x) = \sum_{n=1}^{\infty}u_n(x)$ и говорят, что функция $S(x)$ раскладывается в ряд $\sum_{n=1}^{\infty} u_n(x)$

  \underline{Определение: } Если функция, ряд $\forall$ фиксируемом $x \in \chi$ сходится абсолютно, то он называется
  абсолютно сходящимся на $\chi$

  \underline{Пример:}
  \begin{enumerate}
    \item Рассмотрим $\sum_{n=1}^{\infty} \frac{x^n}{n!} x \in R$ \\%хех икс ин Р
    Рассмотрим $|u_n| = |\frac{x^n}{n!}| \hspace{20pt} |u_{n+1}|=\frac{x^{n+1}}{(n+1)!}$\\
    Рассмотрим $\lim_{n \to \infty} |\frac{u_{n+1}}{u_n}| = \lim_{n \to \infty} |\frac{x^{n+1}n!}{(n+1)!x^n}|=
    \lim_{n \to \infty}|\frac{x}{n+1}| = 0 < 1$ сходится $\forall x \in R$ по признаку Даламбера\\
    Тогда исходный ряд сходится абсолютно $\forall x \in R$
    \item Рассмотрим $x^2+\frac{x^2}{1+x^2}+\dots+\frac{x^2}{(1+x^2)^n}+\dots$\\
    $\sqsupset x=0 \hspace{20pt} S(0)=0$\\
    $\sqsupset x\not = 0 \hspace{20pt} \text{ Обозначим } q = \frac{1}{1+x^2} \hspace{20pt} 0<q<1$
    \[x^2q^0+x^2q^1+\dots+x^2q^n+\dots=x^2(1+q+a^2+\dots+q^n+\dots)=x^2\frac{1}{1-q}=x^2 \frac{1}{1-\frac{1}{1+x^2}}
    =\]
    \[= x^2  \frac{1+x^2}{1+x^2-1}=1+x^2, x \not = 0\]\\
    \pagebreak

    Изобразим график функции S(x)
    \begin{center}
      \includegraphics[scale=0.5]{10.9.1.png}
    \end{center}

    Сумма сходящихся (даже абсолютно сходящихся) ряда, все члены которого являются непрерывными функциями
    может оказаться разрывной функцией. Т.е. даже на абсолютно сходящиеся ряда не переносится свойство конечных
    сумм: сумма конечного числа непрерывного на множестве функций также непрерывно на этом множестве.
    Для описания рядов функций, на которые это свойство переносится, вводится понятие равномерно сходящегося
    ряда.
  \end{enumerate}
  

  \subsection{Равномерная сходимость функциональных последовательностей и рядов.}
  \underline{Определение: } Функция последовательность $f_n(x)$ называется равномерно сходящейся к функции 
  $f(x)$ на множестве $\chi$ если $\forall \varepsilon > 0 \exists n_0=n_0(\varepsilon): \forall x \in \chi$ и 
  $\forall n > n_0 \Rightarrow |f_n(x)-f(x)| <\varepsilon$

  \underline{Обозначение: }
  $\begin{matrix}
    f_n(x) \underset{\chi}{\rightarrow} f(x) \text{ - сходимость } \{f_n(x) \text{ к } f(x) \text{ на } \chi\}\\
    f_n(x) \underset{\chi}{\rightrightarrows} f(x) \text{ - равномерная сходимость } \{f_n(x) \text{ к } f(x) \text{ на } \chi\}
  \end{matrix}$

  Разница между обычной сходимостью и равномерной состоит в том, что при обычной сходимости для каждой $(\cdot)$
  $x \in \chi \exists$ свой номер $n_0=n_0(\varepsilon,x)\forall n>n_0 \Rightarrow |f_n(x)-f(x)|<\varepsilon$ и может оказаться
  , что для всех $x \in \chi$ невозможно подобрать общий $n_0$, обладающий указанным свойством.
  Равномерная сходимость функции последовательности на множестве $\chi$ означает, что такой номер 
  $n_0$ подобрать возможно.
  
  \textbf{Лемма :} Для того чтобы $\{f_n(x)\}$ равномерно сходилась на $\chi$ к функции $f(x) \Leftrightarrow
  \lim_{n \to \infty} \underset{\chi}{sup}|f_n(x)-f(x)|=0$
  
  \underline{Доказательство:}
  \begin{adjustwidth}{1.5em}{1.5em}
    \circled{$\Rightarrow$} Пусть $f_n(x) \underset{\chi}{\rightrightarrows} f(x)$. Зададим $\varepsilon > 0$
    . Тогда $\exists n_0 \forall x \in \chi$ и $\forall n>n_0 |f_n(x) - f(x)|<\varepsilon.$ Тогда это неравенство
    будет выполняться для $x \in \chi$, при котором левая часть будет принимать наибольшее значение
    $\underset{\chi}{sup}|f_n(x)-f(x)|<\varepsilon.$ Переходя к $\lim_{n \to \infty}$ получим
  \end{adjustwidth}
  \begin{center}
    \textbf{Ч.т.д.}
  \end{center}

  \begin{adjustwidth}{1.5em}{1.5em}
    \circled{$\Leftarrow$} Пусть $\lim_{n \to \infty} \underset{\chi}{sup} |f_n(x)-f(x)|=0$. Зафиксируем
    $\varepsilon>0$. Тогда по определению предел числовой последовательности $\exists n_0 =n_0(\varepsilon)
    \forall n>n_0 sup |f_n(x)-f(x)|<\varepsilon$. Т.е. 
    \[\underset{\chi}{sup}|f_n(x)-f(x)|<\varepsilon \Rightarrow \forall x \in \chi |f_n(x) -f(x)|<\varepsilon\]
    А это означает что $f_n(x) \underset{\chi}{\rightrightarrows} f(x)$
  \end{adjustwidth}
  \begin{center}
    \textbf{Ч.т.д.}
  \end{center}

  \textbf{Следствие:} Если $\exists \{\alpha_n\}: \lim_{n \to \infty} \alpha_n =0:\forall x \in \chi$\\
  \[|f_n(x)-f(x)|\leq \alpha_n, \text{ то } f_n(x) \underset{\chi}{\rightrightarrows} f(x)\]

  \underline{Доказательство:}
  \begin{adjustwidth}{1.5em}{1.5em}
    Так как $\forall x \in \chi |f_n(x)-f(x)|<\alpha_n$ тогда это неравенство выполнится при $ x \in \chi$
    при котором левая часть примет наибольшее значение, т.е. $\underset{\chi}{sup} |f_n(x)-f(x)|\leq \alpha_n$.
    Переходя к $\lim_{n \to \infty}$ получим что $\lim_{n \to \infty} \underset{\chi}{sup}|f_n(x)-f(x)|=0,$
    т.е. $f_n(x)\underset{\chi}{\rightrightarrows} f(x)$ по лемме.
  \end{adjustwidth}

  \begin{center}
    \textbf{Ч.т.д.}
  \end{center}

  \underline{Примеры:}
  \begin{enumerate}
    \item $f_n(x)=x^n \hspace{20pt} \chi = [0,q] \hspace{20pt} 0<q<1$\\
    Исследуем на равномерную непрерывность $\{f_n(x)\}$ на мн-ве $\chi$.\\
    \[\lim_{n \to \infty} f_n(x)= \lim_{n \to \infty} x^n =0 \hspace{20pt} \lim_{n \to \infty} \underset{\chi}{sup} |f_n(x)-f(x)|=
    \lim_{n \to \infty} q^n =0\]
    \[|f_n(x) -f(x)|=|x^n-0|=x^n \hspace{20pt} \text{Лемма выполнилась}\]
    \[\underset{\chi}{sup} |f_n(x)-f(x)|=\underset{\chi}{sup}x^n=q^n \hspace{20pt} f_n(x) \underset{x \in [0,q]}{\rightrightarrows} 0\]
    \item $f_n(x)=x^n \hspace{10pt} \chi \in [0;1) \hspace{20pt} \lim_{n \to \infty} \underset{\chi}{sup}|f_0(x)-f(x)|=\lim_{n \to \infty} 1 =1$
    \\ $\lim_{n \to \infty} f_n(x)=\lim_{n \to \infty}x^n=0$ \hspace{20pt} Лемма не выполнилась\\
    Рассмотрим $|f_n(x)-f(x)|=|x^n-0|=x^n$ \hspace{20pt} $f_n(x)\underset{\chi \in [0;1)}{\rightarrow} 0$\\
    $\underset{\chi}{sup} |f_n(x)-f(x)|=\underset{\chi}{sup \hspace{5pt}}x^n=1$
    \item $f_n(x)=x^n \hspace{20pt} \chi=[0;1]$\\
    $\lim_{n \to \infty} f_n(x) = \Bigg\{ 
      \begin{matrix}
        0, & x \not = 1\\
        1, & x = 1
      \end{matrix}$\\
      Так как $f_n(x)$ не сходится равномерно к $f(x)$ на [0,1), то $f_n(x)$ не будет сходится равномерно
      к $f(x)$ на [0,1], так как $[0,1) \subset [0,1]$
  \end{enumerate}

  \subsection{Критерий Коши равномерной сходимости функциональных \\ последовательностей и рядов.}

  \subsubsection*{Теорема 10.11.1 Критерий Коши равномерной сходимости функциональных последовательностей}\label{th:10.11.1}
  \par\noindent
  Для того, чтобы $\{f_n(x)\}$ равномерно сходилось на $\chi$ к $f(x) \Leftrightarrow$
  \[\forall \varepsilon > 0 \exists n_0=n_0(\varepsilon): \forall x \in \chi, \forall n > n_0, \forall p \geq 0 \Rightarrow |f_{n+p}(x)-f_n(x)|< \varepsilon\]

  \underline{Доказательство:}
  \begin{adjustwidth}{1.5em}{1.5em}
    \circled{$\Rightarrow$} Пусть $f_n(x) \underset{\chi}{\rightrightarrows} f(x).$ Зафиксируем $\varepsilon>0$. Тогда
    по определению равномерной сходимости функциональных последовательностей
    \[\exists n_0 \forall n>n_0 \forall x \in \chi |f_n(x)-f(x)|<\frac{\varepsilon}{2}\]
    \[\measuredangle  |f_{n+p}(x)-f_n(x)| = |f_{n+p}(x)-f(x)+f(x)-f_n(x)|\leq |f_{n+p}(x)-f(x)|+|f_n(x)-f(x)| < \varepsilon\]
    \circled{$\Leftarrow$} Пусть выполняется $|f_{n+p}(x)-f(x)|<\varepsilon$\\
    Тогда $\forall$ фиксированного $x \in \chi$ числовая последовательность удовлетворяет критерию Коши сходимости
    этой последовательности. Следовательно $\exists f(x) \lim_{n \to \infty}f_n(x)=f(x)$\\
    \[\text{ Т.е.} \forall x \in \chi \forall \varepsilon > 0 \exists n_0 \forall n>n_0 |f_n(x)-f(x)|<\varepsilon\]
    А это значит $f_n(x) \underset{\chi}{\rightrightarrows} f(x)$
  \end{adjustwidth}

  \underline{Определение: } Ряд $\sum_{n=1}^{\infty} u_n(x)$ называется равномерно сходящимся на $\chi$ если на этом
  множестве равномерно сходится последовательность его частичная сумма. Т.е. если 
  \[S(x)=\sum_{n=1}^{\infty}u_n(x),S_n(x)=\sum_{k=1}^{\infty}u_k(x), S_n(x)\underset{\chi}{\rightrightarrows} S(x)\]

  \[\text{ или } r_n(x)=S(x)-S_n(x)=\sum_{k=n+1}^{\infty}u_k(x) \text{ - n-ый остаток ряда.}\]
  
  Т.е. условие равномерной сходимости функционального ряда можно записать в трех вариантах

  \[S_n(x)-S(x) \underset{\chi}{\rightrightarrows}0\]
  \[r_n(x)\underset{\chi}{\rightrightarrows} 0\]
  \[\lim_{n \to \infty} \underset{\chi}{sup}|r_n(x)|=0\]

  \subsubsection*{Теорема 10.11.2 Необходимое условие равномерности сходимости функционального ряда}\label{th:10.11.2}
  \par\noindent
  Если ряд $\sum_{n=1}^{\infty} u_n(x)$ равномерно сходится на $\chi,$ то $u_n(x)\underset{\chi}{\rightrightarrows} 0$

  \underline{Доказательство:}
  \begin{adjustwidth}{1.5em}{1.5em}
    \[u_n(x)=S_n(x)-S_{n-1}(x)\]
    По условию ряд $\sum_{n=1}^{\infty} u_n(x)$ равномерно сходится на $\chi \Rightarrow S_n(x) \underset{\chi}{\rightrightarrows} S(x)$
    но будет выполняться $S_{n+1}(x)\underset{\chi}{\rightrightarrows} S(x)$. Тогда 
    \[S_n(x)-S{n-1}(x)\underset{\chi}{\rightrightarrows} 0 \Rightarrow u_n(x)\underset{\chi}{\rightrightarrows} 0\]

  \end{adjustwidth}
  \begin{center}
    \textbf{Ч.т.д.}
  \end{center}

  \subsubsection*{Теорема 10.11.3 Критерий Коши равномерной сходимости функционального ряда}\label{th:10.11.3}
  \par\noindent
  Для того чтобы $\sum_{n=1}^{\infty}u_n(x)$ равномерно сходился на $\chi \Leftrightarrow$
  \[\forall \varepsilon> 0 \exists n_0: \forall n>n_0 \forall x \in \chi \forall p \geq 0 \Rightarrow |u_n(x)+u_{n+p}(x)|<\varepsilon\]
  
  \underline{Доказательство:}
  \begin{adjustwidth}{1.5em}{1.5em}
    \circled{$\Rightarrow$} Пусть $\sum_{n=1}^{\infty} u_n(x)$ равномерно сходится на $\chi$ Тогда на этом множестве
    по определению
    $S_n(x)\underset{\chi}{\rightrightarrows} S(x),$ т.е.\\
    \[S_{n+p}(x)\underset{\chi}{\rightrightarrows} S(x) u S_{n-1}\underset{\chi}{\rightrightarrows} S(x),t.e. S_{n+p}(x)-S_{n-1}(x)
    \underset{\chi}{\rightrightarrows} 0 \overset{\hyperref[th:10.11.1]{Th 11.1}}{\Rightarrow} 
    |u_n(x)+\dots+u_{n+p}|<\varepsilon\]
    \[\forall x \in \chi \forall p \geq 0\]\\

    \circled{$\Leftarrow$} Пусть $|u_n(x)+\dots+u_{n+p}(x)|<\varepsilon \forall x \in \chi \forall p \geq 0$\\
    Тогда \[u_n(x)+\dots+u_{n+p}(x)=S_{n+p}(x)-S_{n-1}(x) u |S_{n+p}(x)+S_{n-1}(x)|<\varepsilon \forall x \in \chi
    \forall p\geq0 \Rightarrow\]
    $S_n(x)$ равномерно сходится на $\chi \underset{\hyperref[th:10.11.3]{\text{опр}}}{\Rightarrow}
    \sum_{n=1}^{\infty} u_n(x)$ равномерно сходится на $\chi$. \\
  \end{adjustwidth}

  \textbf{Свойство: } Если $\sum_{n=1}^{\infty} u_n(x)$ равномерно сходится на $\chi$, а $f(x)$
    ограничена на этом множестве, то $\sum_{n=1}^{\infty} f(x)u_n(x)$ также равномерно сходится на $\chi$
  \underline{Доказательство:}
  \begin{adjustwidth}{1.5em}{1.5em}
    \[\forall x \in \chi |f(x)|\leq C\]
    \[\forall p \geq 0 \forall x \in \chi \measuredangle |f(x)u_n(x)+f(x)u_{n+1}(x)+\dots+f(x)u_{n+p}(x)|
    =|f(x)(u_n(x)+\dots+u_{n+p}(x))| < C \cdot \varepsilon = \varepsilon_1\]
    Т.е. $\sum_{n=1}^{\infty} f(x)u_n(x)$ удовлетворяет критерию Коши равномерной сходимости ряда на $\chi$
  \end{adjustwidth}
  \begin{center}
    \textbf{Ч.т.д.}
  \end{center}

  \subsubsection*{Теорема 10.11.4 Признак Вейерштрасса равномерной сходимости функционального ряда}\label{th:}
  \par\noindent
  Если числовой ряд $\sum_{n=1}^{\infty} \alpha_n, \alpha_n \geq 0$ сходится и $\forall x \in \chi$ и 
  $\forall n \in N$ выполняется $|u_n(x)|\leq \alpha_n$, то ряд $\sum_{n=1}^{\infty} u_n(x)$ сходится 
  абсолютно равномерно на $\chi$.

  \underline{Доказательство:}
  \begin{adjustwidth}{1.5em}{1.5em}
    По условию $\sum_{n=1}^{\infty} \alpha_n$ сходится, тогда так как 
    $\forall x \in \chi$ и $\forall n \in N |u_n(x)|\leq \alpha_n \Rightarrow \sum_{n=1}^{\infty} |u_n(x)|$
    сходится по I признаку сравнения\\
    Тогда $\sum_{n=1}^{\infty} u_n(x)$ абсолютно сходится. Так как числовой ряд $\sum_{n=1}^{\infty} \alpha_n$
    сходится то \[\exists n_0:\forall n>n_0 \sum_{k=n+1}^{\infty}\alpha_k < \varepsilon (\lim_{n \to \infty} r_n =0)\]
    \[\measuredangle |r_n(x)|=|\sum_{k=n+1}^{\infty} u_k(x)| \leq \sum_{k=n+1}^{\infty} \alpha_k<\varepsilon\]
    Т.е.
    \[\forall x \in \chi \forall n >n_0 |r_n(x)| < \varepsilon \Rightarrow r_n(x) \underset{\chi}{\rightrightarrows}
    0 \Rightarrow \sum_{n=1}^{\infty} u_n(x) \text{ равномерно сходится на } \chi\]
  \end{adjustwidth}
  \subsection{Свойства равномерно сходящихся последовательностей}
    \subsection*{Свойство 1(формулировка для функциональных рядов)}
    Если функция $u_n(x)$ непрерывна в $(\cdot) x_0 \in \chi$ и 
    ряд $\sum_{n=1}^{\infty} u_n(x)$ равномерно сходится на $\chi$, то его сумма
    $S(x)=\sum_{n=1}^{\infty}u_n(x)$ также непрерывна в $(\cdot) x_0$   

    \underline{Доказательство:}
    \begin{adjustwidth}{1.5em}{1.5em}
      Зафиксируем $\varepsilon >0$. Пусть $S_n(x)=\sum_{k=1}^{n} u_k(x)$-частичные суммы ряда.
      Так как $\sum_{n=1}^{\infty} u_n(x)$ равномерно сходится на $\chi \Rightarrow S_n(x) \underset{\chi}{\rightrightarrows}S(x)$
      Это значит, что
      \[\exists n_0 \forall n>n_0 \forall x \in \chi \Rightarrow |S(x)-S_n(x)|< \dfrac{\varepsilon}{3}\]
      Это значит, что будет выполняться 
      \[|S(x_0)-S_n(x_0)|< \dfrac{\varepsilon}{3} \text{ т.к. } x_0 \in \chi\]
      Рассмотрим $S_n(x)=\sum_{k=1}^{n} u_k(x)$ - конечная сумма непрерывной в $(\cdot) x_0$
      функции. Значит $S_n(x)$ непрерывна $\forall x \in \chi,$ т.е. и в $(\cdot) x_0 \in \chi$.
      Т.е. по определению Коши непрерывности функции в ($\cdot$) получим
      \[\forall \varepsilon > 0 \exists \delta = \delta(\varepsilon) >0: \forall x:|x-x_0|< \delta \Rightarrow |S_n(x)-S_n(x_0)|<\dfrac{\varepsilon}{3} \]
      Рассмотрим $|S(x)-S(x_0)| = |S(x)-S_n(x_0)+S_n(x_0)-S_n(x)+S_n(x)-S(x_0)|\leq
      |S(x)-S_n(x)|+|S_n(x)-S_n(x_0)|+|S_n(x_0)-S(x_0)|\leq \dfrac{\varepsilon}{3}+\dfrac{\varepsilon}{3}+\dfrac{\varepsilon}{3}=\varepsilon$\\
      Это значит, что $\lim_{x \to x_0} S(x)=S(x_0)$ Т.е. по определению функция непрерывна в $(\cdot) x_0$
    \end{adjustwidth}
    \begin{center}
      \textbf{Ч.т.д.}
    \end{center}

    \subsection*{Формулировка свойства в терминах последовательностей}
    Если последовательность $f_n(x) x \in \chi$ равномерно сходится на этом множестве
    к функции $f(x)$ и $\forall n \in N f_n(x)$ непрерывна в $\cdot x_0 \in \chi,$ то $f(x)$ непрерывна
    в $(\cdot) x_0$

    \subsection*{Свойство 2}
    Пусть функция $u_n(x) x \in [a;b]$ непрерывна на этом отрезке и ряд $\sum_{n=1}^{\infty} u_n(x)$ равномерно
    сходится на $[a,b]$. Тогда какова не была бы $(\cdot) x_0 \in [a;b]$ ряд $\sum_{n=1}^{\infty} \int^x_{x_0} u_n(t)
    dt^*$ также равномерно сходится на отрезке $[a;b]$ и $\int^x_{x_0} (\sum_{n=1}^{\infty}u_n(t))dt=
    \sum_{n=1}^{\infty}\int^x_{x_0}u_n(t)dt$(ряд можно почленно интегрировать)

    \subsection*{Свойство 2 в терминах последовательностей}
    Если $f_n(x)$ непрерывна $\forall x \in [a;b]$ и последовательность $\{f_n(x)\}$ равномерно сходится на 
    $[a;b]$ к функции $f(x)$ то какова бы не была $(\cdot) x_0 \in [a;b]$ последовательность
    $\int_{x_0}^x f_n(t)dt=\int_{x_0}^{x} \lim_{n \to \infty} f_n(t)dt = \int_{x_0}^{x} f(t)dt$

    \subsection*{Свойство 3} 
    Пусть функции $u_n(x)$ непрерывно дифференцируемы на $[a;b]$ и ряд $\sum_{n=1}^{\infty} u_n'(x)$ равномерно
    сходится на $[a;b]$. Тогда если ряд $\sum_{n=1}^{\infty} u_n(x)$ сходится хотя бы в одной
    $(\cdot) x_0 \in [a;b]$ то он равномерно сходится на $[a;b]$ и его сумма $S(x)=\sum_{n=1}^{\infty}u_n(x)$
    является непрерывно дифференцируемой функцией и $S(x)=\sum_{n=1}^{\infty}u'_n$ или
    $(\sum_{n=1}^{\infty}u_n(x))'=\sum_{n=1}^{\infty}u_n'(x)$

    \subsection*{Свойство 3 в терминах последовательностей}
    Если последовательность непрерывно дифференцируемых функций $\{f_n(x)\}$ на $[a;b]$ сходится в 
    некоторой $(\cdot) x_0 \in [a;b], a \{f_n'(x)\}$ равномерно сходится на $[a;b]$ к некоторой функции
    $\varphi(x),$ то $\{f_n(x)\}$ равномерно сходится на $[a;b]$ к непрерывной дифференцируемой функции $f(x)$ и
    $f'(x)=\varphi(x)$

    \subsection{Степенные ряды}

    \underline{Определение: } Степенным рядом называется ряд вида 
    $\begin{matrix}
      \sum_{n=0}^{\infty} a_n(x-x_0)^n (*)\\
      \sum_{n=0}^{\infty} a_n x^n (**) - \text{ частный случай функц. ряда}
    \end{matrix}$

    \subsubsection*{Теорема 10.13.1 Теорема Абеля}\label{th:10.13.1}
    \par\noindent
    \begin{enumerate}
      \item Если $\sum_{n=0}^{\infty} a_n x^n (**)$ сходится в некоторой $(\cdot) x=x_0,$ То
      \[\forall x: |x|< |x_0|(**)\text{ сходятся абсолютно}\]
      \item Если (**) рсх в некоторой $(\cdot) x=x_0,$ то
      \[\forall x: |x|>|x_0| (**)\text{ также рсх}\]
    \end{enumerate}

    \underline{Доказательство:}
    \begin{adjustwidth}{1.5em}{1.5em}
      \begin{enumerate}
        \item Пусть $\sum_{n=0}^{\infty}a_n x^n_0$ сходится $\Rightarrow$ необходимый признак выполняется,
        т.е. $\lim_{n \to \infty} a_n x_0^n=0$. Так как числовая последовательность сходится $\Rightarrow$ она ограничена.
        Это значит \[\exists c: |a_n x_0^n|\leq C \forall n=0,1,2\]
        Рассмотрим $|a_n x^n|=|a_n x^n|\cdot |\frac{x_0}{x_0}|^n = |a_n x_0^n|\cdot |\frac{x}{x_0}|^n\leq C |\frac{x}{x_0}|^n=Cq^n \hspace{20pt} q= |\frac{x}{x_0}|$\\
        Рассмотрим $\sum_{n=0}^{\infty} q^n$ сходится как б.у. геометрическая прогрессия $\Rightarrow \sum_{n=0}^{\infty}
        C q^n сходится \Rightarrow \sum_{n=0}^{\infty} |a_n x^n|$ сходится по I признаку сравнения.
        Тогда $\sum_{n=0}^{\infty} a_n x^n$ сходится абсолютно
        \item  Пусть в $\cdot x=x_0(**)$ рсх\\
        Тогда при $|x|>|x_0|$ он не может сходится в $(\cdot)x$, так как если бы он сходился в 
        $(\cdot)x$, то по п.1 он сходился бы в $(\cdot) x=x_0$, а это не так. Т.е. при $|x|>|x_0|$ ряд (**) рсх
      \end{enumerate}
    \end{adjustwidth}
    \begin{center}
      \textbf{Ч.т.д.}
    \end{center}

    \textbf{Обозначим}\\
    $\chi$ - множество всех действительных неотрицательных x', при которых (**) сходится. Обозначим $R=\underset{\chi}{\sup}
    \chi 0 \leq R \leq \infty$.\\
    Если $R>0$ и $|x|<R$, тогда $\exists x':|x|<x'<R$\\
    В $(\cdot) x=x'(**)$ сходится $\Rightarrow$ по th. Абеля он будет сходится абсолютно в $(\cdot) x$,
    т.е. получим $\forall x: |x|<R$ ряд сх\\
    Если $R< \infty$ и $|x|>R$ тогда $\exists x': R<x'<|x|$\\
    В $(\cdot) x'(**)$ рсх $\Rightarrow$ по th Абеля (**) будет рсх $\forall x: |x|>R$.\\
    Аналогично, если рассмотреть $\sum_{n=0}^{\infty}a_n(x-x_0)^n(*),$ то
    $\begin{matrix}
      \forall x:|x-x_0|<R \text{ ряд сх}\\
      \forall x:|x-x_0|>R \text{ ряд рсх}
    \end{matrix}$ \\

    \underline{Определение: } Число $R\geq 0$ называется радиусом сходимости ряда $\sum_{n=0}^{\infty} a_n(x-x_0)^n$,
    \\
    если 
    $\begin{matrix}
      \forall x:|x-x_0|<R \text{ ряд сх}\\
      \forall x:|x-x_0|>R \text{ ряд рсх}
    \end{matrix}$. Интервал вида $(x_0-R;x_0+R)$ называется интервалом сходимости ряда (*). Аналогично (-R;R)-
    интервал сходимости ряда (**)

    \subsubsection*{Теорема 10.13.2}\label{th:10.13.2}
    \par\noindent
    Для любого степенного ряда $\exists$ радиус сходимости $0\leq R\leq\infty$ при
    этом если $|x-x_0|<R$, то $(\cdot) x$ ряд сходится абсолютно,
    а если $0<r<R,$ то $\forall x:|x-x_0|\leq r$ ряд сходится равномерно.
    
    \underline{Доказательство:}
    \begin{adjustwidth}{1.5em}{1.5em}
      \begin{enumerate}
        \item $R=\underset{\chi}{\sup}\chi$
        \[|x|<R \text{ ряд абсолютно сходится по th 13.1(th Абеля)}\]
        \item Пусть $x=r$, тогда 
        \[|a_nx^n|\leq|a_nr^n|, \measuredangle \sum_{n=0}^{\infty} |a_nr^n|(1)\text{ сх } (r<R,R=\underset{\chi}{\sup}\chi)\]
        Тогда $\sum_{n=0}^{\infty} |a_n x^n|$ сходится равномерно по правилу Вейерштрасса
      \end{enumerate}
    \end{adjustwidth}
    
    \begin{center}
      \textbf{Ч.т.д.}
    \end{center}

    \subsection{Определение радиуса сходимости степенного ряда}
    Рассмотрим $\sum_{n=0}^{\infty}a_nx^n$. Рассмотрим $\sum_{n=0}^{\infty} |a_nx^n|$

    \begin{enumerate}
      \item Воспользуемся признаком Даламбера\\
      Рассмотрим
      \[\lim_{n \to \infty} |\frac{a_{n+1}x^{n+1}}{a_nx^n}|=\lim_{n \to \infty} |\frac{a_{n+1}}{a_n}||x|=
      \Big| \sqsupset \lim_{n \to \infty}\frac{a_{n+1}}{a_n}=q\Big|=
      \begin{matrix}
        ю\\
        q|x|<1\\
        ю\\
        |x|<\frac{1}{q} \hspace{5pt} R=\frac{1}{q}\Rightarrow R=\lim_{n \to \infty} |\frac{a_n}{a_{n+1}}|
      \end{matrix}\]

      \item Воспользуемся радикальным признаком Коши\\
      Рассмотрим 
      \[\lim_{n \to \infty} \sqrt[n]{|a_nx^n|}=\lim_{n \to \infty}|x|q\circled{=} q=\sqrt[n]{a_n}\circled{=}
      \begin{matrix}
        ю\\
        q|x|<1\\
        ю\\
        |x|<\frac{1}{q}= R
      \end{matrix}\]
      \[R=\lim_{n \to \infty} \frac{1}{\sqrt[n]{a_n}}\]
    \end{enumerate}

    \subsubsection*{Теорема 10.14.1}\label{th:10.14.1}
    \par\noindent
    Пусть $F(x)=\sum_{n=0}^{\infty} a_n(x-x_0)^n(1)$. Если функция $f(x)$ раскладывается в $u(x_0)$
    в степенной ряд с радиусом сходимости R>0, то
    \begin{enumerate}
      \item функция $f(x) $ имеет на $(x_0-R;x_0+R)$ производные всех порядков которые находятся
      почленным дифференцированием.
      \[f^{(m)}(x)=\sum_{n=m}^{\infty} n(n-1)(n-m+1)a_nx^{n-m}(2)\]
      \item \[\forall x \in (x_0-R;x_0+R) \int_{x_0}^{x}f(t)dt=\sum_{n=0}^{\infty} \dfrac{a_n}{n+1}(x-x_0)^{n+1}(3)\]
      Т.е. (1) можно почленно интегрировать на $(x_0-R;x_0+R)$
      \item Ряды (1),(2),(3) имеют одинаковые радиусы сходимости, $т.е. R=R_1=R_2$
    \end{enumerate}

    \subsection{Формула Тейлора}
    \subsubsection*{Теорема 10.15.1}\label{th:10.15.1}
    \par\noindent
    Если f(x) раскладывается в некоторой окрестности $(\cdot) x_0$ в ряд $f(x)=\sum_{n=0}^{\infty}a_n(x-x_0)^n$,
    то $a_n=\frac{f^{(n)}(x_0)}{n!}$ и справедлива формула
    $f(x)=\sum_{n=0}^{\infty} \frac{f^{(n)}(x_0)}{n!}(x-x_0)^n$

    \underline{Определение: } Пусть действительная функция $f(x)$ определена в некоторой $u(x_0)$ и
    имеет в этой $(\cdot)$ производные всех порядков. Тогда ряд $\sum_{n=0}^{\infty} \frac{f^{(n)}(x_0)}{n!}
    (x-x_0)^n$ называется рядом Тейлора. Если $x_n=0$, то называется рядом Маклорена.\\

    \underline{Замечание(!!!):}
    Любая бесконечно дифференцируемая функция в некоторой $(\cdot)$ раскладывается в этой $(\cdot)$ в ряд Тейлора.
    Но может случиться так, что она не будет равна сумме своего ряда ни в одной $(\cdot)$ некоторой $u(x_0)$

    \subsection*{Сходимость рядов Тейлора}
    Пусть $f(x)$ - бесконечно дифференцируемая в $(\cdot) x_0$ функция и 
    \[\sum_{n=0}^{\infty} \frac{f^{(n)}(x_0)}{n!}(x-x_0)^n \text{ - ее ряд Тейлора}\]
    \[S_n(x)=\sum_{k=1}^{n} \frac{f^{(k)}}{k!}(x-x_0)^k\text{ - n-ая частичная сумма}\]
    \[r_n(x)=f(x)-S_n(x)\]
    
    \textbf{Утверждение:}Для того, чтобы $f(x)$ равнялась сумме своего ряда Тейлора в некоторой
    $u(x_0)$ необходимо чтобы $\lim_{n \to \infty}r_n(x)=0$

    \subsection*{Запись остаточного члена формулы Тейлора}
    \begin{enumerate}
      \item Формы Лагранжа
      \[r_n(x)=\frac{f^{(n+1)}(\xi)}{(n+1)!}(x-x_0)^{n+1}\]
      \item Форма Теано
      \[r_n(x)=o((x-x_0)^n), x \to x_0\]
      \item Интегральная форма
      \[r_n(x)=\frac{1}{n!}\int_{x_0}^{x}f^{(n+1)}(t)(x-t)^ndt\]
      \item Форма Коши
      \[r_n(x)=\frac{f^{(n+1)}(x_0+\theta(x-x_0))}{n!}\]
    \end{enumerate}

    \underline{Доказательство: п.3}
    \begin{adjustwidth}{1.5em}{1.5em}
      \begin{enumerate}
        \item n=0
        \[f(x)=f(x_0)+r_0(x)=f(x_0)+\int_{x_0}^{x}f'(t)dt \circled{=}f(x_0)+f'(\xi)(x-x_0)\]
        \[\circled{=} f(x_0)-\int_{x_0}^{x}f'(t)d(x-t)=\Bigg| 
        \begin{matrix}
          u=f'(t) & du =f''(t)dt\\
          dv=d(x-t) & V=x-t
        \end{matrix}\Bigg| =\]
        \[= f(x_0)-(f'(t)(x-t)|^x_{x_0}-\int_{x_0}^{x}f''(t)(x-t)dt)=
        f(x_0)+f'(x_0)(x-x_0)+\int_{x_0}^{x}f''(t)(x-t)dt\]
        \item Предположение индукции\\
        Пусть формула верна для некоторого $n=m=1$. Т.е. пусть справедливо $f(x)=\sum_{k=0}^{m=0}
        \frac{f^{(k)}(x_0)}{k!}(x-x_0)^k+\frac{1}{(m+1)!}\int_{x_0}^{x}f^{(m)}(t)(x-t)^{m-1}dt$

        \item Шаг индукции\\
        Докажем справедливость формулы для $n=m$.\\
        \underline{Доказательство:}
        \begin{adjustwidth}{1.5em}{1.5em}
          Рассмотрим $\frac{1}{(m-1)!}\int_{x_0}^{x}f^{(m)}(x-t)^{m-1}dt=-\frac{1}{(m-1)!}-\frac{1}{m}
          \int_{x_0}^{x}f^{(m)}(t)d((x-t)^m)=
          -\frac{1}{m!} \int_{x_0}^{x}f^{(m)}(t)d((x-t)^n)=\Bigg|
          \begin{matrix}
            u=f^{(m)}(t) & du =f^{(m+1)}(t)dt\\
            dv=d((x-t)^m) & V=(x-t)^m
          \end{matrix} \Bigg|=$
          \\ $=-\frac{1}{m!}(f^{(m)}(t)(x-t)^m|^x_{x_0}-\int_{x_0}^{x}f^{(m+1)}(t)(x-t)^m dt)=
          \frac{1}{m!}f^{(m)}(x_0)(x-x_0)^m+\frac{1}{m!}\int_{x_0}^{x}f^{(m+1)}(t)(x-t)^m dt$
        \end{adjustwidth}
      \end{enumerate}
    \end{adjustwidth}
    \begin{center}
      \textbf{Ч.т.д.}
    \end{center}
    \pagebreak

    \underline{Доказательство: п.4}
    \begin{adjustwidth}{1.5em}{1.5em}
      Рассмотрим $r_n(x)$ в интегральной форме
      \[r_n(x)=\frac{1}{n!}\int_{x_0}^{x}f^{(n-1)}(t)(x-t)^n dt = \frac{f^{(n+1)}(\xi)(x-\xi)^n}{n!}(x-x_0) \circled{=}\]
      (*)$\xi=x_0+\theta(x-x_0)$\\
      Вычтем левую и правую часть * из x
      \[\begin{matrix}
        x-\xi=(x-x_0)-\theta(x-x_0)\\
        x-\xi=(x-x_0)(1-\theta)
      \end{matrix}
      \circled{=}\dfrac{f^{(n+1)}(x_0+\theta(x-x_0))(x-x_0)^n (1-\theta)^n}{n!}(x-x_0)=\]
      \[= \dfrac{f^{(n+1)}(x_0+\theta(x-x_0))}{n!}(1-\theta)^n(x-x_0)^{n+1}\]
    \end{adjustwidth}
    \begin{center}
      \textbf{Ч.т.д.}
    \end{center}

    \subsubsection*{Теорема 10.15.2 Достаточно условие разложимости функции в ряд Тейлора}\label{th:10.15.2}
    \par\noindent
    Если функция в окрестности $(\cdot) x_0$ имеет все производные ограниченные в этой окрестности, то функция
    $f(x)$ раскладывается в степенной ряд в окрестности $(\cdot) x_0$.

    \underline{Доказательство:}
    \begin{adjustwidth}{1.5em}{1.5em}
      Пусть $f(x)$ на $(x_0-h;x_0+h)$ имеет производные всех порядков и существует.
      \[C>0:\forall n =0,1,2,\dots |f^{(n)}(x)|\leq C\]
      \underline{Замечание:} 
      \[\lim_{n \to \infty} \frac{a^n}{n!}=0\]
      Рассмотрим $\sum_{n=0}^{\infty}\frac{a^n}{n!}$. Рассмотрим $\lim_{n \to \infty} \frac{a^{n+1}n!}{(n+1)!a^n}=
      \lim_{n\to\infty} \frac{a}{n+1}=0<1 \Rightarrow (1)$ сходится по признаку Даламбера $\Rightarrow$
      необходимый признак выполняется $\Rightarrow \lim_{n \to \infty }\frac{a^n}{n!}=0$.\\
      Рассмотрим $|r_n(x)|=|\frac{f^{(n+1)}(\xi)}{(n+1)!}(x-x_0)^{n+1}|\leq C |\frac{(x-x_0)^{n+1}}{(n+1)!}|$
      \[\lim_{n \to \infty} |r_n(x)|=\lim_{n \to \infty} |C \frac{(x-x_0)^{n+1}}{(n+1)!}|= 0\]
      Т.е. степенной ряд сходится
    \end{adjustwidth}

    \begin{center}
      \textbf{Ч.т.д.}
    \end{center}

    \subsection{Разложение функций в ряд Тейлора}
    \begin{enumerate}
      \item Используя формулу суммы членов б. у. геометрической прогрессии
      \[a+aq+aq^2+\dots+aq^n+\dots=\sum_{n=0}^{\infty}aq^n=\frac{a}{1-q}\]
      \[1. \frac{1}{1-x}=1+x+x^2+\dots+x^n+\dots\]
      \[2. \frac{1}{1-x^2}=\frac{1}{1-(-x^2)}=1-x^2+x^4-x^6+\dots=\sum_{n=0}^{\infty}(-1)^n x^{2n} |x|\leq 1\]
      \[3. ln(1+x)=\int_{0}^{x}\frac{dt}{1+t}=\int_{0}^{x}(1-t+t^2-t^3+\dots)dt=t|^x_0-\frac{t^2}{2}|^x_0+\frac{t^3}{3}|^x_0-\dots=\]
      \[x-\frac{x^2}{2}+\frac{x^3}{3}-\dots+(-1)^{n+1}\frac{x^n}{n}+\dots \hspace{20pt} x \in (-1;1)\]

      \item Используя формулу Тейлора непосредственно
      \[e^x=\sum_{n=0}^{\infty}\frac{x^n}{n!}\]
      \[\sin(x)=\sum_{n=1}^{\infty}(-1)^{n+1}\frac{x^{2n-1}}{(2n-1)!}\]
      \[\cos(x)=\sum_{n=0}^{\infty}(-1)^n \frac{x^{2n}}{(2n)!}\]

      \item Используя известные разложения функций
      \[e^x=\sum_{n=0}^{\infty} \frac{x^{2n}}{n!}\]

      \item Биноминальный ряд
      \[(1+x)^m=1+mx+\frac{m(m-1)}{2!}x^2+\dots+\frac{m(m-1)\dots(m-n+1)}{n!}x^n+\dots\]
    \end{enumerate}

    \underline{Замечание:} Если $m \in N$, то биноминальный ряд имеет конечное число слагаемых.

    Исследуем ряд на сходимость по признаку Даламбера:
    \[\lim_{n \to \infty} |\frac{u_{n+1}}{u_n}|=
    \lim_{n \to \infty}|\frac{m(m-1)\dots(m-n+1)(m-n)x^{n+1}n!}{(n+1)!m(m-1)\dots(m-n+1)x^n}|=
    \lim_{n \to \infty} |\frac{(m-n)x}{n+1}|=|x|<1\]

    \subsection*{Приближенное вычисление функций}
    Задача: вычислить $\approx f(x)$ в $(\cdot) x=x_0$\\
    
    $
     f(x_0)=\sum_{k=0}^{n} a_k x_0^k+r_n(x_0) \hspace{30pt}
     f(x_0) \approx \sum_{k=0}^{n} a_k x_0^k 
    $

    \underline{Замечание:} Для знакочередующегося ряда\\
    $
    |S_n-S|\leq u_{n+1}\\
    |r_n(x)|\leq u_{n+1}
    $
    
    
\end{document}