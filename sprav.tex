\documentclass[12pt]{article}
\usepackage[a4paper, total={7in,10in}]{geometry}
\usepackage{polyglossia}
\usepackage{ragged2e}
\usepackage{amsmath}
\usepackage{amssymb}
\usepackage{microtype}
\usepackage{graphicx}
\let\ORIincludegraphics\includegraphics
\renewcommand{\includegraphics}[2][]{\ORIincludegraphics[scale=0.65,#1]{#2}}
\usepackage{changepage}
\usepackage{hyperref}
\usepackage{cancel}
\graphicspath{{./images/}}
\setmainlanguage{russian}
\setotherlanguage{english}
\newfontfamily\russianfont[Script=Cyrillic]{Times New Roman}
\newfontfamily\englishfont{Times New Roman}
\setlength{\parindent}{0em}
\setlength{\parskip}{6pt}

\def\posl#1#2{\{#1_{#2}\}}
\DeclareMathOperator*{\sh-like}{\sinh-like}
\DeclareMathOperator*{\ch-like}{\cosh-like}
\DeclareMathOperator*{\th-like}{\tanh-like}
\DeclareMathOperator*{\cth-like}{\coth-like}
\DeclareMathOperator*{\tg-like}{\tan-like}
\DeclareMathOperator*{\ctg-like}{\cot-like}
\DeclareMathOperator*{\arctg-like}{\arctan-like}
\DeclareMathOperator*{\arcctg-like}{\arctan-like}

\setcounter{section}{7}

\begin{document}
    \justifying
    \begin{titlepage}
        \begin{center}
            \hfill \break
            \large{МИНОБРНАУКИ РОССИИ}\\
            \footnotesize{ФЕДЕРАЛЬНОЕ ГОСУДАРСТВЕННОЕ АВТОНОМНОЕ ОБРАЗОВАТЕЛЬНОЕ УЧРЕЖДЕНИЕ}\\ 
            \footnotesize{ВЫСШЕГО ПРОФЕССИОНАЛЬНОГО ОБРАЗОВАНИЯ}\\
            \small{\textbf{«Дальневосточный федеральный университет»}}\\
            \hfill \break
            \normalsize{ИНСТИТУТ МАТЕМАТИКИ И КОМПЬЮТЕРНЫХ ТЕХНОЛОГИЙ}\\
             \hfill \break
            \normalsize{Департамент программной инженерии и искусственного интеллекта}\\
            \hfill\break
            \hfill \break
            \hfill \break
            \hfill \break
            \large{Лекции 2 курса по дисциплине}\\
            \large{Математический Анализ}\\
            \hfill \break
            \hfill \break
            \hfill \break
            \begin{flushright}
              Подготовлено студентами гр.\\
              Б9123-02.03.03тп\\
              Макевкин C.C.\\
              Тарасенко Т.В.\\
              \hfill \break
            \end{flushright}
            \begin{center}
                Для\\
              Зиновьева Павла Владимировича
            \end{center}
            \hfill \break
            
            \hfill \break
            
            \hfill \break
            \hfill \break
            \end{center}
             
            \hfill \break
             
            \normalsize{ 
            
            }
            \begin{center} Владивосток \\ 2025 \end{center}
            \thispagestyle{empty}
    \end{titlepage}
    \pagebreak
    \tableofcontents
    \pagebreak
    \section{Криволинейные интегралы первого рода}
    \underline{Определение: } Кривая $\overline{r}(t)=x(t)\overline{i}+y(t)\overline{j}+z(t)\overline{c}$ \;$a\leq t \leq b$
    Называется непрерывной гладкой кривой, если x(t),y(t),z(t) непрерывна диф-мы на [a;b] и
    $x(t)^2+y(t)^2+z(t)^2\not = 0$\\
    \underline{Определение: } Кривая называется непрерывной кусочно-гладкой прямой, если она состоит из
    конечного числа гладких прямых  \\
    %%пикча надо нарисовать
    $\exists $кривая имеет массу $\rho=\frac{kg}{m}$
\end{document}